\documentclass{book} %preprint, review, authoryear
%\documentclass[3p,number,sort&compress,twocolumn]{elsarticle} 
\usepackage{titlesec}
\usepackage[colorlinks=true]{hyperref}
\usepackage{parskip}
\usepackage[utf8]{inputenc}
\usepackage{verbatim}
\usepackage[font=footnotesize,labelfont=bf]{caption}
\usepackage{units}
\usepackage{graphicx}
\usepackage{tikz}
\usepackage{siunitx}
\usepackage{makeidx}

\makeindex

\sisetup{round-mode = places}% in preamble    

\usepackage{fancyhdr}
\fancypagestyle{plain}{%
  \fancyhf{}
  \fancyhead[L]{\sc Barnes}
  \fancyhead[R]{\sc Networks Survey}
  \cfoot{\thepage}
  \renewcommand{\headrulewidth}{0pt}
  \renewcommand{\footrulewidth}{0pt}}
\pagestyle{plain}

\newcommand{\tscore}[1]{
\raisebox{-.5\height}{
\begin{tikzpicture}
\draw (0,0)--(1,0);
\filldraw[black] (#1,0) circle (0.05);
\node at (0.5,-0.2) {\tiny \num[round-precision=2]{#1}};
\end{tikzpicture}
}
}

\titleformat{\chapter}[display]
  {\normalfont\huge\bfseries}{}{0pt}{\Huge}

\newcommand{\imtitle}[1]{\begin{center}#1\index{#1}\end{center}}
\newcommand{\bchap}[1]{\chapter{#1}}

\newcommand{\todo}[1]{\footnote{#1}}

\hypersetup{
  pdfauthor   = {Richard Barnes (ORCID: 0000-0002-0204-6040)},
  pdftitle    = {District Census Networks},
  pdfkeywords = {geographic information system (GIS), TODO},
  pdfproducer = {LaTeX},
  pdfcreator  = {pdfLaTeX}
}

%Favour layouts with figures tightly packed in
\renewcommand{\topfraction}{0.85}
\renewcommand{\textfraction}{0.1}
\renewcommand{\floatpagefraction}{0.85}

\newcommand{\ssuf}[3]{#1#2$^{#3}$}

\title{District Census Networks: A Survey}
\author{Richard Barnes \\ rbar@berkeley.edu}
\date{}

\begin{document}
\thispagestyle{empty}

\maketitle

\tableofcontents

\newpage

\chapter{Introduction}

\section{Methodology}

In the following, a graph representation of neighbouring political units is extracted from census data. For each U.S.\ Congressional District, images are grouped into two rows of three columns. The columns, from left to right, are representations based on Census Tracts, Census Block Groups, and Census Blocks. The first row shows the political units, along with the boundary of their parent Congressional District. The second row shows only the graph representation.

Since this document is a draft, I have in many cases omitted the Block representation, since plotting this is computationally expensive. Similarly, some Congressional Districts and states are missing entirely, presumably because my batch download processes did not acquire the needed source data.

Data used to make this document were drawn from the United States Census
Bureau's Cartographic/Line 2010 Shapefiles available at
\url{https://www.census.gov/geo/maps-data/data/tiger-line.html}.

Subunits (tracts, block groups, blocks) were paired with one or more parents (congressional districts) by first filtering by minimum bounding box overlaps and then calculating a polygon intersection.

Neighbouring subunits were found by expanding subunit minimum bounding boxes by a small margin, filtering for overlapping boxes. The subunit borders were then densified and it was determined whether the subunits had a pair of points which were within a small distance of each other.

Algorithms are available in CompactnessLib (\url{https://github.com/r-barnes/compactnesslib}).

\newpage

\bchap{Alabama}
\begin{minipage}{\columnwidth}
\begin{tabular}{lccc}
\raisebox{-0.5\height}{\includegraphics[width=0.32\columnwidth]{tract-0101-full_shrunk.png}} & \raisebox{-0.5\height}{\includegraphics[width=0.32\columnwidth]{blockgroup-0101-full_shrunk.png}} & Missing \\
\raisebox{-0.5\height}{\includegraphics[width=0.32\columnwidth]{tract-0101-net_shrunk.png}} & \raisebox{-0.5\height}{\includegraphics[width=0.32\columnwidth]{blockgroup-0101-net_shrunk.png}} & Missing \\
\end{tabular}
\imtitle{Alabama 1}
\end{minipage}

\begin{minipage}{\columnwidth}
\begin{tabular}{lccc}
\raisebox{-0.5\height}{\includegraphics[width=0.32\columnwidth]{tract-0102-full_shrunk.png}} & \raisebox{-0.5\height}{\includegraphics[width=0.32\columnwidth]{blockgroup-0102-full_shrunk.png}} & Missing \\
\raisebox{-0.5\height}{\includegraphics[width=0.32\columnwidth]{tract-0102-net_shrunk.png}} & \raisebox{-0.5\height}{\includegraphics[width=0.32\columnwidth]{blockgroup-0102-net_shrunk.png}} & \raisebox{-0.5\height}{\includegraphics[width=0.32\columnwidth]{block-0102-net_shrunk.png}} \\
\end{tabular}
\imtitle{Alabama 2}
\end{minipage}

\begin{minipage}{\columnwidth}
\begin{tabular}{lccc}
\raisebox{-0.5\height}{\includegraphics[width=0.32\columnwidth]{tract-0103-full_shrunk.png}} & \raisebox{-0.5\height}{\includegraphics[width=0.32\columnwidth]{blockgroup-0103-full_shrunk.png}} & Missing \\
\raisebox{-0.5\height}{\includegraphics[width=0.32\columnwidth]{tract-0103-net_shrunk.png}} & \raisebox{-0.5\height}{\includegraphics[width=0.32\columnwidth]{blockgroup-0103-net_shrunk.png}} & \raisebox{-0.5\height}{\includegraphics[width=0.32\columnwidth]{block-0103-net_shrunk.png}} \\
\end{tabular}
\imtitle{Alabama 3}
\end{minipage}

\begin{minipage}{\columnwidth}
\begin{tabular}{lccc}
\raisebox{-0.5\height}{\includegraphics[width=0.32\columnwidth]{tract-0104-full_shrunk.png}} & \raisebox{-0.5\height}{\includegraphics[width=0.32\columnwidth]{blockgroup-0104-full_shrunk.png}} & Missing \\
\raisebox{-0.5\height}{\includegraphics[width=0.32\columnwidth]{tract-0104-net_shrunk.png}} & \raisebox{-0.5\height}{\includegraphics[width=0.32\columnwidth]{blockgroup-0104-net_shrunk.png}} & Missing \\
\end{tabular}
\imtitle{Alabama 4}
\end{minipage}

\begin{minipage}{\columnwidth}
\begin{tabular}{lccc}
\raisebox{-0.5\height}{\includegraphics[width=0.32\columnwidth]{tract-0105-full_shrunk.png}} & \raisebox{-0.5\height}{\includegraphics[width=0.32\columnwidth]{blockgroup-0105-full_shrunk.png}} & Missing \\
\raisebox{-0.5\height}{\includegraphics[width=0.32\columnwidth]{tract-0105-net_shrunk.png}} & \raisebox{-0.5\height}{\includegraphics[width=0.32\columnwidth]{blockgroup-0105-net_shrunk.png}} & \raisebox{-0.5\height}{\includegraphics[width=0.32\columnwidth]{block-0105-net_shrunk.png}} \\
\end{tabular}
\imtitle{Alabama 5}
\end{minipage}

\begin{minipage}{\columnwidth}
\begin{tabular}{lccc}
\raisebox{-0.5\height}{\includegraphics[width=0.32\columnwidth]{tract-0106-full_shrunk.png}} & \raisebox{-0.5\height}{\includegraphics[width=0.32\columnwidth]{blockgroup-0106-full_shrunk.png}} & Missing \\
\raisebox{-0.5\height}{\includegraphics[width=0.32\columnwidth]{tract-0106-net_shrunk.png}} & \raisebox{-0.5\height}{\includegraphics[width=0.32\columnwidth]{blockgroup-0106-net_shrunk.png}} & \raisebox{-0.5\height}{\includegraphics[width=0.32\columnwidth]{block-0106-net_shrunk.png}} \\
\end{tabular}
\imtitle{Alabama 6}
\end{minipage}

\begin{minipage}{\columnwidth}
\begin{tabular}{lccc}
\raisebox{-0.5\height}{\includegraphics[width=0.32\columnwidth]{tract-0107-full_shrunk.png}} & \raisebox{-0.5\height}{\includegraphics[width=0.32\columnwidth]{blockgroup-0107-full_shrunk.png}} & Missing \\
\raisebox{-0.5\height}{\includegraphics[width=0.32\columnwidth]{tract-0107-net_shrunk.png}} & \raisebox{-0.5\height}{\includegraphics[width=0.32\columnwidth]{blockgroup-0107-net_shrunk.png}} & \raisebox{-0.5\height}{\includegraphics[width=0.32\columnwidth]{block-0107-net_shrunk.png}} \\
\end{tabular}
\imtitle{Alabama 7}
\end{minipage}
\bchap{Alaska}
\begin{minipage}{\columnwidth}
\begin{tabular}{lccc}
\raisebox{-0.5\height}{\includegraphics[width=0.32\columnwidth]{tract-0200-full_shrunk.png}} & \raisebox{-0.5\height}{\includegraphics[width=0.32\columnwidth]{blockgroup-0200-full_shrunk.png}} & \raisebox{-0.5\height}{\includegraphics[width=0.32\columnwidth]{block-0200-full_shrunk.png}} \\
\raisebox{-0.5\height}{\includegraphics[width=0.32\columnwidth]{tract-0200-net_shrunk.png}} & \raisebox{-0.5\height}{\includegraphics[width=0.32\columnwidth]{blockgroup-0200-net_shrunk.png}} & \raisebox{-0.5\height}{\includegraphics[width=0.32\columnwidth]{block-0200-net_shrunk.png}} \\
\end{tabular}
\imtitle{Alaska 0}
\end{minipage}
\bchap{Arizona}
\begin{minipage}{\columnwidth}
\begin{tabular}{lccc}
\raisebox{-0.5\height}{\includegraphics[width=0.32\columnwidth]{tract-0401-full_shrunk.png}} & \raisebox{-0.5\height}{\includegraphics[width=0.32\columnwidth]{blockgroup-0401-full_shrunk.png}} & Missing \\
\raisebox{-0.5\height}{\includegraphics[width=0.32\columnwidth]{tract-0401-net_shrunk.png}} & \raisebox{-0.5\height}{\includegraphics[width=0.32\columnwidth]{blockgroup-0401-net_shrunk.png}} & \raisebox{-0.5\height}{\includegraphics[width=0.32\columnwidth]{block-0401-net_shrunk.png}} \\
\end{tabular}
\imtitle{Arizona 1}
\end{minipage}

\begin{minipage}{\columnwidth}
\begin{tabular}{lccc}
\raisebox{-0.5\height}{\includegraphics[width=0.32\columnwidth]{tract-0402-full_shrunk.png}} & \raisebox{-0.5\height}{\includegraphics[width=0.32\columnwidth]{blockgroup-0402-full_shrunk.png}} & Missing \\
\raisebox{-0.5\height}{\includegraphics[width=0.32\columnwidth]{tract-0402-net_shrunk.png}} & \raisebox{-0.5\height}{\includegraphics[width=0.32\columnwidth]{blockgroup-0402-net_shrunk.png}} & \raisebox{-0.5\height}{\includegraphics[width=0.32\columnwidth]{block-0402-net_shrunk.png}} \\
\end{tabular}
\imtitle{Arizona 2}
\end{minipage}

\begin{minipage}{\columnwidth}
\begin{tabular}{lccc}
\raisebox{-0.5\height}{\includegraphics[width=0.32\columnwidth]{tract-0403-full_shrunk.png}} & \raisebox{-0.5\height}{\includegraphics[width=0.32\columnwidth]{blockgroup-0403-full_shrunk.png}} & \raisebox{-0.5\height}{\includegraphics[width=0.32\columnwidth]{block-0403-full_shrunk.png}} \\
\raisebox{-0.5\height}{\includegraphics[width=0.32\columnwidth]{tract-0403-net_shrunk.png}} & \raisebox{-0.5\height}{\includegraphics[width=0.32\columnwidth]{blockgroup-0403-net_shrunk.png}} & \raisebox{-0.5\height}{\includegraphics[width=0.32\columnwidth]{block-0403-net_shrunk.png}} \\
\end{tabular}
\imtitle{Arizona 3}
\end{minipage}

\begin{minipage}{\columnwidth}
\begin{tabular}{lccc}
\raisebox{-0.5\height}{\includegraphics[width=0.32\columnwidth]{tract-0404-full_shrunk.png}} & \raisebox{-0.5\height}{\includegraphics[width=0.32\columnwidth]{blockgroup-0404-full_shrunk.png}} & \raisebox{-0.5\height}{\includegraphics[width=0.32\columnwidth]{block-0404-full_shrunk.png}} \\
\raisebox{-0.5\height}{\includegraphics[width=0.32\columnwidth]{tract-0404-net_shrunk.png}} & \raisebox{-0.5\height}{\includegraphics[width=0.32\columnwidth]{blockgroup-0404-net_shrunk.png}} & \raisebox{-0.5\height}{\includegraphics[width=0.32\columnwidth]{block-0404-net_shrunk.png}} \\
\end{tabular}
\imtitle{Arizona 4}
\end{minipage}

\begin{minipage}{\columnwidth}
\begin{tabular}{lccc}
\raisebox{-0.5\height}{\includegraphics[width=0.32\columnwidth]{tract-0405-full_shrunk.png}} & \raisebox{-0.5\height}{\includegraphics[width=0.32\columnwidth]{blockgroup-0405-full_shrunk.png}} & \raisebox{-0.5\height}{\includegraphics[width=0.32\columnwidth]{block-0405-full_shrunk.png}} \\
\raisebox{-0.5\height}{\includegraphics[width=0.32\columnwidth]{tract-0405-net_shrunk.png}} & \raisebox{-0.5\height}{\includegraphics[width=0.32\columnwidth]{blockgroup-0405-net_shrunk.png}} & \raisebox{-0.5\height}{\includegraphics[width=0.32\columnwidth]{block-0405-net_shrunk.png}} \\
\end{tabular}
\imtitle{Arizona 5}
\end{minipage}

\begin{minipage}{\columnwidth}
\begin{tabular}{lccc}
\raisebox{-0.5\height}{\includegraphics[width=0.32\columnwidth]{tract-0406-full_shrunk.png}} & \raisebox{-0.5\height}{\includegraphics[width=0.32\columnwidth]{blockgroup-0406-full_shrunk.png}} & \raisebox{-0.5\height}{\includegraphics[width=0.32\columnwidth]{block-0406-full_shrunk.png}} \\
\raisebox{-0.5\height}{\includegraphics[width=0.32\columnwidth]{tract-0406-net_shrunk.png}} & \raisebox{-0.5\height}{\includegraphics[width=0.32\columnwidth]{blockgroup-0406-net_shrunk.png}} & \raisebox{-0.5\height}{\includegraphics[width=0.32\columnwidth]{block-0406-net_shrunk.png}} \\
\end{tabular}
\imtitle{Arizona 6}
\end{minipage}

\begin{minipage}{\columnwidth}
\begin{tabular}{lccc}
\raisebox{-0.5\height}{\includegraphics[width=0.32\columnwidth]{tract-0407-full_shrunk.png}} & \raisebox{-0.5\height}{\includegraphics[width=0.32\columnwidth]{blockgroup-0407-full_shrunk.png}} & Missing \\
\raisebox{-0.5\height}{\includegraphics[width=0.32\columnwidth]{tract-0407-net_shrunk.png}} & \raisebox{-0.5\height}{\includegraphics[width=0.32\columnwidth]{blockgroup-0407-net_shrunk.png}} & Missing \\
\end{tabular}
\imtitle{Arizona 7}
\end{minipage}

\begin{minipage}{\columnwidth}
\begin{tabular}{lccc}
\raisebox{-0.5\height}{\includegraphics[width=0.32\columnwidth]{tract-0408-full_shrunk.png}} & \raisebox{-0.5\height}{\includegraphics[width=0.32\columnwidth]{blockgroup-0408-full_shrunk.png}} & Missing \\
\raisebox{-0.5\height}{\includegraphics[width=0.32\columnwidth]{tract-0408-net_shrunk.png}} & \raisebox{-0.5\height}{\includegraphics[width=0.32\columnwidth]{blockgroup-0408-net_shrunk.png}} & Missing \\
\end{tabular}
\imtitle{Arizona 8}
\end{minipage}

\begin{minipage}{\columnwidth}
\begin{tabular}{lccc}
Missing & Missing & Missing \\
Missing & Missing & Missing \\
\end{tabular}
\imtitle{Arizona 9}
\end{minipage}
\bchap{Arkansas}
\begin{minipage}{\columnwidth}
\begin{tabular}{lccc}
\raisebox{-0.5\height}{\includegraphics[width=0.32\columnwidth]{tract-0501-full_shrunk.png}} & \raisebox{-0.5\height}{\includegraphics[width=0.32\columnwidth]{blockgroup-0501-full_shrunk.png}} & Missing \\
\raisebox{-0.5\height}{\includegraphics[width=0.32\columnwidth]{tract-0501-net_shrunk.png}} & \raisebox{-0.5\height}{\includegraphics[width=0.32\columnwidth]{blockgroup-0501-net_shrunk.png}} & \raisebox{-0.5\height}{\includegraphics[width=0.32\columnwidth]{block-0501-net_shrunk.png}} \\
\end{tabular}
\imtitle{Arkansas 1}
\end{minipage}

\begin{minipage}{\columnwidth}
\begin{tabular}{lccc}
\raisebox{-0.5\height}{\includegraphics[width=0.32\columnwidth]{tract-0502-full_shrunk.png}} & \raisebox{-0.5\height}{\includegraphics[width=0.32\columnwidth]{blockgroup-0502-full_shrunk.png}} & \raisebox{-0.5\height}{\includegraphics[width=0.32\columnwidth]{block-0502-full_shrunk.png}} \\
\raisebox{-0.5\height}{\includegraphics[width=0.32\columnwidth]{tract-0502-net_shrunk.png}} & \raisebox{-0.5\height}{\includegraphics[width=0.32\columnwidth]{blockgroup-0502-net_shrunk.png}} & Missing \\
\end{tabular}
\imtitle{Arkansas 2}
\end{minipage}

\begin{minipage}{\columnwidth}
\begin{tabular}{lccc}
\raisebox{-0.5\height}{\includegraphics[width=0.32\columnwidth]{tract-0503-full_shrunk.png}} & \raisebox{-0.5\height}{\includegraphics[width=0.32\columnwidth]{blockgroup-0503-full_shrunk.png}} & \raisebox{-0.5\height}{\includegraphics[width=0.32\columnwidth]{block-0503-full_shrunk.png}} \\
\raisebox{-0.5\height}{\includegraphics[width=0.32\columnwidth]{tract-0503-net_shrunk.png}} & \raisebox{-0.5\height}{\includegraphics[width=0.32\columnwidth]{blockgroup-0503-net_shrunk.png}} & \raisebox{-0.5\height}{\includegraphics[width=0.32\columnwidth]{block-0503-net_shrunk.png}} \\
\end{tabular}
\imtitle{Arkansas 3}
\end{minipage}

\begin{minipage}{\columnwidth}
\begin{tabular}{lccc}
\raisebox{-0.5\height}{\includegraphics[width=0.32\columnwidth]{tract-0504-full_shrunk.png}} & \raisebox{-0.5\height}{\includegraphics[width=0.32\columnwidth]{blockgroup-0504-full_shrunk.png}} & Missing \\
\raisebox{-0.5\height}{\includegraphics[width=0.32\columnwidth]{tract-0504-net_shrunk.png}} & \raisebox{-0.5\height}{\includegraphics[width=0.32\columnwidth]{blockgroup-0504-net_shrunk.png}} & \raisebox{-0.5\height}{\includegraphics[width=0.32\columnwidth]{block-0504-net_shrunk.png}} \\
\end{tabular}
\imtitle{Arkansas 4}
\end{minipage}
\bchap{California}
\begin{minipage}{\columnwidth}
\begin{tabular}{lccc}
\raisebox{-0.5\height}{\includegraphics[width=0.32\columnwidth]{tract-0601-full_shrunk.png}} & \raisebox{-0.5\height}{\includegraphics[width=0.32\columnwidth]{blockgroup-0601-full_shrunk.png}} & Missing \\
\raisebox{-0.5\height}{\includegraphics[width=0.32\columnwidth]{tract-0601-net_shrunk.png}} & \raisebox{-0.5\height}{\includegraphics[width=0.32\columnwidth]{blockgroup-0601-net_shrunk.png}} & Missing \\
\end{tabular}
\imtitle{California 1}
\end{minipage}

\begin{minipage}{\columnwidth}
\begin{tabular}{lccc}
\raisebox{-0.5\height}{\includegraphics[width=0.32\columnwidth]{tract-0602-full_shrunk.png}} & \raisebox{-0.5\height}{\includegraphics[width=0.32\columnwidth]{blockgroup-0602-full_shrunk.png}} & Missing \\
\raisebox{-0.5\height}{\includegraphics[width=0.32\columnwidth]{tract-0602-net_shrunk.png}} & \raisebox{-0.5\height}{\includegraphics[width=0.32\columnwidth]{blockgroup-0602-net_shrunk.png}} & Missing \\
\end{tabular}
\imtitle{California 2}
\end{minipage}

\begin{minipage}{\columnwidth}
\begin{tabular}{lccc}
\raisebox{-0.5\height}{\includegraphics[width=0.32\columnwidth]{tract-0603-full_shrunk.png}} & \raisebox{-0.5\height}{\includegraphics[width=0.32\columnwidth]{blockgroup-0603-full_shrunk.png}} & Missing \\
\raisebox{-0.5\height}{\includegraphics[width=0.32\columnwidth]{tract-0603-net_shrunk.png}} & \raisebox{-0.5\height}{\includegraphics[width=0.32\columnwidth]{blockgroup-0603-net_shrunk.png}} & \raisebox{-0.5\height}{\includegraphics[width=0.32\columnwidth]{block-0603-net_shrunk.png}} \\
\end{tabular}
\imtitle{California 3}
\end{minipage}

\begin{minipage}{\columnwidth}
\begin{tabular}{lccc}
\raisebox{-0.5\height}{\includegraphics[width=0.32\columnwidth]{tract-0604-full_shrunk.png}} & \raisebox{-0.5\height}{\includegraphics[width=0.32\columnwidth]{blockgroup-0604-full_shrunk.png}} & Missing \\
\raisebox{-0.5\height}{\includegraphics[width=0.32\columnwidth]{tract-0604-net_shrunk.png}} & \raisebox{-0.5\height}{\includegraphics[width=0.32\columnwidth]{blockgroup-0604-net_shrunk.png}} & Missing \\
\end{tabular}
\imtitle{California 4}
\end{minipage}

\begin{minipage}{\columnwidth}
\begin{tabular}{lccc}
\raisebox{-0.5\height}{\includegraphics[width=0.32\columnwidth]{tract-0605-full_shrunk.png}} & \raisebox{-0.5\height}{\includegraphics[width=0.32\columnwidth]{blockgroup-0605-full_shrunk.png}} & Missing \\
\raisebox{-0.5\height}{\includegraphics[width=0.32\columnwidth]{tract-0605-net_shrunk.png}} & \raisebox{-0.5\height}{\includegraphics[width=0.32\columnwidth]{blockgroup-0605-net_shrunk.png}} & \raisebox{-0.5\height}{\includegraphics[width=0.32\columnwidth]{block-0605-net_shrunk.png}} \\
\end{tabular}
\imtitle{California 5}
\end{minipage}

\begin{minipage}{\columnwidth}
\begin{tabular}{lccc}
\raisebox{-0.5\height}{\includegraphics[width=0.32\columnwidth]{tract-0606-full_shrunk.png}} & \raisebox{-0.5\height}{\includegraphics[width=0.32\columnwidth]{blockgroup-0606-full_shrunk.png}} & Missing \\
\raisebox{-0.5\height}{\includegraphics[width=0.32\columnwidth]{tract-0606-net_shrunk.png}} & \raisebox{-0.5\height}{\includegraphics[width=0.32\columnwidth]{blockgroup-0606-net_shrunk.png}} & Missing \\
\end{tabular}
\imtitle{California 6}
\end{minipage}

\begin{minipage}{\columnwidth}
\begin{tabular}{lccc}
\raisebox{-0.5\height}{\includegraphics[width=0.32\columnwidth]{tract-0607-full_shrunk.png}} & \raisebox{-0.5\height}{\includegraphics[width=0.32\columnwidth]{blockgroup-0607-full_shrunk.png}} & Missing \\
\raisebox{-0.5\height}{\includegraphics[width=0.32\columnwidth]{tract-0607-net_shrunk.png}} & \raisebox{-0.5\height}{\includegraphics[width=0.32\columnwidth]{blockgroup-0607-net_shrunk.png}} & Missing \\
\end{tabular}
\imtitle{California 7}
\end{minipage}

\begin{minipage}{\columnwidth}
\begin{tabular}{lccc}
\raisebox{-0.5\height}{\includegraphics[width=0.32\columnwidth]{tract-0608-full_shrunk.png}} & \raisebox{-0.5\height}{\includegraphics[width=0.32\columnwidth]{blockgroup-0608-full_shrunk.png}} & Missing \\
\raisebox{-0.5\height}{\includegraphics[width=0.32\columnwidth]{tract-0608-net_shrunk.png}} & \raisebox{-0.5\height}{\includegraphics[width=0.32\columnwidth]{blockgroup-0608-net_shrunk.png}} & Missing \\
\end{tabular}
\imtitle{California 8}
\end{minipage}

\begin{minipage}{\columnwidth}
\begin{tabular}{lccc}
\raisebox{-0.5\height}{\includegraphics[width=0.32\columnwidth]{tract-0609-full_shrunk.png}} & \raisebox{-0.5\height}{\includegraphics[width=0.32\columnwidth]{blockgroup-0609-full_shrunk.png}} & Missing \\
\raisebox{-0.5\height}{\includegraphics[width=0.32\columnwidth]{tract-0609-net_shrunk.png}} & \raisebox{-0.5\height}{\includegraphics[width=0.32\columnwidth]{blockgroup-0609-net_shrunk.png}} & Missing \\
\end{tabular}
\imtitle{California 9}
\end{minipage}

\begin{minipage}{\columnwidth}
\begin{tabular}{lccc}
\raisebox{-0.5\height}{\includegraphics[width=0.32\columnwidth]{tract-0610-full_shrunk.png}} & \raisebox{-0.5\height}{\includegraphics[width=0.32\columnwidth]{blockgroup-0610-full_shrunk.png}} & Missing \\
\raisebox{-0.5\height}{\includegraphics[width=0.32\columnwidth]{tract-0610-net_shrunk.png}} & \raisebox{-0.5\height}{\includegraphics[width=0.32\columnwidth]{blockgroup-0610-net_shrunk.png}} & Missing \\
\end{tabular}
\imtitle{California 10}
\end{minipage}

\begin{minipage}{\columnwidth}
\begin{tabular}{lccc}
\raisebox{-0.5\height}{\includegraphics[width=0.32\columnwidth]{tract-0611-full_shrunk.png}} & \raisebox{-0.5\height}{\includegraphics[width=0.32\columnwidth]{blockgroup-0611-full_shrunk.png}} & Missing \\
\raisebox{-0.5\height}{\includegraphics[width=0.32\columnwidth]{tract-0611-net_shrunk.png}} & \raisebox{-0.5\height}{\includegraphics[width=0.32\columnwidth]{blockgroup-0611-net_shrunk.png}} & Missing \\
\end{tabular}
\imtitle{California 11}
\end{minipage}

\begin{minipage}{\columnwidth}
\begin{tabular}{lccc}
\raisebox{-0.5\height}{\includegraphics[width=0.32\columnwidth]{tract-0612-full_shrunk.png}} & \raisebox{-0.5\height}{\includegraphics[width=0.32\columnwidth]{blockgroup-0612-full_shrunk.png}} & Missing \\
\raisebox{-0.5\height}{\includegraphics[width=0.32\columnwidth]{tract-0612-net_shrunk.png}} & \raisebox{-0.5\height}{\includegraphics[width=0.32\columnwidth]{blockgroup-0612-net_shrunk.png}} & Missing \\
\end{tabular}
\imtitle{California 12}
\end{minipage}

\begin{minipage}{\columnwidth}
\begin{tabular}{lccc}
\raisebox{-0.5\height}{\includegraphics[width=0.32\columnwidth]{tract-0613-full_shrunk.png}} & \raisebox{-0.5\height}{\includegraphics[width=0.32\columnwidth]{blockgroup-0613-full_shrunk.png}} & Missing \\
\raisebox{-0.5\height}{\includegraphics[width=0.32\columnwidth]{tract-0613-net_shrunk.png}} & \raisebox{-0.5\height}{\includegraphics[width=0.32\columnwidth]{blockgroup-0613-net_shrunk.png}} & Missing \\
\end{tabular}
\imtitle{California 13}
\end{minipage}

\begin{minipage}{\columnwidth}
\begin{tabular}{lccc}
\raisebox{-0.5\height}{\includegraphics[width=0.32\columnwidth]{tract-0614-full_shrunk.png}} & \raisebox{-0.5\height}{\includegraphics[width=0.32\columnwidth]{blockgroup-0614-full_shrunk.png}} & Missing \\
\raisebox{-0.5\height}{\includegraphics[width=0.32\columnwidth]{tract-0614-net_shrunk.png}} & \raisebox{-0.5\height}{\includegraphics[width=0.32\columnwidth]{blockgroup-0614-net_shrunk.png}} & Missing \\
\end{tabular}
\imtitle{California 14}
\end{minipage}

\begin{minipage}{\columnwidth}
\begin{tabular}{lccc}
\raisebox{-0.5\height}{\includegraphics[width=0.32\columnwidth]{tract-0615-full_shrunk.png}} & \raisebox{-0.5\height}{\includegraphics[width=0.32\columnwidth]{blockgroup-0615-full_shrunk.png}} & Missing \\
\raisebox{-0.5\height}{\includegraphics[width=0.32\columnwidth]{tract-0615-net_shrunk.png}} & \raisebox{-0.5\height}{\includegraphics[width=0.32\columnwidth]{blockgroup-0615-net_shrunk.png}} & \raisebox{-0.5\height}{\includegraphics[width=0.32\columnwidth]{block-0615-net_shrunk.png}} \\
\end{tabular}
\imtitle{California 15}
\end{minipage}

\begin{minipage}{\columnwidth}
\begin{tabular}{lccc}
\raisebox{-0.5\height}{\includegraphics[width=0.32\columnwidth]{tract-0616-full_shrunk.png}} & \raisebox{-0.5\height}{\includegraphics[width=0.32\columnwidth]{blockgroup-0616-full_shrunk.png}} & Missing \\
\raisebox{-0.5\height}{\includegraphics[width=0.32\columnwidth]{tract-0616-net_shrunk.png}} & \raisebox{-0.5\height}{\includegraphics[width=0.32\columnwidth]{blockgroup-0616-net_shrunk.png}} & \raisebox{-0.5\height}{\includegraphics[width=0.32\columnwidth]{block-0616-net_shrunk.png}} \\
\end{tabular}
\imtitle{California 16}
\end{minipage}

\begin{minipage}{\columnwidth}
\begin{tabular}{lccc}
\raisebox{-0.5\height}{\includegraphics[width=0.32\columnwidth]{tract-0617-full_shrunk.png}} & \raisebox{-0.5\height}{\includegraphics[width=0.32\columnwidth]{blockgroup-0617-full_shrunk.png}} & Missing \\
\raisebox{-0.5\height}{\includegraphics[width=0.32\columnwidth]{tract-0617-net_shrunk.png}} & \raisebox{-0.5\height}{\includegraphics[width=0.32\columnwidth]{blockgroup-0617-net_shrunk.png}} & \raisebox{-0.5\height}{\includegraphics[width=0.32\columnwidth]{block-0617-net_shrunk.png}} \\
\end{tabular}
\imtitle{California 17}
\end{minipage}

\begin{minipage}{\columnwidth}
\begin{tabular}{lccc}
\raisebox{-0.5\height}{\includegraphics[width=0.32\columnwidth]{tract-0618-full_shrunk.png}} & \raisebox{-0.5\height}{\includegraphics[width=0.32\columnwidth]{blockgroup-0618-full_shrunk.png}} & Missing \\
\raisebox{-0.5\height}{\includegraphics[width=0.32\columnwidth]{tract-0618-net_shrunk.png}} & \raisebox{-0.5\height}{\includegraphics[width=0.32\columnwidth]{blockgroup-0618-net_shrunk.png}} & Missing \\
\end{tabular}
\imtitle{California 18}
\end{minipage}

\begin{minipage}{\columnwidth}
\begin{tabular}{lccc}
\raisebox{-0.5\height}{\includegraphics[width=0.32\columnwidth]{tract-0619-full_shrunk.png}} & \raisebox{-0.5\height}{\includegraphics[width=0.32\columnwidth]{blockgroup-0619-full_shrunk.png}} & Missing \\
\raisebox{-0.5\height}{\includegraphics[width=0.32\columnwidth]{tract-0619-net_shrunk.png}} & \raisebox{-0.5\height}{\includegraphics[width=0.32\columnwidth]{blockgroup-0619-net_shrunk.png}} & \raisebox{-0.5\height}{\includegraphics[width=0.32\columnwidth]{block-0619-net_shrunk.png}} \\
\end{tabular}
\imtitle{California 19}
\end{minipage}

\begin{minipage}{\columnwidth}
\begin{tabular}{lccc}
\raisebox{-0.5\height}{\includegraphics[width=0.32\columnwidth]{tract-0620-full_shrunk.png}} & \raisebox{-0.5\height}{\includegraphics[width=0.32\columnwidth]{blockgroup-0620-full_shrunk.png}} & Missing \\
\raisebox{-0.5\height}{\includegraphics[width=0.32\columnwidth]{tract-0620-net_shrunk.png}} & \raisebox{-0.5\height}{\includegraphics[width=0.32\columnwidth]{blockgroup-0620-net_shrunk.png}} & Missing \\
\end{tabular}
\imtitle{California 20}
\end{minipage}

\begin{minipage}{\columnwidth}
\begin{tabular}{lccc}
\raisebox{-0.5\height}{\includegraphics[width=0.32\columnwidth]{tract-0621-full_shrunk.png}} & \raisebox{-0.5\height}{\includegraphics[width=0.32\columnwidth]{blockgroup-0621-full_shrunk.png}} & Missing \\
\raisebox{-0.5\height}{\includegraphics[width=0.32\columnwidth]{tract-0621-net_shrunk.png}} & \raisebox{-0.5\height}{\includegraphics[width=0.32\columnwidth]{blockgroup-0621-net_shrunk.png}} & Missing \\
\end{tabular}
\imtitle{California 21}
\end{minipage}

\begin{minipage}{\columnwidth}
\begin{tabular}{lccc}
\raisebox{-0.5\height}{\includegraphics[width=0.32\columnwidth]{tract-0622-full_shrunk.png}} & \raisebox{-0.5\height}{\includegraphics[width=0.32\columnwidth]{blockgroup-0622-full_shrunk.png}} & Missing \\
\raisebox{-0.5\height}{\includegraphics[width=0.32\columnwidth]{tract-0622-net_shrunk.png}} & \raisebox{-0.5\height}{\includegraphics[width=0.32\columnwidth]{blockgroup-0622-net_shrunk.png}} & Missing \\
\end{tabular}
\imtitle{California 22}
\end{minipage}

\begin{minipage}{\columnwidth}
\begin{tabular}{lccc}
\raisebox{-0.5\height}{\includegraphics[width=0.32\columnwidth]{tract-0623-full_shrunk.png}} & \raisebox{-0.5\height}{\includegraphics[width=0.32\columnwidth]{blockgroup-0623-full_shrunk.png}} & Missing \\
\raisebox{-0.5\height}{\includegraphics[width=0.32\columnwidth]{tract-0623-net_shrunk.png}} & \raisebox{-0.5\height}{\includegraphics[width=0.32\columnwidth]{blockgroup-0623-net_shrunk.png}} & Missing \\
\end{tabular}
\imtitle{California 23}
\end{minipage}

\begin{minipage}{\columnwidth}
\begin{tabular}{lccc}
\raisebox{-0.5\height}{\includegraphics[width=0.32\columnwidth]{tract-0624-full_shrunk.png}} & \raisebox{-0.5\height}{\includegraphics[width=0.32\columnwidth]{blockgroup-0624-full_shrunk.png}} & Missing \\
\raisebox{-0.5\height}{\includegraphics[width=0.32\columnwidth]{tract-0624-net_shrunk.png}} & \raisebox{-0.5\height}{\includegraphics[width=0.32\columnwidth]{blockgroup-0624-net_shrunk.png}} & Missing \\
\end{tabular}
\imtitle{California 24}
\end{minipage}

\begin{minipage}{\columnwidth}
\begin{tabular}{lccc}
\raisebox{-0.5\height}{\includegraphics[width=0.32\columnwidth]{tract-0625-full_shrunk.png}} & \raisebox{-0.5\height}{\includegraphics[width=0.32\columnwidth]{blockgroup-0625-full_shrunk.png}} & Missing \\
\raisebox{-0.5\height}{\includegraphics[width=0.32\columnwidth]{tract-0625-net_shrunk.png}} & \raisebox{-0.5\height}{\includegraphics[width=0.32\columnwidth]{blockgroup-0625-net_shrunk.png}} & Missing \\
\end{tabular}
\imtitle{California 25}
\end{minipage}

\begin{minipage}{\columnwidth}
\begin{tabular}{lccc}
\raisebox{-0.5\height}{\includegraphics[width=0.32\columnwidth]{tract-0626-full_shrunk.png}} & \raisebox{-0.5\height}{\includegraphics[width=0.32\columnwidth]{blockgroup-0626-full_shrunk.png}} & Missing \\
\raisebox{-0.5\height}{\includegraphics[width=0.32\columnwidth]{tract-0626-net_shrunk.png}} & \raisebox{-0.5\height}{\includegraphics[width=0.32\columnwidth]{blockgroup-0626-net_shrunk.png}} & Missing \\
\end{tabular}
\imtitle{California 26}
\end{minipage}

\begin{minipage}{\columnwidth}
\begin{tabular}{lccc}
\raisebox{-0.5\height}{\includegraphics[width=0.32\columnwidth]{tract-0627-full_shrunk.png}} & \raisebox{-0.5\height}{\includegraphics[width=0.32\columnwidth]{blockgroup-0627-full_shrunk.png}} & Missing \\
\raisebox{-0.5\height}{\includegraphics[width=0.32\columnwidth]{tract-0627-net_shrunk.png}} & \raisebox{-0.5\height}{\includegraphics[width=0.32\columnwidth]{blockgroup-0627-net_shrunk.png}} & \raisebox{-0.5\height}{\includegraphics[width=0.32\columnwidth]{block-0627-net_shrunk.png}} \\
\end{tabular}
\imtitle{California 27}
\end{minipage}

\begin{minipage}{\columnwidth}
\begin{tabular}{lccc}
\raisebox{-0.5\height}{\includegraphics[width=0.32\columnwidth]{tract-0628-full_shrunk.png}} & \raisebox{-0.5\height}{\includegraphics[width=0.32\columnwidth]{blockgroup-0628-full_shrunk.png}} & Missing \\
\raisebox{-0.5\height}{\includegraphics[width=0.32\columnwidth]{tract-0628-net_shrunk.png}} & \raisebox{-0.5\height}{\includegraphics[width=0.32\columnwidth]{blockgroup-0628-net_shrunk.png}} & \raisebox{-0.5\height}{\includegraphics[width=0.32\columnwidth]{block-0628-net_shrunk.png}} \\
\end{tabular}
\imtitle{California 28}
\end{minipage}

\begin{minipage}{\columnwidth}
\begin{tabular}{lccc}
\raisebox{-0.5\height}{\includegraphics[width=0.32\columnwidth]{tract-0629-full_shrunk.png}} & \raisebox{-0.5\height}{\includegraphics[width=0.32\columnwidth]{blockgroup-0629-full_shrunk.png}} & Missing \\
\raisebox{-0.5\height}{\includegraphics[width=0.32\columnwidth]{tract-0629-net_shrunk.png}} & \raisebox{-0.5\height}{\includegraphics[width=0.32\columnwidth]{blockgroup-0629-net_shrunk.png}} & \raisebox{-0.5\height}{\includegraphics[width=0.32\columnwidth]{block-0629-net_shrunk.png}} \\
\end{tabular}
\imtitle{California 29}
\end{minipage}

\begin{minipage}{\columnwidth}
\begin{tabular}{lccc}
\raisebox{-0.5\height}{\includegraphics[width=0.32\columnwidth]{tract-0630-full_shrunk.png}} & \raisebox{-0.5\height}{\includegraphics[width=0.32\columnwidth]{blockgroup-0630-full_shrunk.png}} & Missing \\
\raisebox{-0.5\height}{\includegraphics[width=0.32\columnwidth]{tract-0630-net_shrunk.png}} & \raisebox{-0.5\height}{\includegraphics[width=0.32\columnwidth]{blockgroup-0630-net_shrunk.png}} & \raisebox{-0.5\height}{\includegraphics[width=0.32\columnwidth]{block-0630-net_shrunk.png}} \\
\end{tabular}
\imtitle{California 30}
\end{minipage}

\begin{minipage}{\columnwidth}
\begin{tabular}{lccc}
\raisebox{-0.5\height}{\includegraphics[width=0.32\columnwidth]{tract-0631-full_shrunk.png}} & \raisebox{-0.5\height}{\includegraphics[width=0.32\columnwidth]{blockgroup-0631-full_shrunk.png}} & Missing \\
\raisebox{-0.5\height}{\includegraphics[width=0.32\columnwidth]{tract-0631-net_shrunk.png}} & \raisebox{-0.5\height}{\includegraphics[width=0.32\columnwidth]{blockgroup-0631-net_shrunk.png}} & \raisebox{-0.5\height}{\includegraphics[width=0.32\columnwidth]{block-0631-net_shrunk.png}} \\
\end{tabular}
\imtitle{California 31}
\end{minipage}

\begin{minipage}{\columnwidth}
\begin{tabular}{lccc}
\raisebox{-0.5\height}{\includegraphics[width=0.32\columnwidth]{tract-0632-full_shrunk.png}} & \raisebox{-0.5\height}{\includegraphics[width=0.32\columnwidth]{blockgroup-0632-full_shrunk.png}} & Missing \\
\raisebox{-0.5\height}{\includegraphics[width=0.32\columnwidth]{tract-0632-net_shrunk.png}} & \raisebox{-0.5\height}{\includegraphics[width=0.32\columnwidth]{blockgroup-0632-net_shrunk.png}} & \raisebox{-0.5\height}{\includegraphics[width=0.32\columnwidth]{block-0632-net_shrunk.png}} \\
\end{tabular}
\imtitle{California 32}
\end{minipage}

\begin{minipage}{\columnwidth}
\begin{tabular}{lccc}
\raisebox{-0.5\height}{\includegraphics[width=0.32\columnwidth]{tract-0633-full_shrunk.png}} & \raisebox{-0.5\height}{\includegraphics[width=0.32\columnwidth]{blockgroup-0633-full_shrunk.png}} & Missing \\
\raisebox{-0.5\height}{\includegraphics[width=0.32\columnwidth]{tract-0633-net_shrunk.png}} & \raisebox{-0.5\height}{\includegraphics[width=0.32\columnwidth]{blockgroup-0633-net_shrunk.png}} & \raisebox{-0.5\height}{\includegraphics[width=0.32\columnwidth]{block-0633-net_shrunk.png}} \\
\end{tabular}
\imtitle{California 33}
\end{minipage}

\begin{minipage}{\columnwidth}
\begin{tabular}{lccc}
\raisebox{-0.5\height}{\includegraphics[width=0.32\columnwidth]{tract-0634-full_shrunk.png}} & \raisebox{-0.5\height}{\includegraphics[width=0.32\columnwidth]{blockgroup-0634-full_shrunk.png}} & Missing \\
\raisebox{-0.5\height}{\includegraphics[width=0.32\columnwidth]{tract-0634-net_shrunk.png}} & \raisebox{-0.5\height}{\includegraphics[width=0.32\columnwidth]{blockgroup-0634-net_shrunk.png}} & \raisebox{-0.5\height}{\includegraphics[width=0.32\columnwidth]{block-0634-net_shrunk.png}} \\
\end{tabular}
\imtitle{California 34}
\end{minipage}

\begin{minipage}{\columnwidth}
\begin{tabular}{lccc}
\raisebox{-0.5\height}{\includegraphics[width=0.32\columnwidth]{tract-0635-full_shrunk.png}} & \raisebox{-0.5\height}{\includegraphics[width=0.32\columnwidth]{blockgroup-0635-full_shrunk.png}} & Missing \\
\raisebox{-0.5\height}{\includegraphics[width=0.32\columnwidth]{tract-0635-net_shrunk.png}} & \raisebox{-0.5\height}{\includegraphics[width=0.32\columnwidth]{blockgroup-0635-net_shrunk.png}} & \raisebox{-0.5\height}{\includegraphics[width=0.32\columnwidth]{block-0635-net_shrunk.png}} \\
\end{tabular}
\imtitle{California 35}
\end{minipage}

\begin{minipage}{\columnwidth}
\begin{tabular}{lccc}
\raisebox{-0.5\height}{\includegraphics[width=0.32\columnwidth]{tract-0636-full_shrunk.png}} & \raisebox{-0.5\height}{\includegraphics[width=0.32\columnwidth]{blockgroup-0636-full_shrunk.png}} & Missing \\
\raisebox{-0.5\height}{\includegraphics[width=0.32\columnwidth]{tract-0636-net_shrunk.png}} & \raisebox{-0.5\height}{\includegraphics[width=0.32\columnwidth]{blockgroup-0636-net_shrunk.png}} & \raisebox{-0.5\height}{\includegraphics[width=0.32\columnwidth]{block-0636-net_shrunk.png}} \\
\end{tabular}
\imtitle{California 36}
\end{minipage}

\begin{minipage}{\columnwidth}
\begin{tabular}{lccc}
\raisebox{-0.5\height}{\includegraphics[width=0.32\columnwidth]{tract-0637-full_shrunk.png}} & \raisebox{-0.5\height}{\includegraphics[width=0.32\columnwidth]{blockgroup-0637-full_shrunk.png}} & Missing \\
\raisebox{-0.5\height}{\includegraphics[width=0.32\columnwidth]{tract-0637-net_shrunk.png}} & \raisebox{-0.5\height}{\includegraphics[width=0.32\columnwidth]{blockgroup-0637-net_shrunk.png}} & \raisebox{-0.5\height}{\includegraphics[width=0.32\columnwidth]{block-0637-net_shrunk.png}} \\
\end{tabular}
\imtitle{California 37}
\end{minipage}

\begin{minipage}{\columnwidth}
\begin{tabular}{lccc}
\raisebox{-0.5\height}{\includegraphics[width=0.32\columnwidth]{tract-0638-full_shrunk.png}} & \raisebox{-0.5\height}{\includegraphics[width=0.32\columnwidth]{blockgroup-0638-full_shrunk.png}} & Missing \\
\raisebox{-0.5\height}{\includegraphics[width=0.32\columnwidth]{tract-0638-net_shrunk.png}} & \raisebox{-0.5\height}{\includegraphics[width=0.32\columnwidth]{blockgroup-0638-net_shrunk.png}} & \raisebox{-0.5\height}{\includegraphics[width=0.32\columnwidth]{block-0638-net_shrunk.png}} \\
\end{tabular}
\imtitle{California 38}
\end{minipage}

\begin{minipage}{\columnwidth}
\begin{tabular}{lccc}
\raisebox{-0.5\height}{\includegraphics[width=0.32\columnwidth]{tract-0639-full_shrunk.png}} & \raisebox{-0.5\height}{\includegraphics[width=0.32\columnwidth]{blockgroup-0639-full_shrunk.png}} & Missing \\
\raisebox{-0.5\height}{\includegraphics[width=0.32\columnwidth]{tract-0639-net_shrunk.png}} & \raisebox{-0.5\height}{\includegraphics[width=0.32\columnwidth]{blockgroup-0639-net_shrunk.png}} & \raisebox{-0.5\height}{\includegraphics[width=0.32\columnwidth]{block-0639-net_shrunk.png}} \\
\end{tabular}
\imtitle{California 39}
\end{minipage}

\begin{minipage}{\columnwidth}
\begin{tabular}{lccc}
\raisebox{-0.5\height}{\includegraphics[width=0.32\columnwidth]{tract-0640-full_shrunk.png}} & \raisebox{-0.5\height}{\includegraphics[width=0.32\columnwidth]{blockgroup-0640-full_shrunk.png}} & Missing \\
\raisebox{-0.5\height}{\includegraphics[width=0.32\columnwidth]{tract-0640-net_shrunk.png}} & \raisebox{-0.5\height}{\includegraphics[width=0.32\columnwidth]{blockgroup-0640-net_shrunk.png}} & \raisebox{-0.5\height}{\includegraphics[width=0.32\columnwidth]{block-0640-net_shrunk.png}} \\
\end{tabular}
\imtitle{California 40}
\end{minipage}

\begin{minipage}{\columnwidth}
\begin{tabular}{lccc}
\raisebox{-0.5\height}{\includegraphics[width=0.32\columnwidth]{tract-0641-full_shrunk.png}} & \raisebox{-0.5\height}{\includegraphics[width=0.32\columnwidth]{blockgroup-0641-full_shrunk.png}} & Missing \\
\raisebox{-0.5\height}{\includegraphics[width=0.32\columnwidth]{tract-0641-net_shrunk.png}} & \raisebox{-0.5\height}{\includegraphics[width=0.32\columnwidth]{blockgroup-0641-net_shrunk.png}} & Missing \\
\end{tabular}
\imtitle{California 41}
\end{minipage}

\begin{minipage}{\columnwidth}
\begin{tabular}{lccc}
\raisebox{-0.5\height}{\includegraphics[width=0.32\columnwidth]{tract-0642-full_shrunk.png}} & \raisebox{-0.5\height}{\includegraphics[width=0.32\columnwidth]{blockgroup-0642-full_shrunk.png}} & Missing \\
\raisebox{-0.5\height}{\includegraphics[width=0.32\columnwidth]{tract-0642-net_shrunk.png}} & \raisebox{-0.5\height}{\includegraphics[width=0.32\columnwidth]{blockgroup-0642-net_shrunk.png}} & \raisebox{-0.5\height}{\includegraphics[width=0.32\columnwidth]{block-0642-net_shrunk.png}} \\
\end{tabular}
\imtitle{California 42}
\end{minipage}

\begin{minipage}{\columnwidth}
\begin{tabular}{lccc}
\raisebox{-0.5\height}{\includegraphics[width=0.32\columnwidth]{tract-0643-full_shrunk.png}} & \raisebox{-0.5\height}{\includegraphics[width=0.32\columnwidth]{blockgroup-0643-full_shrunk.png}} & Missing \\
\raisebox{-0.5\height}{\includegraphics[width=0.32\columnwidth]{tract-0643-net_shrunk.png}} & \raisebox{-0.5\height}{\includegraphics[width=0.32\columnwidth]{blockgroup-0643-net_shrunk.png}} & Missing \\
\end{tabular}
\imtitle{California 43}
\end{minipage}

\begin{minipage}{\columnwidth}
\begin{tabular}{lccc}
\raisebox{-0.5\height}{\includegraphics[width=0.32\columnwidth]{tract-0644-full_shrunk.png}} & \raisebox{-0.5\height}{\includegraphics[width=0.32\columnwidth]{blockgroup-0644-full_shrunk.png}} & Missing \\
\raisebox{-0.5\height}{\includegraphics[width=0.32\columnwidth]{tract-0644-net_shrunk.png}} & \raisebox{-0.5\height}{\includegraphics[width=0.32\columnwidth]{blockgroup-0644-net_shrunk.png}} & Missing \\
\end{tabular}
\imtitle{California 44}
\end{minipage}

\begin{minipage}{\columnwidth}
\begin{tabular}{lccc}
\raisebox{-0.5\height}{\includegraphics[width=0.32\columnwidth]{tract-0645-full_shrunk.png}} & \raisebox{-0.5\height}{\includegraphics[width=0.32\columnwidth]{blockgroup-0645-full_shrunk.png}} & Missing \\
\raisebox{-0.5\height}{\includegraphics[width=0.32\columnwidth]{tract-0645-net_shrunk.png}} & \raisebox{-0.5\height}{\includegraphics[width=0.32\columnwidth]{blockgroup-0645-net_shrunk.png}} & Missing \\
\end{tabular}
\imtitle{California 45}
\end{minipage}

\begin{minipage}{\columnwidth}
\begin{tabular}{lccc}
\raisebox{-0.5\height}{\includegraphics[width=0.32\columnwidth]{tract-0646-full_shrunk.png}} & \raisebox{-0.5\height}{\includegraphics[width=0.32\columnwidth]{blockgroup-0646-full_shrunk.png}} & Missing \\
\raisebox{-0.5\height}{\includegraphics[width=0.32\columnwidth]{tract-0646-net_shrunk.png}} & \raisebox{-0.5\height}{\includegraphics[width=0.32\columnwidth]{blockgroup-0646-net_shrunk.png}} & \raisebox{-0.5\height}{\includegraphics[width=0.32\columnwidth]{block-0646-net_shrunk.png}} \\
\end{tabular}
\imtitle{California 46}
\end{minipage}

\begin{minipage}{\columnwidth}
\begin{tabular}{lccc}
\raisebox{-0.5\height}{\includegraphics[width=0.32\columnwidth]{tract-0647-full_shrunk.png}} & \raisebox{-0.5\height}{\includegraphics[width=0.32\columnwidth]{blockgroup-0647-full_shrunk.png}} & Missing \\
\raisebox{-0.5\height}{\includegraphics[width=0.32\columnwidth]{tract-0647-net_shrunk.png}} & \raisebox{-0.5\height}{\includegraphics[width=0.32\columnwidth]{blockgroup-0647-net_shrunk.png}} & \raisebox{-0.5\height}{\includegraphics[width=0.32\columnwidth]{block-0647-net_shrunk.png}} \\
\end{tabular}
\imtitle{California 47}
\end{minipage}

\begin{minipage}{\columnwidth}
\begin{tabular}{lccc}
\raisebox{-0.5\height}{\includegraphics[width=0.32\columnwidth]{tract-0648-full_shrunk.png}} & \raisebox{-0.5\height}{\includegraphics[width=0.32\columnwidth]{blockgroup-0648-full_shrunk.png}} & Missing \\
\raisebox{-0.5\height}{\includegraphics[width=0.32\columnwidth]{tract-0648-net_shrunk.png}} & \raisebox{-0.5\height}{\includegraphics[width=0.32\columnwidth]{blockgroup-0648-net_shrunk.png}} & \raisebox{-0.5\height}{\includegraphics[width=0.32\columnwidth]{block-0648-net_shrunk.png}} \\
\end{tabular}
\imtitle{California 48}
\end{minipage}

\begin{minipage}{\columnwidth}
\begin{tabular}{lccc}
\raisebox{-0.5\height}{\includegraphics[width=0.32\columnwidth]{tract-0649-full_shrunk.png}} & \raisebox{-0.5\height}{\includegraphics[width=0.32\columnwidth]{blockgroup-0649-full_shrunk.png}} & Missing \\
\raisebox{-0.5\height}{\includegraphics[width=0.32\columnwidth]{tract-0649-net_shrunk.png}} & \raisebox{-0.5\height}{\includegraphics[width=0.32\columnwidth]{blockgroup-0649-net_shrunk.png}} & Missing \\
\end{tabular}
\imtitle{California 49}
\end{minipage}

\begin{minipage}{\columnwidth}
\begin{tabular}{lccc}
\raisebox{-0.5\height}{\includegraphics[width=0.32\columnwidth]{tract-0650-full_shrunk.png}} & \raisebox{-0.5\height}{\includegraphics[width=0.32\columnwidth]{blockgroup-0650-full_shrunk.png}} & Missing \\
\raisebox{-0.5\height}{\includegraphics[width=0.32\columnwidth]{tract-0650-net_shrunk.png}} & \raisebox{-0.5\height}{\includegraphics[width=0.32\columnwidth]{blockgroup-0650-net_shrunk.png}} & Missing \\
\end{tabular}
\imtitle{California 50}
\end{minipage}

\begin{minipage}{\columnwidth}
\begin{tabular}{lccc}
\raisebox{-0.5\height}{\includegraphics[width=0.32\columnwidth]{tract-0651-full_shrunk.png}} & \raisebox{-0.5\height}{\includegraphics[width=0.32\columnwidth]{blockgroup-0651-full_shrunk.png}} & Missing \\
\raisebox{-0.5\height}{\includegraphics[width=0.32\columnwidth]{tract-0651-net_shrunk.png}} & \raisebox{-0.5\height}{\includegraphics[width=0.32\columnwidth]{blockgroup-0651-net_shrunk.png}} & \raisebox{-0.5\height}{\includegraphics[width=0.32\columnwidth]{block-0651-net_shrunk.png}} \\
\end{tabular}
\imtitle{California 51}
\end{minipage}

\begin{minipage}{\columnwidth}
\begin{tabular}{lccc}
\raisebox{-0.5\height}{\includegraphics[width=0.32\columnwidth]{tract-0652-full_shrunk.png}} & \raisebox{-0.5\height}{\includegraphics[width=0.32\columnwidth]{blockgroup-0652-full_shrunk.png}} & Missing \\
\raisebox{-0.5\height}{\includegraphics[width=0.32\columnwidth]{tract-0652-net_shrunk.png}} & \raisebox{-0.5\height}{\includegraphics[width=0.32\columnwidth]{blockgroup-0652-net_shrunk.png}} & Missing \\
\end{tabular}
\imtitle{California 52}
\end{minipage}

\begin{minipage}{\columnwidth}
\begin{tabular}{lccc}
\raisebox{-0.5\height}{\includegraphics[width=0.32\columnwidth]{tract-0653-full_shrunk.png}} & \raisebox{-0.5\height}{\includegraphics[width=0.32\columnwidth]{blockgroup-0653-full_shrunk.png}} & Missing \\
\raisebox{-0.5\height}{\includegraphics[width=0.32\columnwidth]{tract-0653-net_shrunk.png}} & \raisebox{-0.5\height}{\includegraphics[width=0.32\columnwidth]{blockgroup-0653-net_shrunk.png}} & Missing \\
\end{tabular}
\imtitle{California 53}
\end{minipage}
\bchap{Colorado}
\begin{minipage}{\columnwidth}
\begin{tabular}{lccc}
\raisebox{-0.5\height}{\includegraphics[width=0.32\columnwidth]{tract-0801-full_shrunk.png}} & \raisebox{-0.5\height}{\includegraphics[width=0.32\columnwidth]{blockgroup-0801-full_shrunk.png}} & Missing \\
\raisebox{-0.5\height}{\includegraphics[width=0.32\columnwidth]{tract-0801-net_shrunk.png}} & \raisebox{-0.5\height}{\includegraphics[width=0.32\columnwidth]{blockgroup-0801-net_shrunk.png}} & Missing \\
\end{tabular}
\imtitle{Colorado 1}
\end{minipage}

\begin{minipage}{\columnwidth}
\begin{tabular}{lccc}
\raisebox{-0.5\height}{\includegraphics[width=0.32\columnwidth]{tract-0802-full_shrunk.png}} & \raisebox{-0.5\height}{\includegraphics[width=0.32\columnwidth]{blockgroup-0802-full_shrunk.png}} & Missing \\
\raisebox{-0.5\height}{\includegraphics[width=0.32\columnwidth]{tract-0802-net_shrunk.png}} & \raisebox{-0.5\height}{\includegraphics[width=0.32\columnwidth]{blockgroup-0802-net_shrunk.png}} & \raisebox{-0.5\height}{\includegraphics[width=0.32\columnwidth]{block-0802-net_shrunk.png}} \\
\end{tabular}
\imtitle{Colorado 2}
\end{minipage}

\begin{minipage}{\columnwidth}
\begin{tabular}{lccc}
\raisebox{-0.5\height}{\includegraphics[width=0.32\columnwidth]{tract-0803-full_shrunk.png}} & \raisebox{-0.5\height}{\includegraphics[width=0.32\columnwidth]{blockgroup-0803-full_shrunk.png}} & Missing \\
\raisebox{-0.5\height}{\includegraphics[width=0.32\columnwidth]{tract-0803-net_shrunk.png}} & \raisebox{-0.5\height}{\includegraphics[width=0.32\columnwidth]{blockgroup-0803-net_shrunk.png}} & \raisebox{-0.5\height}{\includegraphics[width=0.32\columnwidth]{block-0803-net_shrunk.png}} \\
\end{tabular}
\imtitle{Colorado 3}
\end{minipage}

\begin{minipage}{\columnwidth}
\begin{tabular}{lccc}
\raisebox{-0.5\height}{\includegraphics[width=0.32\columnwidth]{tract-0804-full_shrunk.png}} & \raisebox{-0.5\height}{\includegraphics[width=0.32\columnwidth]{blockgroup-0804-full_shrunk.png}} & Missing \\
\raisebox{-0.5\height}{\includegraphics[width=0.32\columnwidth]{tract-0804-net_shrunk.png}} & \raisebox{-0.5\height}{\includegraphics[width=0.32\columnwidth]{blockgroup-0804-net_shrunk.png}} & \raisebox{-0.5\height}{\includegraphics[width=0.32\columnwidth]{block-0804-net_shrunk.png}} \\
\end{tabular}
\imtitle{Colorado 4}
\end{minipage}

\begin{minipage}{\columnwidth}
\begin{tabular}{lccc}
\raisebox{-0.5\height}{\includegraphics[width=0.32\columnwidth]{tract-0805-full_shrunk.png}} & \raisebox{-0.5\height}{\includegraphics[width=0.32\columnwidth]{blockgroup-0805-full_shrunk.png}} & Missing \\
\raisebox{-0.5\height}{\includegraphics[width=0.32\columnwidth]{tract-0805-net_shrunk.png}} & \raisebox{-0.5\height}{\includegraphics[width=0.32\columnwidth]{blockgroup-0805-net_shrunk.png}} & \raisebox{-0.5\height}{\includegraphics[width=0.32\columnwidth]{block-0805-net_shrunk.png}} \\
\end{tabular}
\imtitle{Colorado 5}
\end{minipage}

\begin{minipage}{\columnwidth}
\begin{tabular}{lccc}
\raisebox{-0.5\height}{\includegraphics[width=0.32\columnwidth]{tract-0806-full_shrunk.png}} & \raisebox{-0.5\height}{\includegraphics[width=0.32\columnwidth]{blockgroup-0806-full_shrunk.png}} & Missing \\
\raisebox{-0.5\height}{\includegraphics[width=0.32\columnwidth]{tract-0806-net_shrunk.png}} & \raisebox{-0.5\height}{\includegraphics[width=0.32\columnwidth]{blockgroup-0806-net_shrunk.png}} & \raisebox{-0.5\height}{\includegraphics[width=0.32\columnwidth]{block-0806-net_shrunk.png}} \\
\end{tabular}
\imtitle{Colorado 6}
\end{minipage}

\begin{minipage}{\columnwidth}
\begin{tabular}{lccc}
\raisebox{-0.5\height}{\includegraphics[width=0.32\columnwidth]{tract-0807-full_shrunk.png}} & \raisebox{-0.5\height}{\includegraphics[width=0.32\columnwidth]{blockgroup-0807-full_shrunk.png}} & Missing \\
\raisebox{-0.5\height}{\includegraphics[width=0.32\columnwidth]{tract-0807-net_shrunk.png}} & \raisebox{-0.5\height}{\includegraphics[width=0.32\columnwidth]{blockgroup-0807-net_shrunk.png}} & Missing \\
\end{tabular}
\imtitle{Colorado 7}
\end{minipage}
\bchap{Connecticut}
\begin{minipage}{\columnwidth}
\begin{tabular}{lccc}
Missing & Missing & \raisebox{-0.5\height}{\includegraphics[width=0.32\columnwidth]{block-0901-full_shrunk.png}} \\
Missing & Missing & \raisebox{-0.5\height}{\includegraphics[width=0.32\columnwidth]{block-0901-net_shrunk.png}} \\
\end{tabular}
\imtitle{Connecticut 1}
\end{minipage}

\begin{minipage}{\columnwidth}
\begin{tabular}{lccc}
Missing & Missing & \raisebox{-0.5\height}{\includegraphics[width=0.32\columnwidth]{block-0902-full_shrunk.png}} \\
Missing & Missing & \raisebox{-0.5\height}{\includegraphics[width=0.32\columnwidth]{block-0902-net_shrunk.png}} \\
\end{tabular}
\imtitle{Connecticut 2}
\end{minipage}

\begin{minipage}{\columnwidth}
\begin{tabular}{lccc}
Missing & Missing & \raisebox{-0.5\height}{\includegraphics[width=0.32\columnwidth]{block-0903-full_shrunk.png}} \\
Missing & Missing & \raisebox{-0.5\height}{\includegraphics[width=0.32\columnwidth]{block-0903-net_shrunk.png}} \\
\end{tabular}
\imtitle{Connecticut 3}
\end{minipage}

\begin{minipage}{\columnwidth}
\begin{tabular}{lccc}
Missing & Missing & \raisebox{-0.5\height}{\includegraphics[width=0.32\columnwidth]{block-0904-full_shrunk.png}} \\
Missing & Missing & \raisebox{-0.5\height}{\includegraphics[width=0.32\columnwidth]{block-0904-net_shrunk.png}} \\
\end{tabular}
\imtitle{Connecticut 4}
\end{minipage}

\begin{minipage}{\columnwidth}
\begin{tabular}{lccc}
Missing & Missing & \raisebox{-0.5\height}{\includegraphics[width=0.32\columnwidth]{block-0905-full_shrunk.png}} \\
Missing & Missing & \raisebox{-0.5\height}{\includegraphics[width=0.32\columnwidth]{block-0905-net_shrunk.png}} \\
\end{tabular}
\imtitle{Connecticut 5}
\end{minipage}
\bchap{Delaware}
\begin{minipage}{\columnwidth}
\begin{tabular}{lccc}
\raisebox{-0.5\height}{\includegraphics[width=0.32\columnwidth]{tract-1000-full_shrunk.png}} & \raisebox{-0.5\height}{\includegraphics[width=0.32\columnwidth]{blockgroup-1000-full_shrunk.png}} & \raisebox{-0.5\height}{\includegraphics[width=0.32\columnwidth]{block-1000-full_shrunk.png}} \\
\raisebox{-0.5\height}{\includegraphics[width=0.32\columnwidth]{tract-1000-net_shrunk.png}} & \raisebox{-0.5\height}{\includegraphics[width=0.32\columnwidth]{blockgroup-1000-net_shrunk.png}} & \raisebox{-0.5\height}{\includegraphics[width=0.32\columnwidth]{block-1000-net_shrunk.png}} \\
\end{tabular}
\imtitle{Delaware 0}
\end{minipage}
\bchap{District of Columbia}
\begin{minipage}{\columnwidth}
\begin{tabular}{lccc}
\raisebox{-0.5\height}{\includegraphics[width=0.32\columnwidth]{tract-1198-full_shrunk.png}} & \raisebox{-0.5\height}{\includegraphics[width=0.32\columnwidth]{blockgroup-1198-full_shrunk.png}} & \raisebox{-0.5\height}{\includegraphics[width=0.32\columnwidth]{block-1198-full_shrunk.png}} \\
\raisebox{-0.5\height}{\includegraphics[width=0.32\columnwidth]{tract-1198-net_shrunk.png}} & \raisebox{-0.5\height}{\includegraphics[width=0.32\columnwidth]{blockgroup-1198-net_shrunk.png}} & \raisebox{-0.5\height}{\includegraphics[width=0.32\columnwidth]{block-1198-net_shrunk.png}} \\
\end{tabular}
\imtitle{District of Columbia 98}
\end{minipage}
\bchap{Florida}
\begin{minipage}{\columnwidth}
\begin{tabular}{lccc}
\raisebox{-0.5\height}{\includegraphics[width=0.32\columnwidth]{tract-1201-full_shrunk.png}} & \raisebox{-0.5\height}{\includegraphics[width=0.32\columnwidth]{blockgroup-1201-full_shrunk.png}} & Missing \\
\raisebox{-0.5\height}{\includegraphics[width=0.32\columnwidth]{tract-1201-net_shrunk.png}} & \raisebox{-0.5\height}{\includegraphics[width=0.32\columnwidth]{blockgroup-1201-net_shrunk.png}} & Missing \\
\end{tabular}
\imtitle{Florida 1}
\end{minipage}

\begin{minipage}{\columnwidth}
\begin{tabular}{lccc}
\raisebox{-0.5\height}{\includegraphics[width=0.32\columnwidth]{tract-1202-full_shrunk.png}} & \raisebox{-0.5\height}{\includegraphics[width=0.32\columnwidth]{blockgroup-1202-full_shrunk.png}} & Missing \\
\raisebox{-0.5\height}{\includegraphics[width=0.32\columnwidth]{tract-1202-net_shrunk.png}} & \raisebox{-0.5\height}{\includegraphics[width=0.32\columnwidth]{blockgroup-1202-net_shrunk.png}} & \raisebox{-0.5\height}{\includegraphics[width=0.32\columnwidth]{block-1202-net_shrunk.png}} \\
\end{tabular}
\imtitle{Florida 2}
\end{minipage}

\begin{minipage}{\columnwidth}
\begin{tabular}{lccc}
\raisebox{-0.5\height}{\includegraphics[width=0.32\columnwidth]{tract-1203-full_shrunk.png}} & \raisebox{-0.5\height}{\includegraphics[width=0.32\columnwidth]{blockgroup-1203-full_shrunk.png}} & Missing \\
\raisebox{-0.5\height}{\includegraphics[width=0.32\columnwidth]{tract-1203-net_shrunk.png}} & \raisebox{-0.5\height}{\includegraphics[width=0.32\columnwidth]{blockgroup-1203-net_shrunk.png}} & \raisebox{-0.5\height}{\includegraphics[width=0.32\columnwidth]{block-1203-net_shrunk.png}} \\
\end{tabular}
\imtitle{Florida 3}
\end{minipage}

\begin{minipage}{\columnwidth}
\begin{tabular}{lccc}
\raisebox{-0.5\height}{\includegraphics[width=0.32\columnwidth]{tract-1204-full_shrunk.png}} & \raisebox{-0.5\height}{\includegraphics[width=0.32\columnwidth]{blockgroup-1204-full_shrunk.png}} & Missing \\
\raisebox{-0.5\height}{\includegraphics[width=0.32\columnwidth]{tract-1204-net_shrunk.png}} & \raisebox{-0.5\height}{\includegraphics[width=0.32\columnwidth]{blockgroup-1204-net_shrunk.png}} & \raisebox{-0.5\height}{\includegraphics[width=0.32\columnwidth]{block-1204-net_shrunk.png}} \\
\end{tabular}
\imtitle{Florida 4}
\end{minipage}

\begin{minipage}{\columnwidth}
\begin{tabular}{lccc}
\raisebox{-0.5\height}{\includegraphics[width=0.32\columnwidth]{tract-1205-full_shrunk.png}} & \raisebox{-0.5\height}{\includegraphics[width=0.32\columnwidth]{blockgroup-1205-full_shrunk.png}} & Missing \\
\raisebox{-0.5\height}{\includegraphics[width=0.32\columnwidth]{tract-1205-net_shrunk.png}} & \raisebox{-0.5\height}{\includegraphics[width=0.32\columnwidth]{blockgroup-1205-net_shrunk.png}} & \raisebox{-0.5\height}{\includegraphics[width=0.32\columnwidth]{block-1205-net_shrunk.png}} \\
\end{tabular}
\imtitle{Florida 5}
\end{minipage}

\begin{minipage}{\columnwidth}
\begin{tabular}{lccc}
\raisebox{-0.5\height}{\includegraphics[width=0.32\columnwidth]{tract-1206-full_shrunk.png}} & \raisebox{-0.5\height}{\includegraphics[width=0.32\columnwidth]{blockgroup-1206-full_shrunk.png}} & Missing \\
\raisebox{-0.5\height}{\includegraphics[width=0.32\columnwidth]{tract-1206-net_shrunk.png}} & \raisebox{-0.5\height}{\includegraphics[width=0.32\columnwidth]{blockgroup-1206-net_shrunk.png}} & \raisebox{-0.5\height}{\includegraphics[width=0.32\columnwidth]{block-1206-net_shrunk.png}} \\
\end{tabular}
\imtitle{Florida 6}
\end{minipage}

\begin{minipage}{\columnwidth}
\begin{tabular}{lccc}
\raisebox{-0.5\height}{\includegraphics[width=0.32\columnwidth]{tract-1207-full_shrunk.png}} & \raisebox{-0.5\height}{\includegraphics[width=0.32\columnwidth]{blockgroup-1207-full_shrunk.png}} & Missing \\
\raisebox{-0.5\height}{\includegraphics[width=0.32\columnwidth]{tract-1207-net_shrunk.png}} & \raisebox{-0.5\height}{\includegraphics[width=0.32\columnwidth]{blockgroup-1207-net_shrunk.png}} & Missing \\
\end{tabular}
\imtitle{Florida 7}
\end{minipage}

\begin{minipage}{\columnwidth}
\begin{tabular}{lccc}
\raisebox{-0.5\height}{\includegraphics[width=0.32\columnwidth]{tract-1208-full_shrunk.png}} & \raisebox{-0.5\height}{\includegraphics[width=0.32\columnwidth]{blockgroup-1208-full_shrunk.png}} & Missing \\
\raisebox{-0.5\height}{\includegraphics[width=0.32\columnwidth]{tract-1208-net_shrunk.png}} & \raisebox{-0.5\height}{\includegraphics[width=0.32\columnwidth]{blockgroup-1208-net_shrunk.png}} & Missing \\
\end{tabular}
\imtitle{Florida 8}
\end{minipage}

\begin{minipage}{\columnwidth}
\begin{tabular}{lccc}
\raisebox{-0.5\height}{\includegraphics[width=0.32\columnwidth]{tract-1209-full_shrunk.png}} & \raisebox{-0.5\height}{\includegraphics[width=0.32\columnwidth]{blockgroup-1209-full_shrunk.png}} & Missing \\
\raisebox{-0.5\height}{\includegraphics[width=0.32\columnwidth]{tract-1209-net_shrunk.png}} & \raisebox{-0.5\height}{\includegraphics[width=0.32\columnwidth]{blockgroup-1209-net_shrunk.png}} & Missing \\
\end{tabular}
\imtitle{Florida 9}
\end{minipage}

\begin{minipage}{\columnwidth}
\begin{tabular}{lccc}
\raisebox{-0.5\height}{\includegraphics[width=0.32\columnwidth]{tract-1210-full_shrunk.png}} & \raisebox{-0.5\height}{\includegraphics[width=0.32\columnwidth]{blockgroup-1210-full_shrunk.png}} & Missing \\
\raisebox{-0.5\height}{\includegraphics[width=0.32\columnwidth]{tract-1210-net_shrunk.png}} & \raisebox{-0.5\height}{\includegraphics[width=0.32\columnwidth]{blockgroup-1210-net_shrunk.png}} & Missing \\
\end{tabular}
\imtitle{Florida 10}
\end{minipage}

\begin{minipage}{\columnwidth}
\begin{tabular}{lccc}
\raisebox{-0.5\height}{\includegraphics[width=0.32\columnwidth]{tract-1211-full_shrunk.png}} & \raisebox{-0.5\height}{\includegraphics[width=0.32\columnwidth]{blockgroup-1211-full_shrunk.png}} & Missing \\
\raisebox{-0.5\height}{\includegraphics[width=0.32\columnwidth]{tract-1211-net_shrunk.png}} & \raisebox{-0.5\height}{\includegraphics[width=0.32\columnwidth]{blockgroup-1211-net_shrunk.png}} & Missing \\
\end{tabular}
\imtitle{Florida 11}
\end{minipage}

\begin{minipage}{\columnwidth}
\begin{tabular}{lccc}
\raisebox{-0.5\height}{\includegraphics[width=0.32\columnwidth]{tract-1212-full_shrunk.png}} & \raisebox{-0.5\height}{\includegraphics[width=0.32\columnwidth]{blockgroup-1212-full_shrunk.png}} & Missing \\
\raisebox{-0.5\height}{\includegraphics[width=0.32\columnwidth]{tract-1212-net_shrunk.png}} & \raisebox{-0.5\height}{\includegraphics[width=0.32\columnwidth]{blockgroup-1212-net_shrunk.png}} & Missing \\
\end{tabular}
\imtitle{Florida 12}
\end{minipage}

\begin{minipage}{\columnwidth}
\begin{tabular}{lccc}
\raisebox{-0.5\height}{\includegraphics[width=0.32\columnwidth]{tract-1213-full_shrunk.png}} & \raisebox{-0.5\height}{\includegraphics[width=0.32\columnwidth]{blockgroup-1213-full_shrunk.png}} & Missing \\
\raisebox{-0.5\height}{\includegraphics[width=0.32\columnwidth]{tract-1213-net_shrunk.png}} & \raisebox{-0.5\height}{\includegraphics[width=0.32\columnwidth]{blockgroup-1213-net_shrunk.png}} & Missing \\
\end{tabular}
\imtitle{Florida 13}
\end{minipage}

\begin{minipage}{\columnwidth}
\begin{tabular}{lccc}
\raisebox{-0.5\height}{\includegraphics[width=0.32\columnwidth]{tract-1214-full_shrunk.png}} & \raisebox{-0.5\height}{\includegraphics[width=0.32\columnwidth]{blockgroup-1214-full_shrunk.png}} & Missing \\
\raisebox{-0.5\height}{\includegraphics[width=0.32\columnwidth]{tract-1214-net_shrunk.png}} & \raisebox{-0.5\height}{\includegraphics[width=0.32\columnwidth]{blockgroup-1214-net_shrunk.png}} & Missing \\
\end{tabular}
\imtitle{Florida 14}
\end{minipage}

\begin{minipage}{\columnwidth}
\begin{tabular}{lccc}
\raisebox{-0.5\height}{\includegraphics[width=0.32\columnwidth]{tract-1215-full_shrunk.png}} & \raisebox{-0.5\height}{\includegraphics[width=0.32\columnwidth]{blockgroup-1215-full_shrunk.png}} & Missing \\
\raisebox{-0.5\height}{\includegraphics[width=0.32\columnwidth]{tract-1215-net_shrunk.png}} & \raisebox{-0.5\height}{\includegraphics[width=0.32\columnwidth]{blockgroup-1215-net_shrunk.png}} & \raisebox{-0.5\height}{\includegraphics[width=0.32\columnwidth]{block-1215-net_shrunk.png}} \\
\end{tabular}
\imtitle{Florida 15}
\end{minipage}

\begin{minipage}{\columnwidth}
\begin{tabular}{lccc}
\raisebox{-0.5\height}{\includegraphics[width=0.32\columnwidth]{tract-1216-full_shrunk.png}} & \raisebox{-0.5\height}{\includegraphics[width=0.32\columnwidth]{blockgroup-1216-full_shrunk.png}} & Missing \\
\raisebox{-0.5\height}{\includegraphics[width=0.32\columnwidth]{tract-1216-net_shrunk.png}} & \raisebox{-0.5\height}{\includegraphics[width=0.32\columnwidth]{blockgroup-1216-net_shrunk.png}} & \raisebox{-0.5\height}{\includegraphics[width=0.32\columnwidth]{block-1216-net_shrunk.png}} \\
\end{tabular}
\imtitle{Florida 16}
\end{minipage}

\begin{minipage}{\columnwidth}
\begin{tabular}{lccc}
\raisebox{-0.5\height}{\includegraphics[width=0.32\columnwidth]{tract-1217-full_shrunk.png}} & \raisebox{-0.5\height}{\includegraphics[width=0.32\columnwidth]{blockgroup-1217-full_shrunk.png}} & Missing \\
\raisebox{-0.5\height}{\includegraphics[width=0.32\columnwidth]{tract-1217-net_shrunk.png}} & \raisebox{-0.5\height}{\includegraphics[width=0.32\columnwidth]{blockgroup-1217-net_shrunk.png}} & Missing \\
\end{tabular}
\imtitle{Florida 17}
\end{minipage}

\begin{minipage}{\columnwidth}
\begin{tabular}{lccc}
\raisebox{-0.5\height}{\includegraphics[width=0.32\columnwidth]{tract-1218-full_shrunk.png}} & \raisebox{-0.5\height}{\includegraphics[width=0.32\columnwidth]{blockgroup-1218-full_shrunk.png}} & Missing \\
\raisebox{-0.5\height}{\includegraphics[width=0.32\columnwidth]{tract-1218-net_shrunk.png}} & \raisebox{-0.5\height}{\includegraphics[width=0.32\columnwidth]{blockgroup-1218-net_shrunk.png}} & \raisebox{-0.5\height}{\includegraphics[width=0.32\columnwidth]{block-1218-net_shrunk.png}} \\
\end{tabular}
\imtitle{Florida 18}
\end{minipage}

\begin{minipage}{\columnwidth}
\begin{tabular}{lccc}
\raisebox{-0.5\height}{\includegraphics[width=0.32\columnwidth]{tract-1219-full_shrunk.png}} & \raisebox{-0.5\height}{\includegraphics[width=0.32\columnwidth]{blockgroup-1219-full_shrunk.png}} & Missing \\
\raisebox{-0.5\height}{\includegraphics[width=0.32\columnwidth]{tract-1219-net_shrunk.png}} & \raisebox{-0.5\height}{\includegraphics[width=0.32\columnwidth]{blockgroup-1219-net_shrunk.png}} & Missing \\
\end{tabular}
\imtitle{Florida 19}
\end{minipage}

\begin{minipage}{\columnwidth}
\begin{tabular}{lccc}
\raisebox{-0.5\height}{\includegraphics[width=0.32\columnwidth]{tract-1220-full_shrunk.png}} & \raisebox{-0.5\height}{\includegraphics[width=0.32\columnwidth]{blockgroup-1220-full_shrunk.png}} & Missing \\
\raisebox{-0.5\height}{\includegraphics[width=0.32\columnwidth]{tract-1220-net_shrunk.png}} & \raisebox{-0.5\height}{\includegraphics[width=0.32\columnwidth]{blockgroup-1220-net_shrunk.png}} & Missing \\
\end{tabular}
\imtitle{Florida 20}
\end{minipage}

\begin{minipage}{\columnwidth}
\begin{tabular}{lccc}
\raisebox{-0.5\height}{\includegraphics[width=0.32\columnwidth]{tract-1221-full_shrunk.png}} & \raisebox{-0.5\height}{\includegraphics[width=0.32\columnwidth]{blockgroup-1221-full_shrunk.png}} & Missing \\
\raisebox{-0.5\height}{\includegraphics[width=0.32\columnwidth]{tract-1221-net_shrunk.png}} & \raisebox{-0.5\height}{\includegraphics[width=0.32\columnwidth]{blockgroup-1221-net_shrunk.png}} & Missing \\
\end{tabular}
\imtitle{Florida 21}
\end{minipage}

\begin{minipage}{\columnwidth}
\begin{tabular}{lccc}
\raisebox{-0.5\height}{\includegraphics[width=0.32\columnwidth]{tract-1222-full_shrunk.png}} & \raisebox{-0.5\height}{\includegraphics[width=0.32\columnwidth]{blockgroup-1222-full_shrunk.png}} & Missing \\
\raisebox{-0.5\height}{\includegraphics[width=0.32\columnwidth]{tract-1222-net_shrunk.png}} & \raisebox{-0.5\height}{\includegraphics[width=0.32\columnwidth]{blockgroup-1222-net_shrunk.png}} & Missing \\
\end{tabular}
\imtitle{Florida 22}
\end{minipage}

\begin{minipage}{\columnwidth}
\begin{tabular}{lccc}
\raisebox{-0.5\height}{\includegraphics[width=0.32\columnwidth]{tract-1223-full_shrunk.png}} & \raisebox{-0.5\height}{\includegraphics[width=0.32\columnwidth]{blockgroup-1223-full_shrunk.png}} & Missing \\
\raisebox{-0.5\height}{\includegraphics[width=0.32\columnwidth]{tract-1223-net_shrunk.png}} & \raisebox{-0.5\height}{\includegraphics[width=0.32\columnwidth]{blockgroup-1223-net_shrunk.png}} & Missing \\
\end{tabular}
\imtitle{Florida 23}
\end{minipage}

\begin{minipage}{\columnwidth}
\begin{tabular}{lccc}
\raisebox{-0.5\height}{\includegraphics[width=0.32\columnwidth]{tract-1224-full_shrunk.png}} & \raisebox{-0.5\height}{\includegraphics[width=0.32\columnwidth]{blockgroup-1224-full_shrunk.png}} & Missing \\
\raisebox{-0.5\height}{\includegraphics[width=0.32\columnwidth]{tract-1224-net_shrunk.png}} & \raisebox{-0.5\height}{\includegraphics[width=0.32\columnwidth]{blockgroup-1224-net_shrunk.png}} & \raisebox{-0.5\height}{\includegraphics[width=0.32\columnwidth]{block-1224-net_shrunk.png}} \\
\end{tabular}
\imtitle{Florida 24}
\end{minipage}

\begin{minipage}{\columnwidth}
\begin{tabular}{lccc}
\raisebox{-0.5\height}{\includegraphics[width=0.32\columnwidth]{tract-1225-full_shrunk.png}} & \raisebox{-0.5\height}{\includegraphics[width=0.32\columnwidth]{blockgroup-1225-full_shrunk.png}} & Missing \\
\raisebox{-0.5\height}{\includegraphics[width=0.32\columnwidth]{tract-1225-net_shrunk.png}} & \raisebox{-0.5\height}{\includegraphics[width=0.32\columnwidth]{blockgroup-1225-net_shrunk.png}} & Missing \\
\end{tabular}
\imtitle{Florida 25}
\end{minipage}

\begin{minipage}{\columnwidth}
\begin{tabular}{lccc}
Missing & Missing & Missing \\
Missing & Missing & Missing \\
\end{tabular}
\imtitle{Florida 26}
\end{minipage}

\begin{minipage}{\columnwidth}
\begin{tabular}{lccc}
Missing & Missing & Missing \\
Missing & Missing & Missing \\
\end{tabular}
\imtitle{Florida 27}
\end{minipage}
\bchap{Georgia}
\begin{minipage}{\columnwidth}
\begin{tabular}{lccc}
\raisebox{-0.5\height}{\includegraphics[width=0.32\columnwidth]{tract-1301-full_shrunk.png}} & \raisebox{-0.5\height}{\includegraphics[width=0.32\columnwidth]{blockgroup-1301-full_shrunk.png}} & \raisebox{-0.5\height}{\includegraphics[width=0.32\columnwidth]{block-1301-full_shrunk.png}} \\
\raisebox{-0.5\height}{\includegraphics[width=0.32\columnwidth]{tract-1301-net_shrunk.png}} & \raisebox{-0.5\height}{\includegraphics[width=0.32\columnwidth]{blockgroup-1301-net_shrunk.png}} & Missing \\
\end{tabular}
\imtitle{Georgia 1}
\end{minipage}

\begin{minipage}{\columnwidth}
\begin{tabular}{lccc}
\raisebox{-0.5\height}{\includegraphics[width=0.32\columnwidth]{tract-1302-full_shrunk.png}} & \raisebox{-0.5\height}{\includegraphics[width=0.32\columnwidth]{blockgroup-1302-full_shrunk.png}} & Missing \\
\raisebox{-0.5\height}{\includegraphics[width=0.32\columnwidth]{tract-1302-net_shrunk.png}} & \raisebox{-0.5\height}{\includegraphics[width=0.32\columnwidth]{blockgroup-1302-net_shrunk.png}} & \raisebox{-0.5\height}{\includegraphics[width=0.32\columnwidth]{block-1302-net_shrunk.png}} \\
\end{tabular}
\imtitle{Georgia 2}
\end{minipage}

\begin{minipage}{\columnwidth}
\begin{tabular}{lccc}
\raisebox{-0.5\height}{\includegraphics[width=0.32\columnwidth]{tract-1303-full_shrunk.png}} & \raisebox{-0.5\height}{\includegraphics[width=0.32\columnwidth]{blockgroup-1303-full_shrunk.png}} & Missing \\
\raisebox{-0.5\height}{\includegraphics[width=0.32\columnwidth]{tract-1303-net_shrunk.png}} & \raisebox{-0.5\height}{\includegraphics[width=0.32\columnwidth]{blockgroup-1303-net_shrunk.png}} & \raisebox{-0.5\height}{\includegraphics[width=0.32\columnwidth]{block-1303-net_shrunk.png}} \\
\end{tabular}
\imtitle{Georgia 3}
\end{minipage}

\begin{minipage}{\columnwidth}
\begin{tabular}{lccc}
\raisebox{-0.5\height}{\includegraphics[width=0.32\columnwidth]{tract-1304-full_shrunk.png}} & \raisebox{-0.5\height}{\includegraphics[width=0.32\columnwidth]{blockgroup-1304-full_shrunk.png}} & Missing \\
\raisebox{-0.5\height}{\includegraphics[width=0.32\columnwidth]{tract-1304-net_shrunk.png}} & \raisebox{-0.5\height}{\includegraphics[width=0.32\columnwidth]{blockgroup-1304-net_shrunk.png}} & \raisebox{-0.5\height}{\includegraphics[width=0.32\columnwidth]{block-1304-net_shrunk.png}} \\
\end{tabular}
\imtitle{Georgia 4}
\end{minipage}

\begin{minipage}{\columnwidth}
\begin{tabular}{lccc}
\raisebox{-0.5\height}{\includegraphics[width=0.32\columnwidth]{tract-1305-full_shrunk.png}} & \raisebox{-0.5\height}{\includegraphics[width=0.32\columnwidth]{blockgroup-1305-full_shrunk.png}} & Missing \\
\raisebox{-0.5\height}{\includegraphics[width=0.32\columnwidth]{tract-1305-net_shrunk.png}} & \raisebox{-0.5\height}{\includegraphics[width=0.32\columnwidth]{blockgroup-1305-net_shrunk.png}} & \raisebox{-0.5\height}{\includegraphics[width=0.32\columnwidth]{block-1305-net_shrunk.png}} \\
\end{tabular}
\imtitle{Georgia 5}
\end{minipage}

\begin{minipage}{\columnwidth}
\begin{tabular}{lccc}
\raisebox{-0.5\height}{\includegraphics[width=0.32\columnwidth]{tract-1306-full_shrunk.png}} & \raisebox{-0.5\height}{\includegraphics[width=0.32\columnwidth]{blockgroup-1306-full_shrunk.png}} & Missing \\
\raisebox{-0.5\height}{\includegraphics[width=0.32\columnwidth]{tract-1306-net_shrunk.png}} & \raisebox{-0.5\height}{\includegraphics[width=0.32\columnwidth]{blockgroup-1306-net_shrunk.png}} & \raisebox{-0.5\height}{\includegraphics[width=0.32\columnwidth]{block-1306-net_shrunk.png}} \\
\end{tabular}
\imtitle{Georgia 6}
\end{minipage}

\begin{minipage}{\columnwidth}
\begin{tabular}{lccc}
\raisebox{-0.5\height}{\includegraphics[width=0.32\columnwidth]{tract-1307-full_shrunk.png}} & \raisebox{-0.5\height}{\includegraphics[width=0.32\columnwidth]{blockgroup-1307-full_shrunk.png}} & Missing \\
\raisebox{-0.5\height}{\includegraphics[width=0.32\columnwidth]{tract-1307-net_shrunk.png}} & \raisebox{-0.5\height}{\includegraphics[width=0.32\columnwidth]{blockgroup-1307-net_shrunk.png}} & \raisebox{-0.5\height}{\includegraphics[width=0.32\columnwidth]{block-1307-net_shrunk.png}} \\
\end{tabular}
\imtitle{Georgia 7}
\end{minipage}

\begin{minipage}{\columnwidth}
\begin{tabular}{lccc}
\raisebox{-0.5\height}{\includegraphics[width=0.32\columnwidth]{tract-1308-full_shrunk.png}} & \raisebox{-0.5\height}{\includegraphics[width=0.32\columnwidth]{blockgroup-1308-full_shrunk.png}} & Missing \\
\raisebox{-0.5\height}{\includegraphics[width=0.32\columnwidth]{tract-1308-net_shrunk.png}} & \raisebox{-0.5\height}{\includegraphics[width=0.32\columnwidth]{blockgroup-1308-net_shrunk.png}} & \raisebox{-0.5\height}{\includegraphics[width=0.32\columnwidth]{block-1308-net_shrunk.png}} \\
\end{tabular}
\imtitle{Georgia 8}
\end{minipage}

\begin{minipage}{\columnwidth}
\begin{tabular}{lccc}
\raisebox{-0.5\height}{\includegraphics[width=0.32\columnwidth]{tract-1309-full_shrunk.png}} & \raisebox{-0.5\height}{\includegraphics[width=0.32\columnwidth]{blockgroup-1309-full_shrunk.png}} & \raisebox{-0.5\height}{\includegraphics[width=0.32\columnwidth]{block-1309-full_shrunk.png}} \\
\raisebox{-0.5\height}{\includegraphics[width=0.32\columnwidth]{tract-1309-net_shrunk.png}} & \raisebox{-0.5\height}{\includegraphics[width=0.32\columnwidth]{blockgroup-1309-net_shrunk.png}} & \raisebox{-0.5\height}{\includegraphics[width=0.32\columnwidth]{block-1309-net_shrunk.png}} \\
\end{tabular}
\imtitle{Georgia 9}
\end{minipage}

\begin{minipage}{\columnwidth}
\begin{tabular}{lccc}
\raisebox{-0.5\height}{\includegraphics[width=0.32\columnwidth]{tract-1310-full_shrunk.png}} & \raisebox{-0.5\height}{\includegraphics[width=0.32\columnwidth]{blockgroup-1310-full_shrunk.png}} & Missing \\
\raisebox{-0.5\height}{\includegraphics[width=0.32\columnwidth]{tract-1310-net_shrunk.png}} & \raisebox{-0.5\height}{\includegraphics[width=0.32\columnwidth]{blockgroup-1310-net_shrunk.png}} & \raisebox{-0.5\height}{\includegraphics[width=0.32\columnwidth]{block-1310-net_shrunk.png}} \\
\end{tabular}
\imtitle{Georgia 10}
\end{minipage}

\begin{minipage}{\columnwidth}
\begin{tabular}{lccc}
\raisebox{-0.5\height}{\includegraphics[width=0.32\columnwidth]{tract-1311-full_shrunk.png}} & \raisebox{-0.5\height}{\includegraphics[width=0.32\columnwidth]{blockgroup-1311-full_shrunk.png}} & Missing \\
\raisebox{-0.5\height}{\includegraphics[width=0.32\columnwidth]{tract-1311-net_shrunk.png}} & \raisebox{-0.5\height}{\includegraphics[width=0.32\columnwidth]{blockgroup-1311-net_shrunk.png}} & \raisebox{-0.5\height}{\includegraphics[width=0.32\columnwidth]{block-1311-net_shrunk.png}} \\
\end{tabular}
\imtitle{Georgia 11}
\end{minipage}

\begin{minipage}{\columnwidth}
\begin{tabular}{lccc}
\raisebox{-0.5\height}{\includegraphics[width=0.32\columnwidth]{tract-1312-full_shrunk.png}} & \raisebox{-0.5\height}{\includegraphics[width=0.32\columnwidth]{blockgroup-1312-full_shrunk.png}} & Missing \\
\raisebox{-0.5\height}{\includegraphics[width=0.32\columnwidth]{tract-1312-net_shrunk.png}} & \raisebox{-0.5\height}{\includegraphics[width=0.32\columnwidth]{blockgroup-1312-net_shrunk.png}} & \raisebox{-0.5\height}{\includegraphics[width=0.32\columnwidth]{block-1312-net_shrunk.png}} \\
\end{tabular}
\imtitle{Georgia 12}
\end{minipage}

\begin{minipage}{\columnwidth}
\begin{tabular}{lccc}
\raisebox{-0.5\height}{\includegraphics[width=0.32\columnwidth]{tract-1313-full_shrunk.png}} & \raisebox{-0.5\height}{\includegraphics[width=0.32\columnwidth]{blockgroup-1313-full_shrunk.png}} & Missing \\
\raisebox{-0.5\height}{\includegraphics[width=0.32\columnwidth]{tract-1313-net_shrunk.png}} & \raisebox{-0.5\height}{\includegraphics[width=0.32\columnwidth]{blockgroup-1313-net_shrunk.png}} & \raisebox{-0.5\height}{\includegraphics[width=0.32\columnwidth]{block-1313-net_shrunk.png}} \\
\end{tabular}
\imtitle{Georgia 13}
\end{minipage}

\begin{minipage}{\columnwidth}
\begin{tabular}{lccc}
Missing & Missing & Missing \\
Missing & Missing & Missing \\
\end{tabular}
\imtitle{Georgia 14}
\end{minipage}
\bchap{Hawaii}
\begin{minipage}{\columnwidth}
\begin{tabular}{lccc}
\raisebox{-0.5\height}{\includegraphics[width=0.32\columnwidth]{tract-1501-full_shrunk.png}} & \raisebox{-0.5\height}{\includegraphics[width=0.32\columnwidth]{blockgroup-1501-full_shrunk.png}} & \raisebox{-0.5\height}{\includegraphics[width=0.32\columnwidth]{block-1501-full_shrunk.png}} \\
\raisebox{-0.5\height}{\includegraphics[width=0.32\columnwidth]{tract-1501-net_shrunk.png}} & \raisebox{-0.5\height}{\includegraphics[width=0.32\columnwidth]{blockgroup-1501-net_shrunk.png}} & \raisebox{-0.5\height}{\includegraphics[width=0.32\columnwidth]{block-1501-net_shrunk.png}} \\
\end{tabular}
\imtitle{Hawaii 1}
\end{minipage}

\begin{minipage}{\columnwidth}
\begin{tabular}{lccc}
\raisebox{-0.5\height}{\includegraphics[width=0.32\columnwidth]{tract-1502-full_shrunk.png}} & \raisebox{-0.5\height}{\includegraphics[width=0.32\columnwidth]{blockgroup-1502-full_shrunk.png}} & \raisebox{-0.5\height}{\includegraphics[width=0.32\columnwidth]{block-1502-full_shrunk.png}} \\
\raisebox{-0.5\height}{\includegraphics[width=0.32\columnwidth]{tract-1502-net_shrunk.png}} & \raisebox{-0.5\height}{\includegraphics[width=0.32\columnwidth]{blockgroup-1502-net_shrunk.png}} & \raisebox{-0.5\height}{\includegraphics[width=0.32\columnwidth]{block-1502-net_shrunk.png}} \\
\end{tabular}
\imtitle{Hawaii 2}
\end{minipage}
\bchap{Idaho}
\begin{minipage}{\columnwidth}
\begin{tabular}{lccc}
\raisebox{-0.5\height}{\includegraphics[width=0.32\columnwidth]{tract-1601-full_shrunk.png}} & \raisebox{-0.5\height}{\includegraphics[width=0.32\columnwidth]{blockgroup-1601-full_shrunk.png}} & Missing \\
\raisebox{-0.5\height}{\includegraphics[width=0.32\columnwidth]{tract-1601-net_shrunk.png}} & \raisebox{-0.5\height}{\includegraphics[width=0.32\columnwidth]{blockgroup-1601-net_shrunk.png}} & \raisebox{-0.5\height}{\includegraphics[width=0.32\columnwidth]{block-1601-net_shrunk.png}} \\
\end{tabular}
\imtitle{Idaho 1}
\end{minipage}

\begin{minipage}{\columnwidth}
\begin{tabular}{lccc}
\raisebox{-0.5\height}{\includegraphics[width=0.32\columnwidth]{tract-1602-full_shrunk.png}} & \raisebox{-0.5\height}{\includegraphics[width=0.32\columnwidth]{blockgroup-1602-full_shrunk.png}} & Missing \\
\raisebox{-0.5\height}{\includegraphics[width=0.32\columnwidth]{tract-1602-net_shrunk.png}} & \raisebox{-0.5\height}{\includegraphics[width=0.32\columnwidth]{blockgroup-1602-net_shrunk.png}} & \raisebox{-0.5\height}{\includegraphics[width=0.32\columnwidth]{block-1602-net_shrunk.png}} \\
\end{tabular}
\imtitle{Idaho 2}
\end{minipage}
\bchap{Illinois}
\begin{minipage}{\columnwidth}
\begin{tabular}{lccc}
Missing & Missing & Missing \\
Missing & Missing & Missing \\
\end{tabular}
\imtitle{Illinois 1}
\end{minipage}

\begin{minipage}{\columnwidth}
\begin{tabular}{lccc}
Missing & Missing & Missing \\
Missing & Missing & Missing \\
\end{tabular}
\imtitle{Illinois 2}
\end{minipage}

\begin{minipage}{\columnwidth}
\begin{tabular}{lccc}
Missing & Missing & Missing \\
Missing & Missing & Missing \\
\end{tabular}
\imtitle{Illinois 3}
\end{minipage}

\begin{minipage}{\columnwidth}
\begin{tabular}{lccc}
Missing & Missing & Missing \\
Missing & Missing & Missing \\
\end{tabular}
\imtitle{Illinois 4}
\end{minipage}

\begin{minipage}{\columnwidth}
\begin{tabular}{lccc}
Missing & Missing & Missing \\
Missing & Missing & Missing \\
\end{tabular}
\imtitle{Illinois 5}
\end{minipage}

\begin{minipage}{\columnwidth}
\begin{tabular}{lccc}
Missing & Missing & \raisebox{-0.5\height}{\includegraphics[width=0.32\columnwidth]{block-1706-full_shrunk.png}} \\
Missing & Missing & \raisebox{-0.5\height}{\includegraphics[width=0.32\columnwidth]{block-1706-net_shrunk.png}} \\
\end{tabular}
\imtitle{Illinois 6}
\end{minipage}

\begin{minipage}{\columnwidth}
\begin{tabular}{lccc}
Missing & Missing & Missing \\
Missing & Missing & Missing \\
\end{tabular}
\imtitle{Illinois 7}
\end{minipage}

\begin{minipage}{\columnwidth}
\begin{tabular}{lccc}
Missing & Missing & Missing \\
Missing & Missing & Missing \\
\end{tabular}
\imtitle{Illinois 8}
\end{minipage}

\begin{minipage}{\columnwidth}
\begin{tabular}{lccc}
Missing & Missing & Missing \\
Missing & Missing & Missing \\
\end{tabular}
\imtitle{Illinois 9}
\end{minipage}

\begin{minipage}{\columnwidth}
\begin{tabular}{lccc}
Missing & Missing & Missing \\
Missing & Missing & Missing \\
\end{tabular}
\imtitle{Illinois 10}
\end{minipage}

\begin{minipage}{\columnwidth}
\begin{tabular}{lccc}
Missing & Missing & \raisebox{-0.5\height}{\includegraphics[width=0.32\columnwidth]{block-1711-full_shrunk.png}} \\
Missing & Missing & \raisebox{-0.5\height}{\includegraphics[width=0.32\columnwidth]{block-1711-net_shrunk.png}} \\
\end{tabular}
\imtitle{Illinois 11}
\end{minipage}

\begin{minipage}{\columnwidth}
\begin{tabular}{lccc}
Missing & Missing & Missing \\
Missing & Missing & \raisebox{-0.5\height}{\includegraphics[width=0.32\columnwidth]{block-1712-net_shrunk.png}} \\
\end{tabular}
\imtitle{Illinois 12}
\end{minipage}

\begin{minipage}{\columnwidth}
\begin{tabular}{lccc}
Missing & Missing & \raisebox{-0.5\height}{\includegraphics[width=0.32\columnwidth]{block-1713-full_shrunk.png}} \\
Missing & Missing & \raisebox{-0.5\height}{\includegraphics[width=0.32\columnwidth]{block-1713-net_shrunk.png}} \\
\end{tabular}
\imtitle{Illinois 13}
\end{minipage}

\begin{minipage}{\columnwidth}
\begin{tabular}{lccc}
Missing & Missing & Missing \\
Missing & Missing & \raisebox{-0.5\height}{\includegraphics[width=0.32\columnwidth]{block-1714-net_shrunk.png}} \\
\end{tabular}
\imtitle{Illinois 14}
\end{minipage}

\begin{minipage}{\columnwidth}
\begin{tabular}{lccc}
Missing & Missing & Missing \\
Missing & Missing & Missing \\
\end{tabular}
\imtitle{Illinois 15}
\end{minipage}

\begin{minipage}{\columnwidth}
\begin{tabular}{lccc}
Missing & Missing & Missing \\
Missing & Missing & Missing \\
\end{tabular}
\imtitle{Illinois 16}
\end{minipage}

\begin{minipage}{\columnwidth}
\begin{tabular}{lccc}
Missing & Missing & Missing \\
Missing & Missing & Missing \\
\end{tabular}
\imtitle{Illinois 17}
\end{minipage}

\begin{minipage}{\columnwidth}
\begin{tabular}{lccc}
Missing & Missing & Missing \\
Missing & Missing & \raisebox{-0.5\height}{\includegraphics[width=0.32\columnwidth]{block-1718-net_shrunk.png}} \\
\end{tabular}
\imtitle{Illinois 18}
\end{minipage}
\bchap{Indiana}
\begin{minipage}{\columnwidth}
\begin{tabular}{lccc}
\raisebox{-0.5\height}{\includegraphics[width=0.32\columnwidth]{tract-1801-full_shrunk.png}} & \raisebox{-0.5\height}{\includegraphics[width=0.32\columnwidth]{blockgroup-1801-full_shrunk.png}} & \raisebox{-0.5\height}{\includegraphics[width=0.32\columnwidth]{block-1801-full_shrunk.png}} \\
\raisebox{-0.5\height}{\includegraphics[width=0.32\columnwidth]{tract-1801-net_shrunk.png}} & \raisebox{-0.5\height}{\includegraphics[width=0.32\columnwidth]{blockgroup-1801-net_shrunk.png}} & \raisebox{-0.5\height}{\includegraphics[width=0.32\columnwidth]{block-1801-net_shrunk.png}} \\
\end{tabular}
\imtitle{Indiana 1}
\end{minipage}

\begin{minipage}{\columnwidth}
\begin{tabular}{lccc}
\raisebox{-0.5\height}{\includegraphics[width=0.32\columnwidth]{tract-1802-full_shrunk.png}} & \raisebox{-0.5\height}{\includegraphics[width=0.32\columnwidth]{blockgroup-1802-full_shrunk.png}} & Missing \\
\raisebox{-0.5\height}{\includegraphics[width=0.32\columnwidth]{tract-1802-net_shrunk.png}} & \raisebox{-0.5\height}{\includegraphics[width=0.32\columnwidth]{blockgroup-1802-net_shrunk.png}} & \raisebox{-0.5\height}{\includegraphics[width=0.32\columnwidth]{block-1802-net_shrunk.png}} \\
\end{tabular}
\imtitle{Indiana 2}
\end{minipage}

\begin{minipage}{\columnwidth}
\begin{tabular}{lccc}
\raisebox{-0.5\height}{\includegraphics[width=0.32\columnwidth]{tract-1803-full_shrunk.png}} & \raisebox{-0.5\height}{\includegraphics[width=0.32\columnwidth]{blockgroup-1803-full_shrunk.png}} & \raisebox{-0.5\height}{\includegraphics[width=0.32\columnwidth]{block-1803-full_shrunk.png}} \\
\raisebox{-0.5\height}{\includegraphics[width=0.32\columnwidth]{tract-1803-net_shrunk.png}} & \raisebox{-0.5\height}{\includegraphics[width=0.32\columnwidth]{blockgroup-1803-net_shrunk.png}} & \raisebox{-0.5\height}{\includegraphics[width=0.32\columnwidth]{block-1803-net_shrunk.png}} \\
\end{tabular}
\imtitle{Indiana 3}
\end{minipage}

\begin{minipage}{\columnwidth}
\begin{tabular}{lccc}
\raisebox{-0.5\height}{\includegraphics[width=0.32\columnwidth]{tract-1804-full_shrunk.png}} & \raisebox{-0.5\height}{\includegraphics[width=0.32\columnwidth]{blockgroup-1804-full_shrunk.png}} & Missing \\
\raisebox{-0.5\height}{\includegraphics[width=0.32\columnwidth]{tract-1804-net_shrunk.png}} & \raisebox{-0.5\height}{\includegraphics[width=0.32\columnwidth]{blockgroup-1804-net_shrunk.png}} & \raisebox{-0.5\height}{\includegraphics[width=0.32\columnwidth]{block-1804-net_shrunk.png}} \\
\end{tabular}
\imtitle{Indiana 4}
\end{minipage}

\begin{minipage}{\columnwidth}
\begin{tabular}{lccc}
\raisebox{-0.5\height}{\includegraphics[width=0.32\columnwidth]{tract-1805-full_shrunk.png}} & \raisebox{-0.5\height}{\includegraphics[width=0.32\columnwidth]{blockgroup-1805-full_shrunk.png}} & Missing \\
\raisebox{-0.5\height}{\includegraphics[width=0.32\columnwidth]{tract-1805-net_shrunk.png}} & \raisebox{-0.5\height}{\includegraphics[width=0.32\columnwidth]{blockgroup-1805-net_shrunk.png}} & \raisebox{-0.5\height}{\includegraphics[width=0.32\columnwidth]{block-1805-net_shrunk.png}} \\
\end{tabular}
\imtitle{Indiana 5}
\end{minipage}

\begin{minipage}{\columnwidth}
\begin{tabular}{lccc}
\raisebox{-0.5\height}{\includegraphics[width=0.32\columnwidth]{tract-1806-full_shrunk.png}} & \raisebox{-0.5\height}{\includegraphics[width=0.32\columnwidth]{blockgroup-1806-full_shrunk.png}} & Missing \\
\raisebox{-0.5\height}{\includegraphics[width=0.32\columnwidth]{tract-1806-net_shrunk.png}} & \raisebox{-0.5\height}{\includegraphics[width=0.32\columnwidth]{blockgroup-1806-net_shrunk.png}} & Missing \\
\end{tabular}
\imtitle{Indiana 6}
\end{minipage}

\begin{minipage}{\columnwidth}
\begin{tabular}{lccc}
\raisebox{-0.5\height}{\includegraphics[width=0.32\columnwidth]{tract-1807-full_shrunk.png}} & \raisebox{-0.5\height}{\includegraphics[width=0.32\columnwidth]{blockgroup-1807-full_shrunk.png}} & Missing \\
\raisebox{-0.5\height}{\includegraphics[width=0.32\columnwidth]{tract-1807-net_shrunk.png}} & \raisebox{-0.5\height}{\includegraphics[width=0.32\columnwidth]{blockgroup-1807-net_shrunk.png}} & Missing \\
\end{tabular}
\imtitle{Indiana 7}
\end{minipage}

\begin{minipage}{\columnwidth}
\begin{tabular}{lccc}
\raisebox{-0.5\height}{\includegraphics[width=0.32\columnwidth]{tract-1808-full_shrunk.png}} & \raisebox{-0.5\height}{\includegraphics[width=0.32\columnwidth]{blockgroup-1808-full_shrunk.png}} & Missing \\
\raisebox{-0.5\height}{\includegraphics[width=0.32\columnwidth]{tract-1808-net_shrunk.png}} & \raisebox{-0.5\height}{\includegraphics[width=0.32\columnwidth]{blockgroup-1808-net_shrunk.png}} & \raisebox{-0.5\height}{\includegraphics[width=0.32\columnwidth]{block-1808-net_shrunk.png}} \\
\end{tabular}
\imtitle{Indiana 8}
\end{minipage}

\begin{minipage}{\columnwidth}
\begin{tabular}{lccc}
\raisebox{-0.5\height}{\includegraphics[width=0.32\columnwidth]{tract-1809-full_shrunk.png}} & \raisebox{-0.5\height}{\includegraphics[width=0.32\columnwidth]{blockgroup-1809-full_shrunk.png}} & Missing \\
\raisebox{-0.5\height}{\includegraphics[width=0.32\columnwidth]{tract-1809-net_shrunk.png}} & \raisebox{-0.5\height}{\includegraphics[width=0.32\columnwidth]{blockgroup-1809-net_shrunk.png}} & \raisebox{-0.5\height}{\includegraphics[width=0.32\columnwidth]{block-1809-net_shrunk.png}} \\
\end{tabular}
\imtitle{Indiana 9}
\end{minipage}
\bchap{Iowa}
\begin{minipage}{\columnwidth}
\begin{tabular}{lccc}
\raisebox{-0.5\height}{\includegraphics[width=0.32\columnwidth]{tract-1901-full_shrunk.png}} & \raisebox{-0.5\height}{\includegraphics[width=0.32\columnwidth]{blockgroup-1901-full_shrunk.png}} & Missing \\
\raisebox{-0.5\height}{\includegraphics[width=0.32\columnwidth]{tract-1901-net_shrunk.png}} & \raisebox{-0.5\height}{\includegraphics[width=0.32\columnwidth]{blockgroup-1901-net_shrunk.png}} & Missing \\
\end{tabular}
\imtitle{Iowa 1}
\end{minipage}

\begin{minipage}{\columnwidth}
\begin{tabular}{lccc}
\raisebox{-0.5\height}{\includegraphics[width=0.32\columnwidth]{tract-1902-full_shrunk.png}} & \raisebox{-0.5\height}{\includegraphics[width=0.32\columnwidth]{blockgroup-1902-full_shrunk.png}} & Missing \\
\raisebox{-0.5\height}{\includegraphics[width=0.32\columnwidth]{tract-1902-net_shrunk.png}} & \raisebox{-0.5\height}{\includegraphics[width=0.32\columnwidth]{blockgroup-1902-net_shrunk.png}} & Missing \\
\end{tabular}
\imtitle{Iowa 2}
\end{minipage}

\begin{minipage}{\columnwidth}
\begin{tabular}{lccc}
\raisebox{-0.5\height}{\includegraphics[width=0.32\columnwidth]{tract-1903-full_shrunk.png}} & \raisebox{-0.5\height}{\includegraphics[width=0.32\columnwidth]{blockgroup-1903-full_shrunk.png}} & \raisebox{-0.5\height}{\includegraphics[width=0.32\columnwidth]{block-1903-full_shrunk.png}} \\
\raisebox{-0.5\height}{\includegraphics[width=0.32\columnwidth]{tract-1903-net_shrunk.png}} & \raisebox{-0.5\height}{\includegraphics[width=0.32\columnwidth]{blockgroup-1903-net_shrunk.png}} & \raisebox{-0.5\height}{\includegraphics[width=0.32\columnwidth]{block-1903-net_shrunk.png}} \\
\end{tabular}
\imtitle{Iowa 3}
\end{minipage}

\begin{minipage}{\columnwidth}
\begin{tabular}{lccc}
\raisebox{-0.5\height}{\includegraphics[width=0.32\columnwidth]{tract-1904-full_shrunk.png}} & \raisebox{-0.5\height}{\includegraphics[width=0.32\columnwidth]{blockgroup-1904-full_shrunk.png}} & Missing \\
\raisebox{-0.5\height}{\includegraphics[width=0.32\columnwidth]{tract-1904-net_shrunk.png}} & \raisebox{-0.5\height}{\includegraphics[width=0.32\columnwidth]{blockgroup-1904-net_shrunk.png}} & \raisebox{-0.5\height}{\includegraphics[width=0.32\columnwidth]{block-1904-net_shrunk.png}} \\
\end{tabular}
\imtitle{Iowa 4}
\end{minipage}
\bchap{Kansas}
\begin{minipage}{\columnwidth}
\begin{tabular}{lccc}
\raisebox{-0.5\height}{\includegraphics[width=0.32\columnwidth]{tract-2001-full_shrunk.png}} & \raisebox{-0.5\height}{\includegraphics[width=0.32\columnwidth]{blockgroup-2001-full_shrunk.png}} & Missing \\
\raisebox{-0.5\height}{\includegraphics[width=0.32\columnwidth]{tract-2001-net_shrunk.png}} & \raisebox{-0.5\height}{\includegraphics[width=0.32\columnwidth]{blockgroup-2001-net_shrunk.png}} & Missing \\
\end{tabular}
\imtitle{Kansas 1}
\end{minipage}

\begin{minipage}{\columnwidth}
\begin{tabular}{lccc}
\raisebox{-0.5\height}{\includegraphics[width=0.32\columnwidth]{tract-2002-full_shrunk.png}} & \raisebox{-0.5\height}{\includegraphics[width=0.32\columnwidth]{blockgroup-2002-full_shrunk.png}} & Missing \\
\raisebox{-0.5\height}{\includegraphics[width=0.32\columnwidth]{tract-2002-net_shrunk.png}} & \raisebox{-0.5\height}{\includegraphics[width=0.32\columnwidth]{blockgroup-2002-net_shrunk.png}} & Missing \\
\end{tabular}
\imtitle{Kansas 2}
\end{minipage}

\begin{minipage}{\columnwidth}
\begin{tabular}{lccc}
\raisebox{-0.5\height}{\includegraphics[width=0.32\columnwidth]{tract-2003-full_shrunk.png}} & \raisebox{-0.5\height}{\includegraphics[width=0.32\columnwidth]{blockgroup-2003-full_shrunk.png}} & Missing \\
\raisebox{-0.5\height}{\includegraphics[width=0.32\columnwidth]{tract-2003-net_shrunk.png}} & \raisebox{-0.5\height}{\includegraphics[width=0.32\columnwidth]{blockgroup-2003-net_shrunk.png}} & Missing \\
\end{tabular}
\imtitle{Kansas 3}
\end{minipage}

\begin{minipage}{\columnwidth}
\begin{tabular}{lccc}
\raisebox{-0.5\height}{\includegraphics[width=0.32\columnwidth]{tract-2004-full_shrunk.png}} & \raisebox{-0.5\height}{\includegraphics[width=0.32\columnwidth]{blockgroup-2004-full_shrunk.png}} & Missing \\
\raisebox{-0.5\height}{\includegraphics[width=0.32\columnwidth]{tract-2004-net_shrunk.png}} & \raisebox{-0.5\height}{\includegraphics[width=0.32\columnwidth]{blockgroup-2004-net_shrunk.png}} & Missing \\
\end{tabular}
\imtitle{Kansas 4}
\end{minipage}
\bchap{Kentucky}
\begin{minipage}{\columnwidth}
\begin{tabular}{lccc}
\raisebox{-0.5\height}{\includegraphics[width=0.32\columnwidth]{tract-2101-full_shrunk.png}} & \raisebox{-0.5\height}{\includegraphics[width=0.32\columnwidth]{blockgroup-2101-full_shrunk.png}} & Missing \\
\raisebox{-0.5\height}{\includegraphics[width=0.32\columnwidth]{tract-2101-net_shrunk.png}} & \raisebox{-0.5\height}{\includegraphics[width=0.32\columnwidth]{blockgroup-2101-net_shrunk.png}} & Missing \\
\end{tabular}
\imtitle{Kentucky 1}
\end{minipage}

\begin{minipage}{\columnwidth}
\begin{tabular}{lccc}
\raisebox{-0.5\height}{\includegraphics[width=0.32\columnwidth]{tract-2102-full_shrunk.png}} & \raisebox{-0.5\height}{\includegraphics[width=0.32\columnwidth]{blockgroup-2102-full_shrunk.png}} & \raisebox{-0.5\height}{\includegraphics[width=0.32\columnwidth]{block-2102-full_shrunk.png}} \\
\raisebox{-0.5\height}{\includegraphics[width=0.32\columnwidth]{tract-2102-net_shrunk.png}} & \raisebox{-0.5\height}{\includegraphics[width=0.32\columnwidth]{blockgroup-2102-net_shrunk.png}} & \raisebox{-0.5\height}{\includegraphics[width=0.32\columnwidth]{block-2102-net_shrunk.png}} \\
\end{tabular}
\imtitle{Kentucky 2}
\end{minipage}

\begin{minipage}{\columnwidth}
\begin{tabular}{lccc}
\raisebox{-0.5\height}{\includegraphics[width=0.32\columnwidth]{tract-2103-full_shrunk.png}} & \raisebox{-0.5\height}{\includegraphics[width=0.32\columnwidth]{blockgroup-2103-full_shrunk.png}} & \raisebox{-0.5\height}{\includegraphics[width=0.32\columnwidth]{block-2103-full_shrunk.png}} \\
\raisebox{-0.5\height}{\includegraphics[width=0.32\columnwidth]{tract-2103-net_shrunk.png}} & \raisebox{-0.5\height}{\includegraphics[width=0.32\columnwidth]{blockgroup-2103-net_shrunk.png}} & \raisebox{-0.5\height}{\includegraphics[width=0.32\columnwidth]{block-2103-net_shrunk.png}} \\
\end{tabular}
\imtitle{Kentucky 3}
\end{minipage}

\begin{minipage}{\columnwidth}
\begin{tabular}{lccc}
\raisebox{-0.5\height}{\includegraphics[width=0.32\columnwidth]{tract-2104-full_shrunk.png}} & \raisebox{-0.5\height}{\includegraphics[width=0.32\columnwidth]{blockgroup-2104-full_shrunk.png}} & Missing \\
\raisebox{-0.5\height}{\includegraphics[width=0.32\columnwidth]{tract-2104-net_shrunk.png}} & \raisebox{-0.5\height}{\includegraphics[width=0.32\columnwidth]{blockgroup-2104-net_shrunk.png}} & Missing \\
\end{tabular}
\imtitle{Kentucky 4}
\end{minipage}

\begin{minipage}{\columnwidth}
\begin{tabular}{lccc}
\raisebox{-0.5\height}{\includegraphics[width=0.32\columnwidth]{tract-2105-full_shrunk.png}} & \raisebox{-0.5\height}{\includegraphics[width=0.32\columnwidth]{blockgroup-2105-full_shrunk.png}} & \raisebox{-0.5\height}{\includegraphics[width=0.32\columnwidth]{block-2105-full_shrunk.png}} \\
\raisebox{-0.5\height}{\includegraphics[width=0.32\columnwidth]{tract-2105-net_shrunk.png}} & \raisebox{-0.5\height}{\includegraphics[width=0.32\columnwidth]{blockgroup-2105-net_shrunk.png}} & Missing \\
\end{tabular}
\imtitle{Kentucky 5}
\end{minipage}

\begin{minipage}{\columnwidth}
\begin{tabular}{lccc}
\raisebox{-0.5\height}{\includegraphics[width=0.32\columnwidth]{tract-2106-full_shrunk.png}} & \raisebox{-0.5\height}{\includegraphics[width=0.32\columnwidth]{blockgroup-2106-full_shrunk.png}} & Missing \\
\raisebox{-0.5\height}{\includegraphics[width=0.32\columnwidth]{tract-2106-net_shrunk.png}} & \raisebox{-0.5\height}{\includegraphics[width=0.32\columnwidth]{blockgroup-2106-net_shrunk.png}} & Missing \\
\end{tabular}
\imtitle{Kentucky 6}
\end{minipage}
\bchap{Louisiana}
\begin{minipage}{\columnwidth}
\begin{tabular}{lccc}
\raisebox{-0.5\height}{\includegraphics[width=0.32\columnwidth]{tract-2201-full_shrunk.png}} & \raisebox{-0.5\height}{\includegraphics[width=0.32\columnwidth]{blockgroup-2201-full_shrunk.png}} & Missing \\
\raisebox{-0.5\height}{\includegraphics[width=0.32\columnwidth]{tract-2201-net_shrunk.png}} & \raisebox{-0.5\height}{\includegraphics[width=0.32\columnwidth]{blockgroup-2201-net_shrunk.png}} & Missing \\
\end{tabular}
\imtitle{Louisiana 1}
\end{minipage}

\begin{minipage}{\columnwidth}
\begin{tabular}{lccc}
\raisebox{-0.5\height}{\includegraphics[width=0.32\columnwidth]{tract-2202-full_shrunk.png}} & \raisebox{-0.5\height}{\includegraphics[width=0.32\columnwidth]{blockgroup-2202-full_shrunk.png}} & Missing \\
\raisebox{-0.5\height}{\includegraphics[width=0.32\columnwidth]{tract-2202-net_shrunk.png}} & \raisebox{-0.5\height}{\includegraphics[width=0.32\columnwidth]{blockgroup-2202-net_shrunk.png}} & Missing \\
\end{tabular}
\imtitle{Louisiana 2}
\end{minipage}

\begin{minipage}{\columnwidth}
\begin{tabular}{lccc}
\raisebox{-0.5\height}{\includegraphics[width=0.32\columnwidth]{tract-2203-full_shrunk.png}} & \raisebox{-0.5\height}{\includegraphics[width=0.32\columnwidth]{blockgroup-2203-full_shrunk.png}} & \raisebox{-0.5\height}{\includegraphics[width=0.32\columnwidth]{block-2203-full_shrunk.png}} \\
\raisebox{-0.5\height}{\includegraphics[width=0.32\columnwidth]{tract-2203-net_shrunk.png}} & \raisebox{-0.5\height}{\includegraphics[width=0.32\columnwidth]{blockgroup-2203-net_shrunk.png}} & \raisebox{-0.5\height}{\includegraphics[width=0.32\columnwidth]{block-2203-net_shrunk.png}} \\
\end{tabular}
\imtitle{Louisiana 3}
\end{minipage}

\begin{minipage}{\columnwidth}
\begin{tabular}{lccc}
\raisebox{-0.5\height}{\includegraphics[width=0.32\columnwidth]{tract-2204-full_shrunk.png}} & \raisebox{-0.5\height}{\includegraphics[width=0.32\columnwidth]{blockgroup-2204-full_shrunk.png}} & Missing \\
\raisebox{-0.5\height}{\includegraphics[width=0.32\columnwidth]{tract-2204-net_shrunk.png}} & \raisebox{-0.5\height}{\includegraphics[width=0.32\columnwidth]{blockgroup-2204-net_shrunk.png}} & Missing \\
\end{tabular}
\imtitle{Louisiana 4}
\end{minipage}

\begin{minipage}{\columnwidth}
\begin{tabular}{lccc}
\raisebox{-0.5\height}{\includegraphics[width=0.32\columnwidth]{tract-2205-full_shrunk.png}} & \raisebox{-0.5\height}{\includegraphics[width=0.32\columnwidth]{blockgroup-2205-full_shrunk.png}} & Missing \\
\raisebox{-0.5\height}{\includegraphics[width=0.32\columnwidth]{tract-2205-net_shrunk.png}} & \raisebox{-0.5\height}{\includegraphics[width=0.32\columnwidth]{blockgroup-2205-net_shrunk.png}} & Missing \\
\end{tabular}
\imtitle{Louisiana 5}
\end{minipage}

\begin{minipage}{\columnwidth}
\begin{tabular}{lccc}
\raisebox{-0.5\height}{\includegraphics[width=0.32\columnwidth]{tract-2206-full_shrunk.png}} & \raisebox{-0.5\height}{\includegraphics[width=0.32\columnwidth]{blockgroup-2206-full_shrunk.png}} & \raisebox{-0.5\height}{\includegraphics[width=0.32\columnwidth]{block-2206-full_shrunk.png}} \\
\raisebox{-0.5\height}{\includegraphics[width=0.32\columnwidth]{tract-2206-net_shrunk.png}} & \raisebox{-0.5\height}{\includegraphics[width=0.32\columnwidth]{blockgroup-2206-net_shrunk.png}} & \raisebox{-0.5\height}{\includegraphics[width=0.32\columnwidth]{block-2206-net_shrunk.png}} \\
\end{tabular}
\imtitle{Louisiana 6}
\end{minipage}
\bchap{Maine}
\begin{minipage}{\columnwidth}
\begin{tabular}{lccc}
\raisebox{-0.5\height}{\includegraphics[width=0.32\columnwidth]{tract-2301-full_shrunk.png}} & \raisebox{-0.5\height}{\includegraphics[width=0.32\columnwidth]{blockgroup-2301-full_shrunk.png}} & \raisebox{-0.5\height}{\includegraphics[width=0.32\columnwidth]{block-2301-full_shrunk.png}} \\
\raisebox{-0.5\height}{\includegraphics[width=0.32\columnwidth]{tract-2301-net_shrunk.png}} & \raisebox{-0.5\height}{\includegraphics[width=0.32\columnwidth]{blockgroup-2301-net_shrunk.png}} & Missing \\
\end{tabular}
\imtitle{Maine 1}
\end{minipage}

\begin{minipage}{\columnwidth}
\begin{tabular}{lccc}
\raisebox{-0.5\height}{\includegraphics[width=0.32\columnwidth]{tract-2302-full_shrunk.png}} & \raisebox{-0.5\height}{\includegraphics[width=0.32\columnwidth]{blockgroup-2302-full_shrunk.png}} & \raisebox{-0.5\height}{\includegraphics[width=0.32\columnwidth]{block-2302-full_shrunk.png}} \\
\raisebox{-0.5\height}{\includegraphics[width=0.32\columnwidth]{tract-2302-net_shrunk.png}} & \raisebox{-0.5\height}{\includegraphics[width=0.32\columnwidth]{blockgroup-2302-net_shrunk.png}} & \raisebox{-0.5\height}{\includegraphics[width=0.32\columnwidth]{block-2302-net_shrunk.png}} \\
\end{tabular}
\imtitle{Maine 2}
\end{minipage}
\bchap{Maryland}
\begin{minipage}{\columnwidth}
\begin{tabular}{lccc}
Missing & Missing & \raisebox{-0.5\height}{\includegraphics[width=0.32\columnwidth]{block-2401-full_shrunk.png}} \\
Missing & Missing & \raisebox{-0.5\height}{\includegraphics[width=0.32\columnwidth]{block-2401-net_shrunk.png}} \\
\end{tabular}
\imtitle{Maryland 1}
\end{minipage}

\begin{minipage}{\columnwidth}
\begin{tabular}{lccc}
Missing & Missing & Missing \\
Missing & Missing & Missing \\
\end{tabular}
\imtitle{Maryland 2}
\end{minipage}

\begin{minipage}{\columnwidth}
\begin{tabular}{lccc}
Missing & Missing & Missing \\
Missing & Missing & Missing \\
\end{tabular}
\imtitle{Maryland 3}
\end{minipage}

\begin{minipage}{\columnwidth}
\begin{tabular}{lccc}
Missing & Missing & Missing \\
Missing & Missing & Missing \\
\end{tabular}
\imtitle{Maryland 4}
\end{minipage}

\begin{minipage}{\columnwidth}
\begin{tabular}{lccc}
Missing & Missing & Missing \\
Missing & Missing & Missing \\
\end{tabular}
\imtitle{Maryland 5}
\end{minipage}

\begin{minipage}{\columnwidth}
\begin{tabular}{lccc}
Missing & Missing & \raisebox{-0.5\height}{\includegraphics[width=0.32\columnwidth]{block-2406-full_shrunk.png}} \\
Missing & Missing & \raisebox{-0.5\height}{\includegraphics[width=0.32\columnwidth]{block-2406-net_shrunk.png}} \\
\end{tabular}
\imtitle{Maryland 6}
\end{minipage}

\begin{minipage}{\columnwidth}
\begin{tabular}{lccc}
Missing & Missing & Missing \\
Missing & Missing & Missing \\
\end{tabular}
\imtitle{Maryland 7}
\end{minipage}

\begin{minipage}{\columnwidth}
\begin{tabular}{lccc}
Missing & Missing & Missing \\
Missing & Missing & Missing \\
\end{tabular}
\imtitle{Maryland 8}
\end{minipage}
\bchap{Massachusetts}
\begin{minipage}{\columnwidth}
\begin{tabular}{lccc}
\raisebox{-0.5\height}{\includegraphics[width=0.32\columnwidth]{tract-2501-full_shrunk.png}} & \raisebox{-0.5\height}{\includegraphics[width=0.32\columnwidth]{blockgroup-2501-full_shrunk.png}} & \raisebox{-0.5\height}{\includegraphics[width=0.32\columnwidth]{block-2501-full_shrunk.png}} \\
\raisebox{-0.5\height}{\includegraphics[width=0.32\columnwidth]{tract-2501-net_shrunk.png}} & \raisebox{-0.5\height}{\includegraphics[width=0.32\columnwidth]{blockgroup-2501-net_shrunk.png}} & \raisebox{-0.5\height}{\includegraphics[width=0.32\columnwidth]{block-2501-net_shrunk.png}} \\
\end{tabular}
\imtitle{Massachusetts 1}
\end{minipage}

\begin{minipage}{\columnwidth}
\begin{tabular}{lccc}
\raisebox{-0.5\height}{\includegraphics[width=0.32\columnwidth]{tract-2502-full_shrunk.png}} & \raisebox{-0.5\height}{\includegraphics[width=0.32\columnwidth]{blockgroup-2502-full_shrunk.png}} & \raisebox{-0.5\height}{\includegraphics[width=0.32\columnwidth]{block-2502-full_shrunk.png}} \\
\raisebox{-0.5\height}{\includegraphics[width=0.32\columnwidth]{tract-2502-net_shrunk.png}} & \raisebox{-0.5\height}{\includegraphics[width=0.32\columnwidth]{blockgroup-2502-net_shrunk.png}} & \raisebox{-0.5\height}{\includegraphics[width=0.32\columnwidth]{block-2502-net_shrunk.png}} \\
\end{tabular}
\imtitle{Massachusetts 2}
\end{minipage}

\begin{minipage}{\columnwidth}
\begin{tabular}{lccc}
\raisebox{-0.5\height}{\includegraphics[width=0.32\columnwidth]{tract-2503-full_shrunk.png}} & \raisebox{-0.5\height}{\includegraphics[width=0.32\columnwidth]{blockgroup-2503-full_shrunk.png}} & Missing \\
\raisebox{-0.5\height}{\includegraphics[width=0.32\columnwidth]{tract-2503-net_shrunk.png}} & \raisebox{-0.5\height}{\includegraphics[width=0.32\columnwidth]{blockgroup-2503-net_shrunk.png}} & Missing \\
\end{tabular}
\imtitle{Massachusetts 3}
\end{minipage}

\begin{minipage}{\columnwidth}
\begin{tabular}{lccc}
\raisebox{-0.5\height}{\includegraphics[width=0.32\columnwidth]{tract-2504-full_shrunk.png}} & \raisebox{-0.5\height}{\includegraphics[width=0.32\columnwidth]{blockgroup-2504-full_shrunk.png}} & \raisebox{-0.5\height}{\includegraphics[width=0.32\columnwidth]{block-2504-full_shrunk.png}} \\
\raisebox{-0.5\height}{\includegraphics[width=0.32\columnwidth]{tract-2504-net_shrunk.png}} & \raisebox{-0.5\height}{\includegraphics[width=0.32\columnwidth]{blockgroup-2504-net_shrunk.png}} & \raisebox{-0.5\height}{\includegraphics[width=0.32\columnwidth]{block-2504-net_shrunk.png}} \\
\end{tabular}
\imtitle{Massachusetts 4}
\end{minipage}

\begin{minipage}{\columnwidth}
\begin{tabular}{lccc}
\raisebox{-0.5\height}{\includegraphics[width=0.32\columnwidth]{tract-2505-full_shrunk.png}} & \raisebox{-0.5\height}{\includegraphics[width=0.32\columnwidth]{blockgroup-2505-full_shrunk.png}} & Missing \\
\raisebox{-0.5\height}{\includegraphics[width=0.32\columnwidth]{tract-2505-net_shrunk.png}} & \raisebox{-0.5\height}{\includegraphics[width=0.32\columnwidth]{blockgroup-2505-net_shrunk.png}} & Missing \\
\end{tabular}
\imtitle{Massachusetts 5}
\end{minipage}

\begin{minipage}{\columnwidth}
\begin{tabular}{lccc}
\raisebox{-0.5\height}{\includegraphics[width=0.32\columnwidth]{tract-2506-full_shrunk.png}} & \raisebox{-0.5\height}{\includegraphics[width=0.32\columnwidth]{blockgroup-2506-full_shrunk.png}} & Missing \\
\raisebox{-0.5\height}{\includegraphics[width=0.32\columnwidth]{tract-2506-net_shrunk.png}} & \raisebox{-0.5\height}{\includegraphics[width=0.32\columnwidth]{blockgroup-2506-net_shrunk.png}} & Missing \\
\end{tabular}
\imtitle{Massachusetts 6}
\end{minipage}

\begin{minipage}{\columnwidth}
\begin{tabular}{lccc}
\raisebox{-0.5\height}{\includegraphics[width=0.32\columnwidth]{tract-2507-full_shrunk.png}} & \raisebox{-0.5\height}{\includegraphics[width=0.32\columnwidth]{blockgroup-2507-full_shrunk.png}} & Missing \\
\raisebox{-0.5\height}{\includegraphics[width=0.32\columnwidth]{tract-2507-net_shrunk.png}} & \raisebox{-0.5\height}{\includegraphics[width=0.32\columnwidth]{blockgroup-2507-net_shrunk.png}} & Missing \\
\end{tabular}
\imtitle{Massachusetts 7}
\end{minipage}

\begin{minipage}{\columnwidth}
\begin{tabular}{lccc}
\raisebox{-0.5\height}{\includegraphics[width=0.32\columnwidth]{tract-2508-full_shrunk.png}} & \raisebox{-0.5\height}{\includegraphics[width=0.32\columnwidth]{blockgroup-2508-full_shrunk.png}} & Missing \\
\raisebox{-0.5\height}{\includegraphics[width=0.32\columnwidth]{tract-2508-net_shrunk.png}} & \raisebox{-0.5\height}{\includegraphics[width=0.32\columnwidth]{blockgroup-2508-net_shrunk.png}} & Missing \\
\end{tabular}
\imtitle{Massachusetts 8}
\end{minipage}

\begin{minipage}{\columnwidth}
\begin{tabular}{lccc}
\raisebox{-0.5\height}{\includegraphics[width=0.32\columnwidth]{tract-2509-full_shrunk.png}} & \raisebox{-0.5\height}{\includegraphics[width=0.32\columnwidth]{blockgroup-2509-full_shrunk.png}} & \raisebox{-0.5\height}{\includegraphics[width=0.32\columnwidth]{block-2509-full_shrunk.png}} \\
\raisebox{-0.5\height}{\includegraphics[width=0.32\columnwidth]{tract-2509-net_shrunk.png}} & \raisebox{-0.5\height}{\includegraphics[width=0.32\columnwidth]{blockgroup-2509-net_shrunk.png}} & \raisebox{-0.5\height}{\includegraphics[width=0.32\columnwidth]{block-2509-net_shrunk.png}} \\
\end{tabular}
\imtitle{Massachusetts 9}
\end{minipage}
\bchap{Michigan}
\begin{minipage}{\columnwidth}
\begin{tabular}{lccc}
Missing & Missing & Missing \\
Missing & Missing & Missing \\
\end{tabular}
\imtitle{Michigan 1}
\end{minipage}

\begin{minipage}{\columnwidth}
\begin{tabular}{lccc}
Missing & Missing & Missing \\
Missing & Missing & Missing \\
\end{tabular}
\imtitle{Michigan 2}
\end{minipage}

\begin{minipage}{\columnwidth}
\begin{tabular}{lccc}
Missing & Missing & Missing \\
Missing & Missing & Missing \\
\end{tabular}
\imtitle{Michigan 3}
\end{minipage}

\begin{minipage}{\columnwidth}
\begin{tabular}{lccc}
Missing & Missing & Missing \\
Missing & Missing & Missing \\
\end{tabular}
\imtitle{Michigan 4}
\end{minipage}

\begin{minipage}{\columnwidth}
\begin{tabular}{lccc}
Missing & Missing & Missing \\
Missing & Missing & Missing \\
\end{tabular}
\imtitle{Michigan 5}
\end{minipage}

\begin{minipage}{\columnwidth}
\begin{tabular}{lccc}
Missing & Missing & Missing \\
Missing & Missing & Missing \\
\end{tabular}
\imtitle{Michigan 6}
\end{minipage}

\begin{minipage}{\columnwidth}
\begin{tabular}{lccc}
Missing & Missing & Missing \\
Missing & Missing & Missing \\
\end{tabular}
\imtitle{Michigan 7}
\end{minipage}

\begin{minipage}{\columnwidth}
\begin{tabular}{lccc}
Missing & Missing & Missing \\
Missing & Missing & Missing \\
\end{tabular}
\imtitle{Michigan 8}
\end{minipage}

\begin{minipage}{\columnwidth}
\begin{tabular}{lccc}
Missing & Missing & Missing \\
Missing & Missing & Missing \\
\end{tabular}
\imtitle{Michigan 9}
\end{minipage}

\begin{minipage}{\columnwidth}
\begin{tabular}{lccc}
Missing & Missing & Missing \\
Missing & Missing & Missing \\
\end{tabular}
\imtitle{Michigan 10}
\end{minipage}

\begin{minipage}{\columnwidth}
\begin{tabular}{lccc}
Missing & Missing & Missing \\
Missing & Missing & Missing \\
\end{tabular}
\imtitle{Michigan 11}
\end{minipage}

\begin{minipage}{\columnwidth}
\begin{tabular}{lccc}
Missing & Missing & Missing \\
Missing & Missing & Missing \\
\end{tabular}
\imtitle{Michigan 12}
\end{minipage}

\begin{minipage}{\columnwidth}
\begin{tabular}{lccc}
Missing & Missing & Missing \\
Missing & Missing & Missing \\
\end{tabular}
\imtitle{Michigan 13}
\end{minipage}

\begin{minipage}{\columnwidth}
\begin{tabular}{lccc}
Missing & Missing & Missing \\
Missing & Missing & Missing \\
\end{tabular}
\imtitle{Michigan 14}
\end{minipage}
\bchap{Minnesota}
\begin{minipage}{\columnwidth}
\begin{tabular}{lccc}
\raisebox{-0.5\height}{\includegraphics[width=0.32\columnwidth]{tract-2701-full_shrunk.png}} & \raisebox{-0.5\height}{\includegraphics[width=0.32\columnwidth]{blockgroup-2701-full_shrunk.png}} & Missing \\
\raisebox{-0.5\height}{\includegraphics[width=0.32\columnwidth]{tract-2701-net_shrunk.png}} & \raisebox{-0.5\height}{\includegraphics[width=0.32\columnwidth]{blockgroup-2701-net_shrunk.png}} & Missing \\
\end{tabular}
\imtitle{Minnesota 1}
\end{minipage}

\begin{minipage}{\columnwidth}
\begin{tabular}{lccc}
\raisebox{-0.5\height}{\includegraphics[width=0.32\columnwidth]{tract-2702-full_shrunk.png}} & \raisebox{-0.5\height}{\includegraphics[width=0.32\columnwidth]{blockgroup-2702-full_shrunk.png}} & Missing \\
\raisebox{-0.5\height}{\includegraphics[width=0.32\columnwidth]{tract-2702-net_shrunk.png}} & \raisebox{-0.5\height}{\includegraphics[width=0.32\columnwidth]{blockgroup-2702-net_shrunk.png}} & Missing \\
\end{tabular}
\imtitle{Minnesota 2}
\end{minipage}

\begin{minipage}{\columnwidth}
\begin{tabular}{lccc}
\raisebox{-0.5\height}{\includegraphics[width=0.32\columnwidth]{tract-2703-full_shrunk.png}} & \raisebox{-0.5\height}{\includegraphics[width=0.32\columnwidth]{blockgroup-2703-full_shrunk.png}} & Missing \\
\raisebox{-0.5\height}{\includegraphics[width=0.32\columnwidth]{tract-2703-net_shrunk.png}} & \raisebox{-0.5\height}{\includegraphics[width=0.32\columnwidth]{blockgroup-2703-net_shrunk.png}} & Missing \\
\end{tabular}
\imtitle{Minnesota 3}
\end{minipage}

\begin{minipage}{\columnwidth}
\begin{tabular}{lccc}
\raisebox{-0.5\height}{\includegraphics[width=0.32\columnwidth]{tract-2704-full_shrunk.png}} & \raisebox{-0.5\height}{\includegraphics[width=0.32\columnwidth]{blockgroup-2704-full_shrunk.png}} & Missing \\
\raisebox{-0.5\height}{\includegraphics[width=0.32\columnwidth]{tract-2704-net_shrunk.png}} & \raisebox{-0.5\height}{\includegraphics[width=0.32\columnwidth]{blockgroup-2704-net_shrunk.png}} & Missing \\
\end{tabular}
\imtitle{Minnesota 4}
\end{minipage}

\begin{minipage}{\columnwidth}
\begin{tabular}{lccc}
\raisebox{-0.5\height}{\includegraphics[width=0.32\columnwidth]{tract-2705-full_shrunk.png}} & \raisebox{-0.5\height}{\includegraphics[width=0.32\columnwidth]{blockgroup-2705-full_shrunk.png}} & Missing \\
\raisebox{-0.5\height}{\includegraphics[width=0.32\columnwidth]{tract-2705-net_shrunk.png}} & \raisebox{-0.5\height}{\includegraphics[width=0.32\columnwidth]{blockgroup-2705-net_shrunk.png}} & Missing \\
\end{tabular}
\imtitle{Minnesota 5}
\end{minipage}

\begin{minipage}{\columnwidth}
\begin{tabular}{lccc}
\raisebox{-0.5\height}{\includegraphics[width=0.32\columnwidth]{tract-2706-full_shrunk.png}} & \raisebox{-0.5\height}{\includegraphics[width=0.32\columnwidth]{blockgroup-2706-full_shrunk.png}} & Missing \\
\raisebox{-0.5\height}{\includegraphics[width=0.32\columnwidth]{tract-2706-net_shrunk.png}} & \raisebox{-0.5\height}{\includegraphics[width=0.32\columnwidth]{blockgroup-2706-net_shrunk.png}} & Missing \\
\end{tabular}
\imtitle{Minnesota 6}
\end{minipage}

\begin{minipage}{\columnwidth}
\begin{tabular}{lccc}
\raisebox{-0.5\height}{\includegraphics[width=0.32\columnwidth]{tract-2707-full_shrunk.png}} & \raisebox{-0.5\height}{\includegraphics[width=0.32\columnwidth]{blockgroup-2707-full_shrunk.png}} & Missing \\
\raisebox{-0.5\height}{\includegraphics[width=0.32\columnwidth]{tract-2707-net_shrunk.png}} & \raisebox{-0.5\height}{\includegraphics[width=0.32\columnwidth]{blockgroup-2707-net_shrunk.png}} & Missing \\
\end{tabular}
\imtitle{Minnesota 7}
\end{minipage}

\begin{minipage}{\columnwidth}
\begin{tabular}{lccc}
\raisebox{-0.5\height}{\includegraphics[width=0.32\columnwidth]{tract-2708-full_shrunk.png}} & \raisebox{-0.5\height}{\includegraphics[width=0.32\columnwidth]{blockgroup-2708-full_shrunk.png}} & \raisebox{-0.5\height}{\includegraphics[width=0.32\columnwidth]{block-2708-full_shrunk.png}} \\
\raisebox{-0.5\height}{\includegraphics[width=0.32\columnwidth]{tract-2708-net_shrunk.png}} & \raisebox{-0.5\height}{\includegraphics[width=0.32\columnwidth]{blockgroup-2708-net_shrunk.png}} & Missing \\
\end{tabular}
\imtitle{Minnesota 8}
\end{minipage}
\bchap{Mississippi}
\begin{minipage}{\columnwidth}
\begin{tabular}{lccc}
\raisebox{-0.5\height}{\includegraphics[width=0.32\columnwidth]{tract-2801-full_shrunk.png}} & \raisebox{-0.5\height}{\includegraphics[width=0.32\columnwidth]{blockgroup-2801-full_shrunk.png}} & Missing \\
\raisebox{-0.5\height}{\includegraphics[width=0.32\columnwidth]{tract-2801-net_shrunk.png}} & \raisebox{-0.5\height}{\includegraphics[width=0.32\columnwidth]{blockgroup-2801-net_shrunk.png}} & Missing \\
\end{tabular}
\imtitle{Mississippi 1}
\end{minipage}

\begin{minipage}{\columnwidth}
\begin{tabular}{lccc}
\raisebox{-0.5\height}{\includegraphics[width=0.32\columnwidth]{tract-2802-full_shrunk.png}} & \raisebox{-0.5\height}{\includegraphics[width=0.32\columnwidth]{blockgroup-2802-full_shrunk.png}} & Missing \\
\raisebox{-0.5\height}{\includegraphics[width=0.32\columnwidth]{tract-2802-net_shrunk.png}} & \raisebox{-0.5\height}{\includegraphics[width=0.32\columnwidth]{blockgroup-2802-net_shrunk.png}} & Missing \\
\end{tabular}
\imtitle{Mississippi 2}
\end{minipage}

\begin{minipage}{\columnwidth}
\begin{tabular}{lccc}
\raisebox{-0.5\height}{\includegraphics[width=0.32\columnwidth]{tract-2803-full_shrunk.png}} & \raisebox{-0.5\height}{\includegraphics[width=0.32\columnwidth]{blockgroup-2803-full_shrunk.png}} & Missing \\
\raisebox{-0.5\height}{\includegraphics[width=0.32\columnwidth]{tract-2803-net_shrunk.png}} & \raisebox{-0.5\height}{\includegraphics[width=0.32\columnwidth]{blockgroup-2803-net_shrunk.png}} & Missing \\
\end{tabular}
\imtitle{Mississippi 3}
\end{minipage}

\begin{minipage}{\columnwidth}
\begin{tabular}{lccc}
\raisebox{-0.5\height}{\includegraphics[width=0.32\columnwidth]{tract-2804-full_shrunk.png}} & \raisebox{-0.5\height}{\includegraphics[width=0.32\columnwidth]{blockgroup-2804-full_shrunk.png}} & \raisebox{-0.5\height}{\includegraphics[width=0.32\columnwidth]{block-2804-full_shrunk.png}} \\
\raisebox{-0.5\height}{\includegraphics[width=0.32\columnwidth]{tract-2804-net_shrunk.png}} & \raisebox{-0.5\height}{\includegraphics[width=0.32\columnwidth]{blockgroup-2804-net_shrunk.png}} & \raisebox{-0.5\height}{\includegraphics[width=0.32\columnwidth]{block-2804-net_shrunk.png}} \\
\end{tabular}
\imtitle{Mississippi 4}
\end{minipage}
\bchap{Missouri}
\begin{minipage}{\columnwidth}
\begin{tabular}{lccc}
\raisebox{-0.5\height}{\includegraphics[width=0.32\columnwidth]{tract-2901-full_shrunk.png}} & \raisebox{-0.5\height}{\includegraphics[width=0.32\columnwidth]{blockgroup-2901-full_shrunk.png}} & Missing \\
\raisebox{-0.5\height}{\includegraphics[width=0.32\columnwidth]{tract-2901-net_shrunk.png}} & \raisebox{-0.5\height}{\includegraphics[width=0.32\columnwidth]{blockgroup-2901-net_shrunk.png}} & Missing \\
\end{tabular}
\imtitle{Missouri 1}
\end{minipage}

\begin{minipage}{\columnwidth}
\begin{tabular}{lccc}
\raisebox{-0.5\height}{\includegraphics[width=0.32\columnwidth]{tract-2902-full_shrunk.png}} & \raisebox{-0.5\height}{\includegraphics[width=0.32\columnwidth]{blockgroup-2902-full_shrunk.png}} & Missing \\
\raisebox{-0.5\height}{\includegraphics[width=0.32\columnwidth]{tract-2902-net_shrunk.png}} & \raisebox{-0.5\height}{\includegraphics[width=0.32\columnwidth]{blockgroup-2902-net_shrunk.png}} & Missing \\
\end{tabular}
\imtitle{Missouri 2}
\end{minipage}

\begin{minipage}{\columnwidth}
\begin{tabular}{lccc}
\raisebox{-0.5\height}{\includegraphics[width=0.32\columnwidth]{tract-2903-full_shrunk.png}} & \raisebox{-0.5\height}{\includegraphics[width=0.32\columnwidth]{blockgroup-2903-full_shrunk.png}} & Missing \\
\raisebox{-0.5\height}{\includegraphics[width=0.32\columnwidth]{tract-2903-net_shrunk.png}} & \raisebox{-0.5\height}{\includegraphics[width=0.32\columnwidth]{blockgroup-2903-net_shrunk.png}} & Missing \\
\end{tabular}
\imtitle{Missouri 3}
\end{minipage}

\begin{minipage}{\columnwidth}
\begin{tabular}{lccc}
\raisebox{-0.5\height}{\includegraphics[width=0.32\columnwidth]{tract-2904-full_shrunk.png}} & \raisebox{-0.5\height}{\includegraphics[width=0.32\columnwidth]{blockgroup-2904-full_shrunk.png}} & Missing \\
\raisebox{-0.5\height}{\includegraphics[width=0.32\columnwidth]{tract-2904-net_shrunk.png}} & \raisebox{-0.5\height}{\includegraphics[width=0.32\columnwidth]{blockgroup-2904-net_shrunk.png}} & Missing \\
\end{tabular}
\imtitle{Missouri 4}
\end{minipage}

\begin{minipage}{\columnwidth}
\begin{tabular}{lccc}
\raisebox{-0.5\height}{\includegraphics[width=0.32\columnwidth]{tract-2905-full_shrunk.png}} & \raisebox{-0.5\height}{\includegraphics[width=0.32\columnwidth]{blockgroup-2905-full_shrunk.png}} & Missing \\
\raisebox{-0.5\height}{\includegraphics[width=0.32\columnwidth]{tract-2905-net_shrunk.png}} & \raisebox{-0.5\height}{\includegraphics[width=0.32\columnwidth]{blockgroup-2905-net_shrunk.png}} & Missing \\
\end{tabular}
\imtitle{Missouri 5}
\end{minipage}

\begin{minipage}{\columnwidth}
\begin{tabular}{lccc}
\raisebox{-0.5\height}{\includegraphics[width=0.32\columnwidth]{tract-2906-full_shrunk.png}} & \raisebox{-0.5\height}{\includegraphics[width=0.32\columnwidth]{blockgroup-2906-full_shrunk.png}} & Missing \\
\raisebox{-0.5\height}{\includegraphics[width=0.32\columnwidth]{tract-2906-net_shrunk.png}} & \raisebox{-0.5\height}{\includegraphics[width=0.32\columnwidth]{blockgroup-2906-net_shrunk.png}} & Missing \\
\end{tabular}
\imtitle{Missouri 6}
\end{minipage}

\begin{minipage}{\columnwidth}
\begin{tabular}{lccc}
\raisebox{-0.5\height}{\includegraphics[width=0.32\columnwidth]{tract-2907-full_shrunk.png}} & \raisebox{-0.5\height}{\includegraphics[width=0.32\columnwidth]{blockgroup-2907-full_shrunk.png}} & Missing \\
\raisebox{-0.5\height}{\includegraphics[width=0.32\columnwidth]{tract-2907-net_shrunk.png}} & \raisebox{-0.5\height}{\includegraphics[width=0.32\columnwidth]{blockgroup-2907-net_shrunk.png}} & Missing \\
\end{tabular}
\imtitle{Missouri 7}
\end{minipage}

\begin{minipage}{\columnwidth}
\begin{tabular}{lccc}
\raisebox{-0.5\height}{\includegraphics[width=0.32\columnwidth]{tract-2908-full_shrunk.png}} & \raisebox{-0.5\height}{\includegraphics[width=0.32\columnwidth]{blockgroup-2908-full_shrunk.png}} & Missing \\
\raisebox{-0.5\height}{\includegraphics[width=0.32\columnwidth]{tract-2908-net_shrunk.png}} & \raisebox{-0.5\height}{\includegraphics[width=0.32\columnwidth]{blockgroup-2908-net_shrunk.png}} & Missing \\
\end{tabular}
\imtitle{Missouri 8}
\end{minipage}
\bchap{Montana}
\begin{minipage}{\columnwidth}
\begin{tabular}{lccc}
\raisebox{-0.5\height}{\includegraphics[width=0.32\columnwidth]{tract-3000-full_shrunk.png}} & \raisebox{-0.5\height}{\includegraphics[width=0.32\columnwidth]{blockgroup-3000-full_shrunk.png}} & Missing \\
\raisebox{-0.5\height}{\includegraphics[width=0.32\columnwidth]{tract-3000-net_shrunk.png}} & \raisebox{-0.5\height}{\includegraphics[width=0.32\columnwidth]{blockgroup-3000-net_shrunk.png}} & Missing \\
\end{tabular}
\imtitle{Montana 0}
\end{minipage}
\bchap{Nebraska}
\begin{minipage}{\columnwidth}
\begin{tabular}{lccc}
\raisebox{-0.5\height}{\includegraphics[width=0.32\columnwidth]{tract-3101-full_shrunk.png}} & \raisebox{-0.5\height}{\includegraphics[width=0.32\columnwidth]{blockgroup-3101-full_shrunk.png}} & Missing \\
\raisebox{-0.5\height}{\includegraphics[width=0.32\columnwidth]{tract-3101-net_shrunk.png}} & \raisebox{-0.5\height}{\includegraphics[width=0.32\columnwidth]{blockgroup-3101-net_shrunk.png}} & Missing \\
\end{tabular}
\imtitle{Nebraska 1}
\end{minipage}

\begin{minipage}{\columnwidth}
\begin{tabular}{lccc}
\raisebox{-0.5\height}{\includegraphics[width=0.32\columnwidth]{tract-3102-full_shrunk.png}} & \raisebox{-0.5\height}{\includegraphics[width=0.32\columnwidth]{blockgroup-3102-full_shrunk.png}} & Missing \\
\raisebox{-0.5\height}{\includegraphics[width=0.32\columnwidth]{tract-3102-net_shrunk.png}} & \raisebox{-0.5\height}{\includegraphics[width=0.32\columnwidth]{blockgroup-3102-net_shrunk.png}} & Missing \\
\end{tabular}
\imtitle{Nebraska 2}
\end{minipage}

\begin{minipage}{\columnwidth}
\begin{tabular}{lccc}
\raisebox{-0.5\height}{\includegraphics[width=0.32\columnwidth]{tract-3103-full_shrunk.png}} & \raisebox{-0.5\height}{\includegraphics[width=0.32\columnwidth]{blockgroup-3103-full_shrunk.png}} & Missing \\
\raisebox{-0.5\height}{\includegraphics[width=0.32\columnwidth]{tract-3103-net_shrunk.png}} & \raisebox{-0.5\height}{\includegraphics[width=0.32\columnwidth]{blockgroup-3103-net_shrunk.png}} & Missing \\
\end{tabular}
\imtitle{Nebraska 3}
\end{minipage}
\bchap{Nevada}
\begin{minipage}{\columnwidth}
\begin{tabular}{lccc}
\raisebox{-0.5\height}{\includegraphics[width=0.32\columnwidth]{tract-3201-full_shrunk.png}} & \raisebox{-0.5\height}{\includegraphics[width=0.32\columnwidth]{blockgroup-3201-full_shrunk.png}} & Missing \\
\raisebox{-0.5\height}{\includegraphics[width=0.32\columnwidth]{tract-3201-net_shrunk.png}} & \raisebox{-0.5\height}{\includegraphics[width=0.32\columnwidth]{blockgroup-3201-net_shrunk.png}} & Missing \\
\end{tabular}
\imtitle{Nevada 1}
\end{minipage}

\begin{minipage}{\columnwidth}
\begin{tabular}{lccc}
\raisebox{-0.5\height}{\includegraphics[width=0.32\columnwidth]{tract-3202-full_shrunk.png}} & \raisebox{-0.5\height}{\includegraphics[width=0.32\columnwidth]{blockgroup-3202-full_shrunk.png}} & Missing \\
\raisebox{-0.5\height}{\includegraphics[width=0.32\columnwidth]{tract-3202-net_shrunk.png}} & \raisebox{-0.5\height}{\includegraphics[width=0.32\columnwidth]{blockgroup-3202-net_shrunk.png}} & Missing \\
\end{tabular}
\imtitle{Nevada 2}
\end{minipage}

\begin{minipage}{\columnwidth}
\begin{tabular}{lccc}
\raisebox{-0.5\height}{\includegraphics[width=0.32\columnwidth]{tract-3203-full_shrunk.png}} & \raisebox{-0.5\height}{\includegraphics[width=0.32\columnwidth]{blockgroup-3203-full_shrunk.png}} & Missing \\
\raisebox{-0.5\height}{\includegraphics[width=0.32\columnwidth]{tract-3203-net_shrunk.png}} & \raisebox{-0.5\height}{\includegraphics[width=0.32\columnwidth]{blockgroup-3203-net_shrunk.png}} & Missing \\
\end{tabular}
\imtitle{Nevada 3}
\end{minipage}

\begin{minipage}{\columnwidth}
\begin{tabular}{lccc}
Missing & Missing & Missing \\
Missing & Missing & Missing \\
\end{tabular}
\imtitle{Nevada 4}
\end{minipage}
\bchap{New Hampshire}
\begin{minipage}{\columnwidth}
\begin{tabular}{lccc}
\raisebox{-0.5\height}{\includegraphics[width=0.32\columnwidth]{tract-3301-full_shrunk.png}} & \raisebox{-0.5\height}{\includegraphics[width=0.32\columnwidth]{blockgroup-3301-full_shrunk.png}} & Missing \\
\raisebox{-0.5\height}{\includegraphics[width=0.32\columnwidth]{tract-3301-net_shrunk.png}} & \raisebox{-0.5\height}{\includegraphics[width=0.32\columnwidth]{blockgroup-3301-net_shrunk.png}} & Missing \\
\end{tabular}
\imtitle{New Hampshire 1}
\end{minipage}

\begin{minipage}{\columnwidth}
\begin{tabular}{lccc}
\raisebox{-0.5\height}{\includegraphics[width=0.32\columnwidth]{tract-3302-full_shrunk.png}} & \raisebox{-0.5\height}{\includegraphics[width=0.32\columnwidth]{blockgroup-3302-full_shrunk.png}} & Missing \\
\raisebox{-0.5\height}{\includegraphics[width=0.32\columnwidth]{tract-3302-net_shrunk.png}} & \raisebox{-0.5\height}{\includegraphics[width=0.32\columnwidth]{blockgroup-3302-net_shrunk.png}} & Missing \\
\end{tabular}
\imtitle{New Hampshire 2}
\end{minipage}
\bchap{New Jersey}
\begin{minipage}{\columnwidth}
\begin{tabular}{lccc}
\raisebox{-0.5\height}{\includegraphics[width=0.32\columnwidth]{tract-3401-full_shrunk.png}} & \raisebox{-0.5\height}{\includegraphics[width=0.32\columnwidth]{blockgroup-3401-full_shrunk.png}} & Missing \\
\raisebox{-0.5\height}{\includegraphics[width=0.32\columnwidth]{tract-3401-net_shrunk.png}} & \raisebox{-0.5\height}{\includegraphics[width=0.32\columnwidth]{blockgroup-3401-net_shrunk.png}} & Missing \\
\end{tabular}
\imtitle{New Jersey 1}
\end{minipage}

\begin{minipage}{\columnwidth}
\begin{tabular}{lccc}
\raisebox{-0.5\height}{\includegraphics[width=0.32\columnwidth]{tract-3402-full_shrunk.png}} & \raisebox{-0.5\height}{\includegraphics[width=0.32\columnwidth]{blockgroup-3402-full_shrunk.png}} & Missing \\
\raisebox{-0.5\height}{\includegraphics[width=0.32\columnwidth]{tract-3402-net_shrunk.png}} & \raisebox{-0.5\height}{\includegraphics[width=0.32\columnwidth]{blockgroup-3402-net_shrunk.png}} & Missing \\
\end{tabular}
\imtitle{New Jersey 2}
\end{minipage}

\begin{minipage}{\columnwidth}
\begin{tabular}{lccc}
\raisebox{-0.5\height}{\includegraphics[width=0.32\columnwidth]{tract-3403-full_shrunk.png}} & \raisebox{-0.5\height}{\includegraphics[width=0.32\columnwidth]{blockgroup-3403-full_shrunk.png}} & Missing \\
\raisebox{-0.5\height}{\includegraphics[width=0.32\columnwidth]{tract-3403-net_shrunk.png}} & \raisebox{-0.5\height}{\includegraphics[width=0.32\columnwidth]{blockgroup-3403-net_shrunk.png}} & Missing \\
\end{tabular}
\imtitle{New Jersey 3}
\end{minipage}

\begin{minipage}{\columnwidth}
\begin{tabular}{lccc}
\raisebox{-0.5\height}{\includegraphics[width=0.32\columnwidth]{tract-3404-full_shrunk.png}} & \raisebox{-0.5\height}{\includegraphics[width=0.32\columnwidth]{blockgroup-3404-full_shrunk.png}} & Missing \\
\raisebox{-0.5\height}{\includegraphics[width=0.32\columnwidth]{tract-3404-net_shrunk.png}} & \raisebox{-0.5\height}{\includegraphics[width=0.32\columnwidth]{blockgroup-3404-net_shrunk.png}} & Missing \\
\end{tabular}
\imtitle{New Jersey 4}
\end{minipage}

\begin{minipage}{\columnwidth}
\begin{tabular}{lccc}
\raisebox{-0.5\height}{\includegraphics[width=0.32\columnwidth]{tract-3405-full_shrunk.png}} & \raisebox{-0.5\height}{\includegraphics[width=0.32\columnwidth]{blockgroup-3405-full_shrunk.png}} & Missing \\
\raisebox{-0.5\height}{\includegraphics[width=0.32\columnwidth]{tract-3405-net_shrunk.png}} & \raisebox{-0.5\height}{\includegraphics[width=0.32\columnwidth]{blockgroup-3405-net_shrunk.png}} & Missing \\
\end{tabular}
\imtitle{New Jersey 5}
\end{minipage}

\begin{minipage}{\columnwidth}
\begin{tabular}{lccc}
\raisebox{-0.5\height}{\includegraphics[width=0.32\columnwidth]{tract-3406-full_shrunk.png}} & \raisebox{-0.5\height}{\includegraphics[width=0.32\columnwidth]{blockgroup-3406-full_shrunk.png}} & Missing \\
\raisebox{-0.5\height}{\includegraphics[width=0.32\columnwidth]{tract-3406-net_shrunk.png}} & \raisebox{-0.5\height}{\includegraphics[width=0.32\columnwidth]{blockgroup-3406-net_shrunk.png}} & Missing \\
\end{tabular}
\imtitle{New Jersey 6}
\end{minipage}

\begin{minipage}{\columnwidth}
\begin{tabular}{lccc}
\raisebox{-0.5\height}{\includegraphics[width=0.32\columnwidth]{tract-3407-full_shrunk.png}} & \raisebox{-0.5\height}{\includegraphics[width=0.32\columnwidth]{blockgroup-3407-full_shrunk.png}} & Missing \\
\raisebox{-0.5\height}{\includegraphics[width=0.32\columnwidth]{tract-3407-net_shrunk.png}} & \raisebox{-0.5\height}{\includegraphics[width=0.32\columnwidth]{blockgroup-3407-net_shrunk.png}} & Missing \\
\end{tabular}
\imtitle{New Jersey 7}
\end{minipage}

\begin{minipage}{\columnwidth}
\begin{tabular}{lccc}
\raisebox{-0.5\height}{\includegraphics[width=0.32\columnwidth]{tract-3408-full_shrunk.png}} & \raisebox{-0.5\height}{\includegraphics[width=0.32\columnwidth]{blockgroup-3408-full_shrunk.png}} & Missing \\
\raisebox{-0.5\height}{\includegraphics[width=0.32\columnwidth]{tract-3408-net_shrunk.png}} & \raisebox{-0.5\height}{\includegraphics[width=0.32\columnwidth]{blockgroup-3408-net_shrunk.png}} & Missing \\
\end{tabular}
\imtitle{New Jersey 8}
\end{minipage}

\begin{minipage}{\columnwidth}
\begin{tabular}{lccc}
\raisebox{-0.5\height}{\includegraphics[width=0.32\columnwidth]{tract-3409-full_shrunk.png}} & \raisebox{-0.5\height}{\includegraphics[width=0.32\columnwidth]{blockgroup-3409-full_shrunk.png}} & Missing \\
\raisebox{-0.5\height}{\includegraphics[width=0.32\columnwidth]{tract-3409-net_shrunk.png}} & \raisebox{-0.5\height}{\includegraphics[width=0.32\columnwidth]{blockgroup-3409-net_shrunk.png}} & Missing \\
\end{tabular}
\imtitle{New Jersey 9}
\end{minipage}

\begin{minipage}{\columnwidth}
\begin{tabular}{lccc}
\raisebox{-0.5\height}{\includegraphics[width=0.32\columnwidth]{tract-3410-full_shrunk.png}} & \raisebox{-0.5\height}{\includegraphics[width=0.32\columnwidth]{blockgroup-3410-full_shrunk.png}} & Missing \\
\raisebox{-0.5\height}{\includegraphics[width=0.32\columnwidth]{tract-3410-net_shrunk.png}} & \raisebox{-0.5\height}{\includegraphics[width=0.32\columnwidth]{blockgroup-3410-net_shrunk.png}} & Missing \\
\end{tabular}
\imtitle{New Jersey 10}
\end{minipage}

\begin{minipage}{\columnwidth}
\begin{tabular}{lccc}
\raisebox{-0.5\height}{\includegraphics[width=0.32\columnwidth]{tract-3411-full_shrunk.png}} & \raisebox{-0.5\height}{\includegraphics[width=0.32\columnwidth]{blockgroup-3411-full_shrunk.png}} & Missing \\
\raisebox{-0.5\height}{\includegraphics[width=0.32\columnwidth]{tract-3411-net_shrunk.png}} & \raisebox{-0.5\height}{\includegraphics[width=0.32\columnwidth]{blockgroup-3411-net_shrunk.png}} & Missing \\
\end{tabular}
\imtitle{New Jersey 11}
\end{minipage}

\begin{minipage}{\columnwidth}
\begin{tabular}{lccc}
\raisebox{-0.5\height}{\includegraphics[width=0.32\columnwidth]{tract-3412-full_shrunk.png}} & \raisebox{-0.5\height}{\includegraphics[width=0.32\columnwidth]{blockgroup-3412-full_shrunk.png}} & Missing \\
\raisebox{-0.5\height}{\includegraphics[width=0.32\columnwidth]{tract-3412-net_shrunk.png}} & \raisebox{-0.5\height}{\includegraphics[width=0.32\columnwidth]{blockgroup-3412-net_shrunk.png}} & Missing \\
\end{tabular}
\imtitle{New Jersey 12}
\end{minipage}
\bchap{New Mexico}
\begin{minipage}{\columnwidth}
\begin{tabular}{lccc}
\raisebox{-0.5\height}{\includegraphics[width=0.32\columnwidth]{tract-3501-full_shrunk.png}} & \raisebox{-0.5\height}{\includegraphics[width=0.32\columnwidth]{blockgroup-3501-full_shrunk.png}} & Missing \\
\raisebox{-0.5\height}{\includegraphics[width=0.32\columnwidth]{tract-3501-net_shrunk.png}} & \raisebox{-0.5\height}{\includegraphics[width=0.32\columnwidth]{blockgroup-3501-net_shrunk.png}} & Missing \\
\end{tabular}
\imtitle{New Mexico 1}
\end{minipage}

\begin{minipage}{\columnwidth}
\begin{tabular}{lccc}
\raisebox{-0.5\height}{\includegraphics[width=0.32\columnwidth]{tract-3502-full_shrunk.png}} & \raisebox{-0.5\height}{\includegraphics[width=0.32\columnwidth]{blockgroup-3502-full_shrunk.png}} & Missing \\
\raisebox{-0.5\height}{\includegraphics[width=0.32\columnwidth]{tract-3502-net_shrunk.png}} & \raisebox{-0.5\height}{\includegraphics[width=0.32\columnwidth]{blockgroup-3502-net_shrunk.png}} & Missing \\
\end{tabular}
\imtitle{New Mexico 2}
\end{minipage}

\begin{minipage}{\columnwidth}
\begin{tabular}{lccc}
\raisebox{-0.5\height}{\includegraphics[width=0.32\columnwidth]{tract-3503-full_shrunk.png}} & \raisebox{-0.5\height}{\includegraphics[width=0.32\columnwidth]{blockgroup-3503-full_shrunk.png}} & \raisebox{-0.5\height}{\includegraphics[width=0.32\columnwidth]{block-3503-full_shrunk.png}} \\
\raisebox{-0.5\height}{\includegraphics[width=0.32\columnwidth]{tract-3503-net_shrunk.png}} & \raisebox{-0.5\height}{\includegraphics[width=0.32\columnwidth]{blockgroup-3503-net_shrunk.png}} & \raisebox{-0.5\height}{\includegraphics[width=0.32\columnwidth]{block-3503-net_shrunk.png}} \\
\end{tabular}
\imtitle{New Mexico 3}
\end{minipage}
\bchap{New York}
\begin{minipage}{\columnwidth}
\begin{tabular}{lccc}
\raisebox{-0.5\height}{\includegraphics[width=0.32\columnwidth]{tract-3601-full_shrunk.png}} & \raisebox{-0.5\height}{\includegraphics[width=0.32\columnwidth]{blockgroup-3601-full_shrunk.png}} & Missing \\
\raisebox{-0.5\height}{\includegraphics[width=0.32\columnwidth]{tract-3601-net_shrunk.png}} & \raisebox{-0.5\height}{\includegraphics[width=0.32\columnwidth]{blockgroup-3601-net_shrunk.png}} & Missing \\
\end{tabular}
\imtitle{New York 1}
\end{minipage}

\begin{minipage}{\columnwidth}
\begin{tabular}{lccc}
\raisebox{-0.5\height}{\includegraphics[width=0.32\columnwidth]{tract-3602-full_shrunk.png}} & \raisebox{-0.5\height}{\includegraphics[width=0.32\columnwidth]{blockgroup-3602-full_shrunk.png}} & Missing \\
\raisebox{-0.5\height}{\includegraphics[width=0.32\columnwidth]{tract-3602-net_shrunk.png}} & \raisebox{-0.5\height}{\includegraphics[width=0.32\columnwidth]{blockgroup-3602-net_shrunk.png}} & Missing \\
\end{tabular}
\imtitle{New York 2}
\end{minipage}

\begin{minipage}{\columnwidth}
\begin{tabular}{lccc}
\raisebox{-0.5\height}{\includegraphics[width=0.32\columnwidth]{tract-3603-full_shrunk.png}} & \raisebox{-0.5\height}{\includegraphics[width=0.32\columnwidth]{blockgroup-3603-full_shrunk.png}} & Missing \\
\raisebox{-0.5\height}{\includegraphics[width=0.32\columnwidth]{tract-3603-net_shrunk.png}} & \raisebox{-0.5\height}{\includegraphics[width=0.32\columnwidth]{blockgroup-3603-net_shrunk.png}} & Missing \\
\end{tabular}
\imtitle{New York 3}
\end{minipage}

\begin{minipage}{\columnwidth}
\begin{tabular}{lccc}
\raisebox{-0.5\height}{\includegraphics[width=0.32\columnwidth]{tract-3604-full_shrunk.png}} & \raisebox{-0.5\height}{\includegraphics[width=0.32\columnwidth]{blockgroup-3604-full_shrunk.png}} & Missing \\
\raisebox{-0.5\height}{\includegraphics[width=0.32\columnwidth]{tract-3604-net_shrunk.png}} & \raisebox{-0.5\height}{\includegraphics[width=0.32\columnwidth]{blockgroup-3604-net_shrunk.png}} & Missing \\
\end{tabular}
\imtitle{New York 4}
\end{minipage}

\begin{minipage}{\columnwidth}
\begin{tabular}{lccc}
\raisebox{-0.5\height}{\includegraphics[width=0.32\columnwidth]{tract-3605-full_shrunk.png}} & \raisebox{-0.5\height}{\includegraphics[width=0.32\columnwidth]{blockgroup-3605-full_shrunk.png}} & Missing \\
\raisebox{-0.5\height}{\includegraphics[width=0.32\columnwidth]{tract-3605-net_shrunk.png}} & \raisebox{-0.5\height}{\includegraphics[width=0.32\columnwidth]{blockgroup-3605-net_shrunk.png}} & Missing \\
\end{tabular}
\imtitle{New York 5}
\end{minipage}

\begin{minipage}{\columnwidth}
\begin{tabular}{lccc}
\raisebox{-0.5\height}{\includegraphics[width=0.32\columnwidth]{tract-3606-full_shrunk.png}} & \raisebox{-0.5\height}{\includegraphics[width=0.32\columnwidth]{blockgroup-3606-full_shrunk.png}} & Missing \\
\raisebox{-0.5\height}{\includegraphics[width=0.32\columnwidth]{tract-3606-net_shrunk.png}} & \raisebox{-0.5\height}{\includegraphics[width=0.32\columnwidth]{blockgroup-3606-net_shrunk.png}} & Missing \\
\end{tabular}
\imtitle{New York 6}
\end{minipage}

\begin{minipage}{\columnwidth}
\begin{tabular}{lccc}
\raisebox{-0.5\height}{\includegraphics[width=0.32\columnwidth]{tract-3607-full_shrunk.png}} & \raisebox{-0.5\height}{\includegraphics[width=0.32\columnwidth]{blockgroup-3607-full_shrunk.png}} & Missing \\
\raisebox{-0.5\height}{\includegraphics[width=0.32\columnwidth]{tract-3607-net_shrunk.png}} & \raisebox{-0.5\height}{\includegraphics[width=0.32\columnwidth]{blockgroup-3607-net_shrunk.png}} & Missing \\
\end{tabular}
\imtitle{New York 7}
\end{minipage}

\begin{minipage}{\columnwidth}
\begin{tabular}{lccc}
\raisebox{-0.5\height}{\includegraphics[width=0.32\columnwidth]{tract-3608-full_shrunk.png}} & \raisebox{-0.5\height}{\includegraphics[width=0.32\columnwidth]{blockgroup-3608-full_shrunk.png}} & Missing \\
\raisebox{-0.5\height}{\includegraphics[width=0.32\columnwidth]{tract-3608-net_shrunk.png}} & \raisebox{-0.5\height}{\includegraphics[width=0.32\columnwidth]{blockgroup-3608-net_shrunk.png}} & Missing \\
\end{tabular}
\imtitle{New York 8}
\end{minipage}

\begin{minipage}{\columnwidth}
\begin{tabular}{lccc}
\raisebox{-0.5\height}{\includegraphics[width=0.32\columnwidth]{tract-3609-full_shrunk.png}} & \raisebox{-0.5\height}{\includegraphics[width=0.32\columnwidth]{blockgroup-3609-full_shrunk.png}} & Missing \\
\raisebox{-0.5\height}{\includegraphics[width=0.32\columnwidth]{tract-3609-net_shrunk.png}} & \raisebox{-0.5\height}{\includegraphics[width=0.32\columnwidth]{blockgroup-3609-net_shrunk.png}} & Missing \\
\end{tabular}
\imtitle{New York 9}
\end{minipage}

\begin{minipage}{\columnwidth}
\begin{tabular}{lccc}
\raisebox{-0.5\height}{\includegraphics[width=0.32\columnwidth]{tract-3610-full_shrunk.png}} & \raisebox{-0.5\height}{\includegraphics[width=0.32\columnwidth]{blockgroup-3610-full_shrunk.png}} & Missing \\
\raisebox{-0.5\height}{\includegraphics[width=0.32\columnwidth]{tract-3610-net_shrunk.png}} & \raisebox{-0.5\height}{\includegraphics[width=0.32\columnwidth]{blockgroup-3610-net_shrunk.png}} & Missing \\
\end{tabular}
\imtitle{New York 10}
\end{minipage}

\begin{minipage}{\columnwidth}
\begin{tabular}{lccc}
\raisebox{-0.5\height}{\includegraphics[width=0.32\columnwidth]{tract-3611-full_shrunk.png}} & \raisebox{-0.5\height}{\includegraphics[width=0.32\columnwidth]{blockgroup-3611-full_shrunk.png}} & Missing \\
\raisebox{-0.5\height}{\includegraphics[width=0.32\columnwidth]{tract-3611-net_shrunk.png}} & \raisebox{-0.5\height}{\includegraphics[width=0.32\columnwidth]{blockgroup-3611-net_shrunk.png}} & Missing \\
\end{tabular}
\imtitle{New York 11}
\end{minipage}

\begin{minipage}{\columnwidth}
\begin{tabular}{lccc}
\raisebox{-0.5\height}{\includegraphics[width=0.32\columnwidth]{tract-3612-full_shrunk.png}} & \raisebox{-0.5\height}{\includegraphics[width=0.32\columnwidth]{blockgroup-3612-full_shrunk.png}} & Missing \\
\raisebox{-0.5\height}{\includegraphics[width=0.32\columnwidth]{tract-3612-net_shrunk.png}} & \raisebox{-0.5\height}{\includegraphics[width=0.32\columnwidth]{blockgroup-3612-net_shrunk.png}} & Missing \\
\end{tabular}
\imtitle{New York 12}
\end{minipage}

\begin{minipage}{\columnwidth}
\begin{tabular}{lccc}
\raisebox{-0.5\height}{\includegraphics[width=0.32\columnwidth]{tract-3613-full_shrunk.png}} & \raisebox{-0.5\height}{\includegraphics[width=0.32\columnwidth]{blockgroup-3613-full_shrunk.png}} & Missing \\
\raisebox{-0.5\height}{\includegraphics[width=0.32\columnwidth]{tract-3613-net_shrunk.png}} & \raisebox{-0.5\height}{\includegraphics[width=0.32\columnwidth]{blockgroup-3613-net_shrunk.png}} & Missing \\
\end{tabular}
\imtitle{New York 13}
\end{minipage}

\begin{minipage}{\columnwidth}
\begin{tabular}{lccc}
\raisebox{-0.5\height}{\includegraphics[width=0.32\columnwidth]{tract-3614-full_shrunk.png}} & \raisebox{-0.5\height}{\includegraphics[width=0.32\columnwidth]{blockgroup-3614-full_shrunk.png}} & Missing \\
\raisebox{-0.5\height}{\includegraphics[width=0.32\columnwidth]{tract-3614-net_shrunk.png}} & \raisebox{-0.5\height}{\includegraphics[width=0.32\columnwidth]{blockgroup-3614-net_shrunk.png}} & Missing \\
\end{tabular}
\imtitle{New York 14}
\end{minipage}

\begin{minipage}{\columnwidth}
\begin{tabular}{lccc}
\raisebox{-0.5\height}{\includegraphics[width=0.32\columnwidth]{tract-3615-full_shrunk.png}} & \raisebox{-0.5\height}{\includegraphics[width=0.32\columnwidth]{blockgroup-3615-full_shrunk.png}} & Missing \\
\raisebox{-0.5\height}{\includegraphics[width=0.32\columnwidth]{tract-3615-net_shrunk.png}} & \raisebox{-0.5\height}{\includegraphics[width=0.32\columnwidth]{blockgroup-3615-net_shrunk.png}} & Missing \\
\end{tabular}
\imtitle{New York 15}
\end{minipage}

\begin{minipage}{\columnwidth}
\begin{tabular}{lccc}
\raisebox{-0.5\height}{\includegraphics[width=0.32\columnwidth]{tract-3616-full_shrunk.png}} & \raisebox{-0.5\height}{\includegraphics[width=0.32\columnwidth]{blockgroup-3616-full_shrunk.png}} & Missing \\
\raisebox{-0.5\height}{\includegraphics[width=0.32\columnwidth]{tract-3616-net_shrunk.png}} & \raisebox{-0.5\height}{\includegraphics[width=0.32\columnwidth]{blockgroup-3616-net_shrunk.png}} & Missing \\
\end{tabular}
\imtitle{New York 16}
\end{minipage}

\begin{minipage}{\columnwidth}
\begin{tabular}{lccc}
\raisebox{-0.5\height}{\includegraphics[width=0.32\columnwidth]{tract-3617-full_shrunk.png}} & \raisebox{-0.5\height}{\includegraphics[width=0.32\columnwidth]{blockgroup-3617-full_shrunk.png}} & Missing \\
\raisebox{-0.5\height}{\includegraphics[width=0.32\columnwidth]{tract-3617-net_shrunk.png}} & \raisebox{-0.5\height}{\includegraphics[width=0.32\columnwidth]{blockgroup-3617-net_shrunk.png}} & Missing \\
\end{tabular}
\imtitle{New York 17}
\end{minipage}

\begin{minipage}{\columnwidth}
\begin{tabular}{lccc}
\raisebox{-0.5\height}{\includegraphics[width=0.32\columnwidth]{tract-3618-full_shrunk.png}} & \raisebox{-0.5\height}{\includegraphics[width=0.32\columnwidth]{blockgroup-3618-full_shrunk.png}} & Missing \\
\raisebox{-0.5\height}{\includegraphics[width=0.32\columnwidth]{tract-3618-net_shrunk.png}} & \raisebox{-0.5\height}{\includegraphics[width=0.32\columnwidth]{blockgroup-3618-net_shrunk.png}} & Missing \\
\end{tabular}
\imtitle{New York 18}
\end{minipage}

\begin{minipage}{\columnwidth}
\begin{tabular}{lccc}
\raisebox{-0.5\height}{\includegraphics[width=0.32\columnwidth]{tract-3619-full_shrunk.png}} & \raisebox{-0.5\height}{\includegraphics[width=0.32\columnwidth]{blockgroup-3619-full_shrunk.png}} & Missing \\
\raisebox{-0.5\height}{\includegraphics[width=0.32\columnwidth]{tract-3619-net_shrunk.png}} & \raisebox{-0.5\height}{\includegraphics[width=0.32\columnwidth]{blockgroup-3619-net_shrunk.png}} & Missing \\
\end{tabular}
\imtitle{New York 19}
\end{minipage}

\begin{minipage}{\columnwidth}
\begin{tabular}{lccc}
\raisebox{-0.5\height}{\includegraphics[width=0.32\columnwidth]{tract-3620-full_shrunk.png}} & \raisebox{-0.5\height}{\includegraphics[width=0.32\columnwidth]{blockgroup-3620-full_shrunk.png}} & Missing \\
\raisebox{-0.5\height}{\includegraphics[width=0.32\columnwidth]{tract-3620-net_shrunk.png}} & \raisebox{-0.5\height}{\includegraphics[width=0.32\columnwidth]{blockgroup-3620-net_shrunk.png}} & Missing \\
\end{tabular}
\imtitle{New York 20}
\end{minipage}

\begin{minipage}{\columnwidth}
\begin{tabular}{lccc}
\raisebox{-0.5\height}{\includegraphics[width=0.32\columnwidth]{tract-3621-full_shrunk.png}} & \raisebox{-0.5\height}{\includegraphics[width=0.32\columnwidth]{blockgroup-3621-full_shrunk.png}} & Missing \\
\raisebox{-0.5\height}{\includegraphics[width=0.32\columnwidth]{tract-3621-net_shrunk.png}} & \raisebox{-0.5\height}{\includegraphics[width=0.32\columnwidth]{blockgroup-3621-net_shrunk.png}} & Missing \\
\end{tabular}
\imtitle{New York 21}
\end{minipage}

\begin{minipage}{\columnwidth}
\begin{tabular}{lccc}
\raisebox{-0.5\height}{\includegraphics[width=0.32\columnwidth]{tract-3622-full_shrunk.png}} & \raisebox{-0.5\height}{\includegraphics[width=0.32\columnwidth]{blockgroup-3622-full_shrunk.png}} & Missing \\
\raisebox{-0.5\height}{\includegraphics[width=0.32\columnwidth]{tract-3622-net_shrunk.png}} & \raisebox{-0.5\height}{\includegraphics[width=0.32\columnwidth]{blockgroup-3622-net_shrunk.png}} & Missing \\
\end{tabular}
\imtitle{New York 22}
\end{minipage}

\begin{minipage}{\columnwidth}
\begin{tabular}{lccc}
\raisebox{-0.5\height}{\includegraphics[width=0.32\columnwidth]{tract-3623-full_shrunk.png}} & \raisebox{-0.5\height}{\includegraphics[width=0.32\columnwidth]{blockgroup-3623-full_shrunk.png}} & Missing \\
\raisebox{-0.5\height}{\includegraphics[width=0.32\columnwidth]{tract-3623-net_shrunk.png}} & \raisebox{-0.5\height}{\includegraphics[width=0.32\columnwidth]{blockgroup-3623-net_shrunk.png}} & Missing \\
\end{tabular}
\imtitle{New York 23}
\end{minipage}

\begin{minipage}{\columnwidth}
\begin{tabular}{lccc}
\raisebox{-0.5\height}{\includegraphics[width=0.32\columnwidth]{tract-3624-full_shrunk.png}} & \raisebox{-0.5\height}{\includegraphics[width=0.32\columnwidth]{blockgroup-3624-full_shrunk.png}} & Missing \\
\raisebox{-0.5\height}{\includegraphics[width=0.32\columnwidth]{tract-3624-net_shrunk.png}} & \raisebox{-0.5\height}{\includegraphics[width=0.32\columnwidth]{blockgroup-3624-net_shrunk.png}} & Missing \\
\end{tabular}
\imtitle{New York 24}
\end{minipage}

\begin{minipage}{\columnwidth}
\begin{tabular}{lccc}
\raisebox{-0.5\height}{\includegraphics[width=0.32\columnwidth]{tract-3625-full_shrunk.png}} & \raisebox{-0.5\height}{\includegraphics[width=0.32\columnwidth]{blockgroup-3625-full_shrunk.png}} & Missing \\
\raisebox{-0.5\height}{\includegraphics[width=0.32\columnwidth]{tract-3625-net_shrunk.png}} & \raisebox{-0.5\height}{\includegraphics[width=0.32\columnwidth]{blockgroup-3625-net_shrunk.png}} & Missing \\
\end{tabular}
\imtitle{New York 25}
\end{minipage}

\begin{minipage}{\columnwidth}
\begin{tabular}{lccc}
\raisebox{-0.5\height}{\includegraphics[width=0.32\columnwidth]{tract-3626-full_shrunk.png}} & \raisebox{-0.5\height}{\includegraphics[width=0.32\columnwidth]{blockgroup-3626-full_shrunk.png}} & Missing \\
\raisebox{-0.5\height}{\includegraphics[width=0.32\columnwidth]{tract-3626-net_shrunk.png}} & \raisebox{-0.5\height}{\includegraphics[width=0.32\columnwidth]{blockgroup-3626-net_shrunk.png}} & Missing \\
\end{tabular}
\imtitle{New York 26}
\end{minipage}

\begin{minipage}{\columnwidth}
\begin{tabular}{lccc}
\raisebox{-0.5\height}{\includegraphics[width=0.32\columnwidth]{tract-3627-full_shrunk.png}} & \raisebox{-0.5\height}{\includegraphics[width=0.32\columnwidth]{blockgroup-3627-full_shrunk.png}} & Missing \\
\raisebox{-0.5\height}{\includegraphics[width=0.32\columnwidth]{tract-3627-net_shrunk.png}} & \raisebox{-0.5\height}{\includegraphics[width=0.32\columnwidth]{blockgroup-3627-net_shrunk.png}} & Missing \\
\end{tabular}
\imtitle{New York 27}
\end{minipage}
\bchap{North Carolina}
\begin{minipage}{\columnwidth}
\begin{tabular}{lccc}
\raisebox{-0.5\height}{\includegraphics[width=0.32\columnwidth]{tract-3701-full_shrunk.png}} & \raisebox{-0.5\height}{\includegraphics[width=0.32\columnwidth]{blockgroup-3701-full_shrunk.png}} & Missing \\
\raisebox{-0.5\height}{\includegraphics[width=0.32\columnwidth]{tract-3701-net_shrunk.png}} & \raisebox{-0.5\height}{\includegraphics[width=0.32\columnwidth]{blockgroup-3701-net_shrunk.png}} & Missing \\
\end{tabular}
\imtitle{North Carolina 1}
\end{minipage}

\begin{minipage}{\columnwidth}
\begin{tabular}{lccc}
\raisebox{-0.5\height}{\includegraphics[width=0.32\columnwidth]{tract-3702-full_shrunk.png}} & \raisebox{-0.5\height}{\includegraphics[width=0.32\columnwidth]{blockgroup-3702-full_shrunk.png}} & Missing \\
\raisebox{-0.5\height}{\includegraphics[width=0.32\columnwidth]{tract-3702-net_shrunk.png}} & \raisebox{-0.5\height}{\includegraphics[width=0.32\columnwidth]{blockgroup-3702-net_shrunk.png}} & Missing \\
\end{tabular}
\imtitle{North Carolina 2}
\end{minipage}

\begin{minipage}{\columnwidth}
\begin{tabular}{lccc}
\raisebox{-0.5\height}{\includegraphics[width=0.32\columnwidth]{tract-3703-full_shrunk.png}} & \raisebox{-0.5\height}{\includegraphics[width=0.32\columnwidth]{blockgroup-3703-full_shrunk.png}} & Missing \\
\raisebox{-0.5\height}{\includegraphics[width=0.32\columnwidth]{tract-3703-net_shrunk.png}} & \raisebox{-0.5\height}{\includegraphics[width=0.32\columnwidth]{blockgroup-3703-net_shrunk.png}} & Missing \\
\end{tabular}
\imtitle{North Carolina 3}
\end{minipage}

\begin{minipage}{\columnwidth}
\begin{tabular}{lccc}
\raisebox{-0.5\height}{\includegraphics[width=0.32\columnwidth]{tract-3704-full_shrunk.png}} & \raisebox{-0.5\height}{\includegraphics[width=0.32\columnwidth]{blockgroup-3704-full_shrunk.png}} & Missing \\
\raisebox{-0.5\height}{\includegraphics[width=0.32\columnwidth]{tract-3704-net_shrunk.png}} & \raisebox{-0.5\height}{\includegraphics[width=0.32\columnwidth]{blockgroup-3704-net_shrunk.png}} & Missing \\
\end{tabular}
\imtitle{North Carolina 4}
\end{minipage}

\begin{minipage}{\columnwidth}
\begin{tabular}{lccc}
\raisebox{-0.5\height}{\includegraphics[width=0.32\columnwidth]{tract-3705-full_shrunk.png}} & \raisebox{-0.5\height}{\includegraphics[width=0.32\columnwidth]{blockgroup-3705-full_shrunk.png}} & Missing \\
\raisebox{-0.5\height}{\includegraphics[width=0.32\columnwidth]{tract-3705-net_shrunk.png}} & \raisebox{-0.5\height}{\includegraphics[width=0.32\columnwidth]{blockgroup-3705-net_shrunk.png}} & Missing \\
\end{tabular}
\imtitle{North Carolina 5}
\end{minipage}

\begin{minipage}{\columnwidth}
\begin{tabular}{lccc}
\raisebox{-0.5\height}{\includegraphics[width=0.32\columnwidth]{tract-3706-full_shrunk.png}} & \raisebox{-0.5\height}{\includegraphics[width=0.32\columnwidth]{blockgroup-3706-full_shrunk.png}} & Missing \\
\raisebox{-0.5\height}{\includegraphics[width=0.32\columnwidth]{tract-3706-net_shrunk.png}} & \raisebox{-0.5\height}{\includegraphics[width=0.32\columnwidth]{blockgroup-3706-net_shrunk.png}} & Missing \\
\end{tabular}
\imtitle{North Carolina 6}
\end{minipage}

\begin{minipage}{\columnwidth}
\begin{tabular}{lccc}
\raisebox{-0.5\height}{\includegraphics[width=0.32\columnwidth]{tract-3707-full_shrunk.png}} & \raisebox{-0.5\height}{\includegraphics[width=0.32\columnwidth]{blockgroup-3707-full_shrunk.png}} & Missing \\
\raisebox{-0.5\height}{\includegraphics[width=0.32\columnwidth]{tract-3707-net_shrunk.png}} & \raisebox{-0.5\height}{\includegraphics[width=0.32\columnwidth]{blockgroup-3707-net_shrunk.png}} & Missing \\
\end{tabular}
\imtitle{North Carolina 7}
\end{minipage}

\begin{minipage}{\columnwidth}
\begin{tabular}{lccc}
\raisebox{-0.5\height}{\includegraphics[width=0.32\columnwidth]{tract-3708-full_shrunk.png}} & \raisebox{-0.5\height}{\includegraphics[width=0.32\columnwidth]{blockgroup-3708-full_shrunk.png}} & Missing \\
\raisebox{-0.5\height}{\includegraphics[width=0.32\columnwidth]{tract-3708-net_shrunk.png}} & \raisebox{-0.5\height}{\includegraphics[width=0.32\columnwidth]{blockgroup-3708-net_shrunk.png}} & Missing \\
\end{tabular}
\imtitle{North Carolina 8}
\end{minipage}

\begin{minipage}{\columnwidth}
\begin{tabular}{lccc}
\raisebox{-0.5\height}{\includegraphics[width=0.32\columnwidth]{tract-3709-full_shrunk.png}} & \raisebox{-0.5\height}{\includegraphics[width=0.32\columnwidth]{blockgroup-3709-full_shrunk.png}} & Missing \\
\raisebox{-0.5\height}{\includegraphics[width=0.32\columnwidth]{tract-3709-net_shrunk.png}} & \raisebox{-0.5\height}{\includegraphics[width=0.32\columnwidth]{blockgroup-3709-net_shrunk.png}} & Missing \\
\end{tabular}
\imtitle{North Carolina 9}
\end{minipage}

\begin{minipage}{\columnwidth}
\begin{tabular}{lccc}
\raisebox{-0.5\height}{\includegraphics[width=0.32\columnwidth]{tract-3710-full_shrunk.png}} & \raisebox{-0.5\height}{\includegraphics[width=0.32\columnwidth]{blockgroup-3710-full_shrunk.png}} & Missing \\
\raisebox{-0.5\height}{\includegraphics[width=0.32\columnwidth]{tract-3710-net_shrunk.png}} & \raisebox{-0.5\height}{\includegraphics[width=0.32\columnwidth]{blockgroup-3710-net_shrunk.png}} & Missing \\
\end{tabular}
\imtitle{North Carolina 10}
\end{minipage}

\begin{minipage}{\columnwidth}
\begin{tabular}{lccc}
\raisebox{-0.5\height}{\includegraphics[width=0.32\columnwidth]{tract-3711-full_shrunk.png}} & \raisebox{-0.5\height}{\includegraphics[width=0.32\columnwidth]{blockgroup-3711-full_shrunk.png}} & Missing \\
\raisebox{-0.5\height}{\includegraphics[width=0.32\columnwidth]{tract-3711-net_shrunk.png}} & \raisebox{-0.5\height}{\includegraphics[width=0.32\columnwidth]{blockgroup-3711-net_shrunk.png}} & Missing \\
\end{tabular}
\imtitle{North Carolina 11}
\end{minipage}

\begin{minipage}{\columnwidth}
\begin{tabular}{lccc}
\raisebox{-0.5\height}{\includegraphics[width=0.32\columnwidth]{tract-3712-full_shrunk.png}} & \raisebox{-0.5\height}{\includegraphics[width=0.32\columnwidth]{blockgroup-3712-full_shrunk.png}} & Missing \\
\raisebox{-0.5\height}{\includegraphics[width=0.32\columnwidth]{tract-3712-net_shrunk.png}} & \raisebox{-0.5\height}{\includegraphics[width=0.32\columnwidth]{blockgroup-3712-net_shrunk.png}} & Missing \\
\end{tabular}
\imtitle{North Carolina 12}
\end{minipage}

\begin{minipage}{\columnwidth}
\begin{tabular}{lccc}
\raisebox{-0.5\height}{\includegraphics[width=0.32\columnwidth]{tract-3713-full_shrunk.png}} & \raisebox{-0.5\height}{\includegraphics[width=0.32\columnwidth]{blockgroup-3713-full_shrunk.png}} & Missing \\
\raisebox{-0.5\height}{\includegraphics[width=0.32\columnwidth]{tract-3713-net_shrunk.png}} & \raisebox{-0.5\height}{\includegraphics[width=0.32\columnwidth]{blockgroup-3713-net_shrunk.png}} & Missing \\
\end{tabular}
\imtitle{North Carolina 13}
\end{minipage}
\bchap{North Dakota}
\begin{minipage}{\columnwidth}
\begin{tabular}{lccc}
\raisebox{-0.5\height}{\includegraphics[width=0.32\columnwidth]{tract-3800-full_shrunk.png}} & \raisebox{-0.5\height}{\includegraphics[width=0.32\columnwidth]{blockgroup-3800-full_shrunk.png}} & Missing \\
\raisebox{-0.5\height}{\includegraphics[width=0.32\columnwidth]{tract-3800-net_shrunk.png}} & \raisebox{-0.5\height}{\includegraphics[width=0.32\columnwidth]{blockgroup-3800-net_shrunk.png}} & Missing \\
\end{tabular}
\imtitle{North Dakota 0}
\end{minipage}
\bchap{Ohio}
\begin{minipage}{\columnwidth}
\begin{tabular}{lccc}
Missing & Missing & Missing \\
Missing & Missing & Missing \\
\end{tabular}
\imtitle{Ohio 1}
\end{minipage}

\begin{minipage}{\columnwidth}
\begin{tabular}{lccc}
Missing & Missing & Missing \\
Missing & Missing & Missing \\
\end{tabular}
\imtitle{Ohio 2}
\end{minipage}

\begin{minipage}{\columnwidth}
\begin{tabular}{lccc}
Missing & Missing & Missing \\
Missing & Missing & Missing \\
\end{tabular}
\imtitle{Ohio 3}
\end{minipage}

\begin{minipage}{\columnwidth}
\begin{tabular}{lccc}
Missing & Missing & Missing \\
Missing & Missing & Missing \\
\end{tabular}
\imtitle{Ohio 4}
\end{minipage}

\begin{minipage}{\columnwidth}
\begin{tabular}{lccc}
Missing & Missing & Missing \\
Missing & Missing & Missing \\
\end{tabular}
\imtitle{Ohio 5}
\end{minipage}

\begin{minipage}{\columnwidth}
\begin{tabular}{lccc}
Missing & Missing & Missing \\
Missing & Missing & Missing \\
\end{tabular}
\imtitle{Ohio 6}
\end{minipage}

\begin{minipage}{\columnwidth}
\begin{tabular}{lccc}
Missing & Missing & Missing \\
Missing & Missing & Missing \\
\end{tabular}
\imtitle{Ohio 7}
\end{minipage}

\begin{minipage}{\columnwidth}
\begin{tabular}{lccc}
Missing & Missing & Missing \\
Missing & Missing & Missing \\
\end{tabular}
\imtitle{Ohio 8}
\end{minipage}

\begin{minipage}{\columnwidth}
\begin{tabular}{lccc}
Missing & Missing & Missing \\
Missing & Missing & Missing \\
\end{tabular}
\imtitle{Ohio 9}
\end{minipage}

\begin{minipage}{\columnwidth}
\begin{tabular}{lccc}
Missing & Missing & Missing \\
Missing & Missing & Missing \\
\end{tabular}
\imtitle{Ohio 10}
\end{minipage}

\begin{minipage}{\columnwidth}
\begin{tabular}{lccc}
Missing & Missing & Missing \\
Missing & Missing & Missing \\
\end{tabular}
\imtitle{Ohio 11}
\end{minipage}

\begin{minipage}{\columnwidth}
\begin{tabular}{lccc}
Missing & Missing & Missing \\
Missing & Missing & Missing \\
\end{tabular}
\imtitle{Ohio 12}
\end{minipage}

\begin{minipage}{\columnwidth}
\begin{tabular}{lccc}
Missing & Missing & Missing \\
Missing & Missing & Missing \\
\end{tabular}
\imtitle{Ohio 13}
\end{minipage}

\begin{minipage}{\columnwidth}
\begin{tabular}{lccc}
Missing & Missing & Missing \\
Missing & Missing & Missing \\
\end{tabular}
\imtitle{Ohio 14}
\end{minipage}

\begin{minipage}{\columnwidth}
\begin{tabular}{lccc}
Missing & Missing & Missing \\
Missing & Missing & Missing \\
\end{tabular}
\imtitle{Ohio 15}
\end{minipage}

\begin{minipage}{\columnwidth}
\begin{tabular}{lccc}
Missing & Missing & Missing \\
Missing & Missing & Missing \\
\end{tabular}
\imtitle{Ohio 16}
\end{minipage}
\bchap{Oklahoma}
\begin{minipage}{\columnwidth}
\begin{tabular}{lccc}
\raisebox{-0.5\height}{\includegraphics[width=0.32\columnwidth]{tract-4001-full_shrunk.png}} & \raisebox{-0.5\height}{\includegraphics[width=0.32\columnwidth]{blockgroup-4001-full_shrunk.png}} & Missing \\
\raisebox{-0.5\height}{\includegraphics[width=0.32\columnwidth]{tract-4001-net_shrunk.png}} & \raisebox{-0.5\height}{\includegraphics[width=0.32\columnwidth]{blockgroup-4001-net_shrunk.png}} & Missing \\
\end{tabular}
\imtitle{Oklahoma 1}
\end{minipage}

\begin{minipage}{\columnwidth}
\begin{tabular}{lccc}
\raisebox{-0.5\height}{\includegraphics[width=0.32\columnwidth]{tract-4002-full_shrunk.png}} & \raisebox{-0.5\height}{\includegraphics[width=0.32\columnwidth]{blockgroup-4002-full_shrunk.png}} & Missing \\
\raisebox{-0.5\height}{\includegraphics[width=0.32\columnwidth]{tract-4002-net_shrunk.png}} & \raisebox{-0.5\height}{\includegraphics[width=0.32\columnwidth]{blockgroup-4002-net_shrunk.png}} & Missing \\
\end{tabular}
\imtitle{Oklahoma 2}
\end{minipage}

\begin{minipage}{\columnwidth}
\begin{tabular}{lccc}
\raisebox{-0.5\height}{\includegraphics[width=0.32\columnwidth]{tract-4003-full_shrunk.png}} & \raisebox{-0.5\height}{\includegraphics[width=0.32\columnwidth]{blockgroup-4003-full_shrunk.png}} & Missing \\
\raisebox{-0.5\height}{\includegraphics[width=0.32\columnwidth]{tract-4003-net_shrunk.png}} & \raisebox{-0.5\height}{\includegraphics[width=0.32\columnwidth]{blockgroup-4003-net_shrunk.png}} & Missing \\
\end{tabular}
\imtitle{Oklahoma 3}
\end{minipage}

\begin{minipage}{\columnwidth}
\begin{tabular}{lccc}
\raisebox{-0.5\height}{\includegraphics[width=0.32\columnwidth]{tract-4004-full_shrunk.png}} & \raisebox{-0.5\height}{\includegraphics[width=0.32\columnwidth]{blockgroup-4004-full_shrunk.png}} & Missing \\
\raisebox{-0.5\height}{\includegraphics[width=0.32\columnwidth]{tract-4004-net_shrunk.png}} & \raisebox{-0.5\height}{\includegraphics[width=0.32\columnwidth]{blockgroup-4004-net_shrunk.png}} & Missing \\
\end{tabular}
\imtitle{Oklahoma 4}
\end{minipage}

\begin{minipage}{\columnwidth}
\begin{tabular}{lccc}
\raisebox{-0.5\height}{\includegraphics[width=0.32\columnwidth]{tract-4005-full_shrunk.png}} & \raisebox{-0.5\height}{\includegraphics[width=0.32\columnwidth]{blockgroup-4005-full_shrunk.png}} & Missing \\
\raisebox{-0.5\height}{\includegraphics[width=0.32\columnwidth]{tract-4005-net_shrunk.png}} & \raisebox{-0.5\height}{\includegraphics[width=0.32\columnwidth]{blockgroup-4005-net_shrunk.png}} & Missing \\
\end{tabular}
\imtitle{Oklahoma 5}
\end{minipage}
\bchap{Oregon}
\begin{minipage}{\columnwidth}
\begin{tabular}{lccc}
\raisebox{-0.5\height}{\includegraphics[width=0.32\columnwidth]{tract-4101-full_shrunk.png}} & \raisebox{-0.5\height}{\includegraphics[width=0.32\columnwidth]{blockgroup-4101-full_shrunk.png}} & Missing \\
\raisebox{-0.5\height}{\includegraphics[width=0.32\columnwidth]{tract-4101-net_shrunk.png}} & \raisebox{-0.5\height}{\includegraphics[width=0.32\columnwidth]{blockgroup-4101-net_shrunk.png}} & Missing \\
\end{tabular}
\imtitle{Oregon 1}
\end{minipage}

\begin{minipage}{\columnwidth}
\begin{tabular}{lccc}
\raisebox{-0.5\height}{\includegraphics[width=0.32\columnwidth]{tract-4102-full_shrunk.png}} & \raisebox{-0.5\height}{\includegraphics[width=0.32\columnwidth]{blockgroup-4102-full_shrunk.png}} & Missing \\
\raisebox{-0.5\height}{\includegraphics[width=0.32\columnwidth]{tract-4102-net_shrunk.png}} & \raisebox{-0.5\height}{\includegraphics[width=0.32\columnwidth]{blockgroup-4102-net_shrunk.png}} & Missing \\
\end{tabular}
\imtitle{Oregon 2}
\end{minipage}

\begin{minipage}{\columnwidth}
\begin{tabular}{lccc}
\raisebox{-0.5\height}{\includegraphics[width=0.32\columnwidth]{tract-4103-full_shrunk.png}} & \raisebox{-0.5\height}{\includegraphics[width=0.32\columnwidth]{blockgroup-4103-full_shrunk.png}} & Missing \\
\raisebox{-0.5\height}{\includegraphics[width=0.32\columnwidth]{tract-4103-net_shrunk.png}} & \raisebox{-0.5\height}{\includegraphics[width=0.32\columnwidth]{blockgroup-4103-net_shrunk.png}} & Missing \\
\end{tabular}
\imtitle{Oregon 3}
\end{minipage}

\begin{minipage}{\columnwidth}
\begin{tabular}{lccc}
\raisebox{-0.5\height}{\includegraphics[width=0.32\columnwidth]{tract-4104-full_shrunk.png}} & \raisebox{-0.5\height}{\includegraphics[width=0.32\columnwidth]{blockgroup-4104-full_shrunk.png}} & Missing \\
\raisebox{-0.5\height}{\includegraphics[width=0.32\columnwidth]{tract-4104-net_shrunk.png}} & \raisebox{-0.5\height}{\includegraphics[width=0.32\columnwidth]{blockgroup-4104-net_shrunk.png}} & Missing \\
\end{tabular}
\imtitle{Oregon 4}
\end{minipage}

\begin{minipage}{\columnwidth}
\begin{tabular}{lccc}
\raisebox{-0.5\height}{\includegraphics[width=0.32\columnwidth]{tract-4105-full_shrunk.png}} & \raisebox{-0.5\height}{\includegraphics[width=0.32\columnwidth]{blockgroup-4105-full_shrunk.png}} & Missing \\
\raisebox{-0.5\height}{\includegraphics[width=0.32\columnwidth]{tract-4105-net_shrunk.png}} & \raisebox{-0.5\height}{\includegraphics[width=0.32\columnwidth]{blockgroup-4105-net_shrunk.png}} & Missing \\
\end{tabular}
\imtitle{Oregon 5}
\end{minipage}
\bchap{Pennsylvania}
\begin{minipage}{\columnwidth}
\begin{tabular}{lccc}
\raisebox{-0.5\height}{\includegraphics[width=0.32\columnwidth]{tract-4201-full_shrunk.png}} & \raisebox{-0.5\height}{\includegraphics[width=0.32\columnwidth]{blockgroup-4201-full_shrunk.png}} & Missing \\
\raisebox{-0.5\height}{\includegraphics[width=0.32\columnwidth]{tract-4201-net_shrunk.png}} & \raisebox{-0.5\height}{\includegraphics[width=0.32\columnwidth]{blockgroup-4201-net_shrunk.png}} & Missing \\
\end{tabular}
\imtitle{Pennsylvania 1}
\end{minipage}

\begin{minipage}{\columnwidth}
\begin{tabular}{lccc}
\raisebox{-0.5\height}{\includegraphics[width=0.32\columnwidth]{tract-4202-full_shrunk.png}} & \raisebox{-0.5\height}{\includegraphics[width=0.32\columnwidth]{blockgroup-4202-full_shrunk.png}} & Missing \\
\raisebox{-0.5\height}{\includegraphics[width=0.32\columnwidth]{tract-4202-net_shrunk.png}} & \raisebox{-0.5\height}{\includegraphics[width=0.32\columnwidth]{blockgroup-4202-net_shrunk.png}} & Missing \\
\end{tabular}
\imtitle{Pennsylvania 2}
\end{minipage}

\begin{minipage}{\columnwidth}
\begin{tabular}{lccc}
\raisebox{-0.5\height}{\includegraphics[width=0.32\columnwidth]{tract-4203-full_shrunk.png}} & \raisebox{-0.5\height}{\includegraphics[width=0.32\columnwidth]{blockgroup-4203-full_shrunk.png}} & Missing \\
\raisebox{-0.5\height}{\includegraphics[width=0.32\columnwidth]{tract-4203-net_shrunk.png}} & \raisebox{-0.5\height}{\includegraphics[width=0.32\columnwidth]{blockgroup-4203-net_shrunk.png}} & Missing \\
\end{tabular}
\imtitle{Pennsylvania 3}
\end{minipage}

\begin{minipage}{\columnwidth}
\begin{tabular}{lccc}
\raisebox{-0.5\height}{\includegraphics[width=0.32\columnwidth]{tract-4204-full_shrunk.png}} & \raisebox{-0.5\height}{\includegraphics[width=0.32\columnwidth]{blockgroup-4204-full_shrunk.png}} & Missing \\
\raisebox{-0.5\height}{\includegraphics[width=0.32\columnwidth]{tract-4204-net_shrunk.png}} & \raisebox{-0.5\height}{\includegraphics[width=0.32\columnwidth]{blockgroup-4204-net_shrunk.png}} & Missing \\
\end{tabular}
\imtitle{Pennsylvania 4}
\end{minipage}

\begin{minipage}{\columnwidth}
\begin{tabular}{lccc}
\raisebox{-0.5\height}{\includegraphics[width=0.32\columnwidth]{tract-4205-full_shrunk.png}} & \raisebox{-0.5\height}{\includegraphics[width=0.32\columnwidth]{blockgroup-4205-full_shrunk.png}} & Missing \\
\raisebox{-0.5\height}{\includegraphics[width=0.32\columnwidth]{tract-4205-net_shrunk.png}} & \raisebox{-0.5\height}{\includegraphics[width=0.32\columnwidth]{blockgroup-4205-net_shrunk.png}} & Missing \\
\end{tabular}
\imtitle{Pennsylvania 5}
\end{minipage}

\begin{minipage}{\columnwidth}
\begin{tabular}{lccc}
\raisebox{-0.5\height}{\includegraphics[width=0.32\columnwidth]{tract-4206-full_shrunk.png}} & \raisebox{-0.5\height}{\includegraphics[width=0.32\columnwidth]{blockgroup-4206-full_shrunk.png}} & Missing \\
\raisebox{-0.5\height}{\includegraphics[width=0.32\columnwidth]{tract-4206-net_shrunk.png}} & \raisebox{-0.5\height}{\includegraphics[width=0.32\columnwidth]{blockgroup-4206-net_shrunk.png}} & Missing \\
\end{tabular}
\imtitle{Pennsylvania 6}
\end{minipage}

\begin{minipage}{\columnwidth}
\begin{tabular}{lccc}
\raisebox{-0.5\height}{\includegraphics[width=0.32\columnwidth]{tract-4207-full_shrunk.png}} & \raisebox{-0.5\height}{\includegraphics[width=0.32\columnwidth]{blockgroup-4207-full_shrunk.png}} & Missing \\
\raisebox{-0.5\height}{\includegraphics[width=0.32\columnwidth]{tract-4207-net_shrunk.png}} & \raisebox{-0.5\height}{\includegraphics[width=0.32\columnwidth]{blockgroup-4207-net_shrunk.png}} & Missing \\
\end{tabular}
\imtitle{Pennsylvania 7}
\end{minipage}

\begin{minipage}{\columnwidth}
\begin{tabular}{lccc}
\raisebox{-0.5\height}{\includegraphics[width=0.32\columnwidth]{tract-4208-full_shrunk.png}} & \raisebox{-0.5\height}{\includegraphics[width=0.32\columnwidth]{blockgroup-4208-full_shrunk.png}} & Missing \\
\raisebox{-0.5\height}{\includegraphics[width=0.32\columnwidth]{tract-4208-net_shrunk.png}} & \raisebox{-0.5\height}{\includegraphics[width=0.32\columnwidth]{blockgroup-4208-net_shrunk.png}} & Missing \\
\end{tabular}
\imtitle{Pennsylvania 8}
\end{minipage}

\begin{minipage}{\columnwidth}
\begin{tabular}{lccc}
\raisebox{-0.5\height}{\includegraphics[width=0.32\columnwidth]{tract-4209-full_shrunk.png}} & \raisebox{-0.5\height}{\includegraphics[width=0.32\columnwidth]{blockgroup-4209-full_shrunk.png}} & Missing \\
\raisebox{-0.5\height}{\includegraphics[width=0.32\columnwidth]{tract-4209-net_shrunk.png}} & \raisebox{-0.5\height}{\includegraphics[width=0.32\columnwidth]{blockgroup-4209-net_shrunk.png}} & Missing \\
\end{tabular}
\imtitle{Pennsylvania 9}
\end{minipage}

\begin{minipage}{\columnwidth}
\begin{tabular}{lccc}
\raisebox{-0.5\height}{\includegraphics[width=0.32\columnwidth]{tract-4210-full_shrunk.png}} & \raisebox{-0.5\height}{\includegraphics[width=0.32\columnwidth]{blockgroup-4210-full_shrunk.png}} & Missing \\
\raisebox{-0.5\height}{\includegraphics[width=0.32\columnwidth]{tract-4210-net_shrunk.png}} & \raisebox{-0.5\height}{\includegraphics[width=0.32\columnwidth]{blockgroup-4210-net_shrunk.png}} & Missing \\
\end{tabular}
\imtitle{Pennsylvania 10}
\end{minipage}

\begin{minipage}{\columnwidth}
\begin{tabular}{lccc}
\raisebox{-0.5\height}{\includegraphics[width=0.32\columnwidth]{tract-4211-full_shrunk.png}} & \raisebox{-0.5\height}{\includegraphics[width=0.32\columnwidth]{blockgroup-4211-full_shrunk.png}} & Missing \\
\raisebox{-0.5\height}{\includegraphics[width=0.32\columnwidth]{tract-4211-net_shrunk.png}} & \raisebox{-0.5\height}{\includegraphics[width=0.32\columnwidth]{blockgroup-4211-net_shrunk.png}} & Missing \\
\end{tabular}
\imtitle{Pennsylvania 11}
\end{minipage}

\begin{minipage}{\columnwidth}
\begin{tabular}{lccc}
\raisebox{-0.5\height}{\includegraphics[width=0.32\columnwidth]{tract-4212-full_shrunk.png}} & \raisebox{-0.5\height}{\includegraphics[width=0.32\columnwidth]{blockgroup-4212-full_shrunk.png}} & Missing \\
\raisebox{-0.5\height}{\includegraphics[width=0.32\columnwidth]{tract-4212-net_shrunk.png}} & \raisebox{-0.5\height}{\includegraphics[width=0.32\columnwidth]{blockgroup-4212-net_shrunk.png}} & Missing \\
\end{tabular}
\imtitle{Pennsylvania 12}
\end{minipage}

\begin{minipage}{\columnwidth}
\begin{tabular}{lccc}
\raisebox{-0.5\height}{\includegraphics[width=0.32\columnwidth]{tract-4213-full_shrunk.png}} & \raisebox{-0.5\height}{\includegraphics[width=0.32\columnwidth]{blockgroup-4213-full_shrunk.png}} & Missing \\
\raisebox{-0.5\height}{\includegraphics[width=0.32\columnwidth]{tract-4213-net_shrunk.png}} & \raisebox{-0.5\height}{\includegraphics[width=0.32\columnwidth]{blockgroup-4213-net_shrunk.png}} & Missing \\
\end{tabular}
\imtitle{Pennsylvania 13}
\end{minipage}

\begin{minipage}{\columnwidth}
\begin{tabular}{lccc}
\raisebox{-0.5\height}{\includegraphics[width=0.32\columnwidth]{tract-4214-full_shrunk.png}} & \raisebox{-0.5\height}{\includegraphics[width=0.32\columnwidth]{blockgroup-4214-full_shrunk.png}} & Missing \\
\raisebox{-0.5\height}{\includegraphics[width=0.32\columnwidth]{tract-4214-net_shrunk.png}} & \raisebox{-0.5\height}{\includegraphics[width=0.32\columnwidth]{blockgroup-4214-net_shrunk.png}} & Missing \\
\end{tabular}
\imtitle{Pennsylvania 14}
\end{minipage}

\begin{minipage}{\columnwidth}
\begin{tabular}{lccc}
\raisebox{-0.5\height}{\includegraphics[width=0.32\columnwidth]{tract-4215-full_shrunk.png}} & \raisebox{-0.5\height}{\includegraphics[width=0.32\columnwidth]{blockgroup-4215-full_shrunk.png}} & Missing \\
\raisebox{-0.5\height}{\includegraphics[width=0.32\columnwidth]{tract-4215-net_shrunk.png}} & \raisebox{-0.5\height}{\includegraphics[width=0.32\columnwidth]{blockgroup-4215-net_shrunk.png}} & Missing \\
\end{tabular}
\imtitle{Pennsylvania 15}
\end{minipage}

\begin{minipage}{\columnwidth}
\begin{tabular}{lccc}
\raisebox{-0.5\height}{\includegraphics[width=0.32\columnwidth]{tract-4216-full_shrunk.png}} & \raisebox{-0.5\height}{\includegraphics[width=0.32\columnwidth]{blockgroup-4216-full_shrunk.png}} & Missing \\
\raisebox{-0.5\height}{\includegraphics[width=0.32\columnwidth]{tract-4216-net_shrunk.png}} & \raisebox{-0.5\height}{\includegraphics[width=0.32\columnwidth]{blockgroup-4216-net_shrunk.png}} & Missing \\
\end{tabular}
\imtitle{Pennsylvania 16}
\end{minipage}

\begin{minipage}{\columnwidth}
\begin{tabular}{lccc}
\raisebox{-0.5\height}{\includegraphics[width=0.32\columnwidth]{tract-4217-full_shrunk.png}} & \raisebox{-0.5\height}{\includegraphics[width=0.32\columnwidth]{blockgroup-4217-full_shrunk.png}} & Missing \\
\raisebox{-0.5\height}{\includegraphics[width=0.32\columnwidth]{tract-4217-net_shrunk.png}} & \raisebox{-0.5\height}{\includegraphics[width=0.32\columnwidth]{blockgroup-4217-net_shrunk.png}} & Missing \\
\end{tabular}
\imtitle{Pennsylvania 17}
\end{minipage}

\begin{minipage}{\columnwidth}
\begin{tabular}{lccc}
\raisebox{-0.5\height}{\includegraphics[width=0.32\columnwidth]{tract-4218-full_shrunk.png}} & \raisebox{-0.5\height}{\includegraphics[width=0.32\columnwidth]{blockgroup-4218-full_shrunk.png}} & Missing \\
\raisebox{-0.5\height}{\includegraphics[width=0.32\columnwidth]{tract-4218-net_shrunk.png}} & \raisebox{-0.5\height}{\includegraphics[width=0.32\columnwidth]{blockgroup-4218-net_shrunk.png}} & Missing \\
\end{tabular}
\imtitle{Pennsylvania 18}
\end{minipage}
\bchap{Rhode Island}
\begin{minipage}{\columnwidth}
\begin{tabular}{lccc}
\raisebox{-0.5\height}{\includegraphics[width=0.32\columnwidth]{tract-4401-full_shrunk.png}} & \raisebox{-0.5\height}{\includegraphics[width=0.32\columnwidth]{blockgroup-4401-full_shrunk.png}} & Missing \\
\raisebox{-0.5\height}{\includegraphics[width=0.32\columnwidth]{tract-4401-net_shrunk.png}} & \raisebox{-0.5\height}{\includegraphics[width=0.32\columnwidth]{blockgroup-4401-net_shrunk.png}} & Missing \\
\end{tabular}
\imtitle{Rhode Island 1}
\end{minipage}

\begin{minipage}{\columnwidth}
\begin{tabular}{lccc}
\raisebox{-0.5\height}{\includegraphics[width=0.32\columnwidth]{tract-4402-full_shrunk.png}} & \raisebox{-0.5\height}{\includegraphics[width=0.32\columnwidth]{blockgroup-4402-full_shrunk.png}} & Missing \\
\raisebox{-0.5\height}{\includegraphics[width=0.32\columnwidth]{tract-4402-net_shrunk.png}} & \raisebox{-0.5\height}{\includegraphics[width=0.32\columnwidth]{blockgroup-4402-net_shrunk.png}} & Missing \\
\end{tabular}
\imtitle{Rhode Island 2}
\end{minipage}
\bchap{South Carolina}
\begin{minipage}{\columnwidth}
\begin{tabular}{lccc}
\raisebox{-0.5\height}{\includegraphics[width=0.32\columnwidth]{tract-4501-full_shrunk.png}} & \raisebox{-0.5\height}{\includegraphics[width=0.32\columnwidth]{blockgroup-4501-full_shrunk.png}} & Missing \\
\raisebox{-0.5\height}{\includegraphics[width=0.32\columnwidth]{tract-4501-net_shrunk.png}} & \raisebox{-0.5\height}{\includegraphics[width=0.32\columnwidth]{blockgroup-4501-net_shrunk.png}} & Missing \\
\end{tabular}
\imtitle{South Carolina 1}
\end{minipage}

\begin{minipage}{\columnwidth}
\begin{tabular}{lccc}
\raisebox{-0.5\height}{\includegraphics[width=0.32\columnwidth]{tract-4502-full_shrunk.png}} & \raisebox{-0.5\height}{\includegraphics[width=0.32\columnwidth]{blockgroup-4502-full_shrunk.png}} & Missing \\
\raisebox{-0.5\height}{\includegraphics[width=0.32\columnwidth]{tract-4502-net_shrunk.png}} & \raisebox{-0.5\height}{\includegraphics[width=0.32\columnwidth]{blockgroup-4502-net_shrunk.png}} & Missing \\
\end{tabular}
\imtitle{South Carolina 2}
\end{minipage}

\begin{minipage}{\columnwidth}
\begin{tabular}{lccc}
\raisebox{-0.5\height}{\includegraphics[width=0.32\columnwidth]{tract-4503-full_shrunk.png}} & \raisebox{-0.5\height}{\includegraphics[width=0.32\columnwidth]{blockgroup-4503-full_shrunk.png}} & Missing \\
\raisebox{-0.5\height}{\includegraphics[width=0.32\columnwidth]{tract-4503-net_shrunk.png}} & \raisebox{-0.5\height}{\includegraphics[width=0.32\columnwidth]{blockgroup-4503-net_shrunk.png}} & Missing \\
\end{tabular}
\imtitle{South Carolina 3}
\end{minipage}

\begin{minipage}{\columnwidth}
\begin{tabular}{lccc}
\raisebox{-0.5\height}{\includegraphics[width=0.32\columnwidth]{tract-4504-full_shrunk.png}} & \raisebox{-0.5\height}{\includegraphics[width=0.32\columnwidth]{blockgroup-4504-full_shrunk.png}} & Missing \\
\raisebox{-0.5\height}{\includegraphics[width=0.32\columnwidth]{tract-4504-net_shrunk.png}} & \raisebox{-0.5\height}{\includegraphics[width=0.32\columnwidth]{blockgroup-4504-net_shrunk.png}} & Missing \\
\end{tabular}
\imtitle{South Carolina 4}
\end{minipage}

\begin{minipage}{\columnwidth}
\begin{tabular}{lccc}
\raisebox{-0.5\height}{\includegraphics[width=0.32\columnwidth]{tract-4505-full_shrunk.png}} & \raisebox{-0.5\height}{\includegraphics[width=0.32\columnwidth]{blockgroup-4505-full_shrunk.png}} & Missing \\
\raisebox{-0.5\height}{\includegraphics[width=0.32\columnwidth]{tract-4505-net_shrunk.png}} & \raisebox{-0.5\height}{\includegraphics[width=0.32\columnwidth]{blockgroup-4505-net_shrunk.png}} & Missing \\
\end{tabular}
\imtitle{South Carolina 5}
\end{minipage}

\begin{minipage}{\columnwidth}
\begin{tabular}{lccc}
\raisebox{-0.5\height}{\includegraphics[width=0.32\columnwidth]{tract-4506-full_shrunk.png}} & \raisebox{-0.5\height}{\includegraphics[width=0.32\columnwidth]{blockgroup-4506-full_shrunk.png}} & Missing \\
\raisebox{-0.5\height}{\includegraphics[width=0.32\columnwidth]{tract-4506-net_shrunk.png}} & \raisebox{-0.5\height}{\includegraphics[width=0.32\columnwidth]{blockgroup-4506-net_shrunk.png}} & Missing \\
\end{tabular}
\imtitle{South Carolina 6}
\end{minipage}

\begin{minipage}{\columnwidth}
\begin{tabular}{lccc}
Missing & Missing & Missing \\
Missing & Missing & Missing \\
\end{tabular}
\imtitle{South Carolina 7}
\end{minipage}
\bchap{South Dakota}
\begin{minipage}{\columnwidth}
\begin{tabular}{lccc}
\raisebox{-0.5\height}{\includegraphics[width=0.32\columnwidth]{tract-4600-full_shrunk.png}} & \raisebox{-0.5\height}{\includegraphics[width=0.32\columnwidth]{blockgroup-4600-full_shrunk.png}} & Missing \\
\raisebox{-0.5\height}{\includegraphics[width=0.32\columnwidth]{tract-4600-net_shrunk.png}} & \raisebox{-0.5\height}{\includegraphics[width=0.32\columnwidth]{blockgroup-4600-net_shrunk.png}} & Missing \\
\end{tabular}
\imtitle{South Dakota 0}
\end{minipage}
\bchap{Tennessee}
\begin{minipage}{\columnwidth}
\begin{tabular}{lccc}
\raisebox{-0.5\height}{\includegraphics[width=0.32\columnwidth]{tract-4701-full_shrunk.png}} & \raisebox{-0.5\height}{\includegraphics[width=0.32\columnwidth]{blockgroup-4701-full_shrunk.png}} & Missing \\
\raisebox{-0.5\height}{\includegraphics[width=0.32\columnwidth]{tract-4701-net_shrunk.png}} & \raisebox{-0.5\height}{\includegraphics[width=0.32\columnwidth]{blockgroup-4701-net_shrunk.png}} & Missing \\
\end{tabular}
\imtitle{Tennessee 1}
\end{minipage}

\begin{minipage}{\columnwidth}
\begin{tabular}{lccc}
\raisebox{-0.5\height}{\includegraphics[width=0.32\columnwidth]{tract-4702-full_shrunk.png}} & \raisebox{-0.5\height}{\includegraphics[width=0.32\columnwidth]{blockgroup-4702-full_shrunk.png}} & Missing \\
\raisebox{-0.5\height}{\includegraphics[width=0.32\columnwidth]{tract-4702-net_shrunk.png}} & \raisebox{-0.5\height}{\includegraphics[width=0.32\columnwidth]{blockgroup-4702-net_shrunk.png}} & Missing \\
\end{tabular}
\imtitle{Tennessee 2}
\end{minipage}

\begin{minipage}{\columnwidth}
\begin{tabular}{lccc}
\raisebox{-0.5\height}{\includegraphics[width=0.32\columnwidth]{tract-4703-full_shrunk.png}} & \raisebox{-0.5\height}{\includegraphics[width=0.32\columnwidth]{blockgroup-4703-full_shrunk.png}} & Missing \\
\raisebox{-0.5\height}{\includegraphics[width=0.32\columnwidth]{tract-4703-net_shrunk.png}} & \raisebox{-0.5\height}{\includegraphics[width=0.32\columnwidth]{blockgroup-4703-net_shrunk.png}} & Missing \\
\end{tabular}
\imtitle{Tennessee 3}
\end{minipage}

\begin{minipage}{\columnwidth}
\begin{tabular}{lccc}
\raisebox{-0.5\height}{\includegraphics[width=0.32\columnwidth]{tract-4704-full_shrunk.png}} & \raisebox{-0.5\height}{\includegraphics[width=0.32\columnwidth]{blockgroup-4704-full_shrunk.png}} & Missing \\
\raisebox{-0.5\height}{\includegraphics[width=0.32\columnwidth]{tract-4704-net_shrunk.png}} & \raisebox{-0.5\height}{\includegraphics[width=0.32\columnwidth]{blockgroup-4704-net_shrunk.png}} & Missing \\
\end{tabular}
\imtitle{Tennessee 4}
\end{minipage}

\begin{minipage}{\columnwidth}
\begin{tabular}{lccc}
\raisebox{-0.5\height}{\includegraphics[width=0.32\columnwidth]{tract-4705-full_shrunk.png}} & \raisebox{-0.5\height}{\includegraphics[width=0.32\columnwidth]{blockgroup-4705-full_shrunk.png}} & Missing \\
\raisebox{-0.5\height}{\includegraphics[width=0.32\columnwidth]{tract-4705-net_shrunk.png}} & \raisebox{-0.5\height}{\includegraphics[width=0.32\columnwidth]{blockgroup-4705-net_shrunk.png}} & Missing \\
\end{tabular}
\imtitle{Tennessee 5}
\end{minipage}

\begin{minipage}{\columnwidth}
\begin{tabular}{lccc}
\raisebox{-0.5\height}{\includegraphics[width=0.32\columnwidth]{tract-4706-full_shrunk.png}} & \raisebox{-0.5\height}{\includegraphics[width=0.32\columnwidth]{blockgroup-4706-full_shrunk.png}} & Missing \\
\raisebox{-0.5\height}{\includegraphics[width=0.32\columnwidth]{tract-4706-net_shrunk.png}} & \raisebox{-0.5\height}{\includegraphics[width=0.32\columnwidth]{blockgroup-4706-net_shrunk.png}} & Missing \\
\end{tabular}
\imtitle{Tennessee 6}
\end{minipage}

\begin{minipage}{\columnwidth}
\begin{tabular}{lccc}
\raisebox{-0.5\height}{\includegraphics[width=0.32\columnwidth]{tract-4707-full_shrunk.png}} & \raisebox{-0.5\height}{\includegraphics[width=0.32\columnwidth]{blockgroup-4707-full_shrunk.png}} & Missing \\
\raisebox{-0.5\height}{\includegraphics[width=0.32\columnwidth]{tract-4707-net_shrunk.png}} & \raisebox{-0.5\height}{\includegraphics[width=0.32\columnwidth]{blockgroup-4707-net_shrunk.png}} & Missing \\
\end{tabular}
\imtitle{Tennessee 7}
\end{minipage}

\begin{minipage}{\columnwidth}
\begin{tabular}{lccc}
\raisebox{-0.5\height}{\includegraphics[width=0.32\columnwidth]{tract-4708-full_shrunk.png}} & \raisebox{-0.5\height}{\includegraphics[width=0.32\columnwidth]{blockgroup-4708-full_shrunk.png}} & Missing \\
\raisebox{-0.5\height}{\includegraphics[width=0.32\columnwidth]{tract-4708-net_shrunk.png}} & \raisebox{-0.5\height}{\includegraphics[width=0.32\columnwidth]{blockgroup-4708-net_shrunk.png}} & Missing \\
\end{tabular}
\imtitle{Tennessee 8}
\end{minipage}

\begin{minipage}{\columnwidth}
\begin{tabular}{lccc}
\raisebox{-0.5\height}{\includegraphics[width=0.32\columnwidth]{tract-4709-full_shrunk.png}} & \raisebox{-0.5\height}{\includegraphics[width=0.32\columnwidth]{blockgroup-4709-full_shrunk.png}} & Missing \\
\raisebox{-0.5\height}{\includegraphics[width=0.32\columnwidth]{tract-4709-net_shrunk.png}} & \raisebox{-0.5\height}{\includegraphics[width=0.32\columnwidth]{blockgroup-4709-net_shrunk.png}} & Missing \\
\end{tabular}
\imtitle{Tennessee 9}
\end{minipage}
\bchap{Texas}
\begin{minipage}{\columnwidth}
\begin{tabular}{lccc}
\raisebox{-0.5\height}{\includegraphics[width=0.32\columnwidth]{tract-4801-full_shrunk.png}} & \raisebox{-0.5\height}{\includegraphics[width=0.32\columnwidth]{blockgroup-4801-full_shrunk.png}} & Missing \\
\raisebox{-0.5\height}{\includegraphics[width=0.32\columnwidth]{tract-4801-net_shrunk.png}} & \raisebox{-0.5\height}{\includegraphics[width=0.32\columnwidth]{blockgroup-4801-net_shrunk.png}} & Missing \\
\end{tabular}
\imtitle{Texas 1}
\end{minipage}

\begin{minipage}{\columnwidth}
\begin{tabular}{lccc}
\raisebox{-0.5\height}{\includegraphics[width=0.32\columnwidth]{tract-4802-full_shrunk.png}} & \raisebox{-0.5\height}{\includegraphics[width=0.32\columnwidth]{blockgroup-4802-full_shrunk.png}} & Missing \\
\raisebox{-0.5\height}{\includegraphics[width=0.32\columnwidth]{tract-4802-net_shrunk.png}} & \raisebox{-0.5\height}{\includegraphics[width=0.32\columnwidth]{blockgroup-4802-net_shrunk.png}} & Missing \\
\end{tabular}
\imtitle{Texas 2}
\end{minipage}

\begin{minipage}{\columnwidth}
\begin{tabular}{lccc}
\raisebox{-0.5\height}{\includegraphics[width=0.32\columnwidth]{tract-4803-full_shrunk.png}} & \raisebox{-0.5\height}{\includegraphics[width=0.32\columnwidth]{blockgroup-4803-full_shrunk.png}} & Missing \\
\raisebox{-0.5\height}{\includegraphics[width=0.32\columnwidth]{tract-4803-net_shrunk.png}} & \raisebox{-0.5\height}{\includegraphics[width=0.32\columnwidth]{blockgroup-4803-net_shrunk.png}} & Missing \\
\end{tabular}
\imtitle{Texas 3}
\end{minipage}

\begin{minipage}{\columnwidth}
\begin{tabular}{lccc}
\raisebox{-0.5\height}{\includegraphics[width=0.32\columnwidth]{tract-4804-full_shrunk.png}} & \raisebox{-0.5\height}{\includegraphics[width=0.32\columnwidth]{blockgroup-4804-full_shrunk.png}} & Missing \\
\raisebox{-0.5\height}{\includegraphics[width=0.32\columnwidth]{tract-4804-net_shrunk.png}} & \raisebox{-0.5\height}{\includegraphics[width=0.32\columnwidth]{blockgroup-4804-net_shrunk.png}} & Missing \\
\end{tabular}
\imtitle{Texas 4}
\end{minipage}

\begin{minipage}{\columnwidth}
\begin{tabular}{lccc}
\raisebox{-0.5\height}{\includegraphics[width=0.32\columnwidth]{tract-4805-full_shrunk.png}} & \raisebox{-0.5\height}{\includegraphics[width=0.32\columnwidth]{blockgroup-4805-full_shrunk.png}} & Missing \\
\raisebox{-0.5\height}{\includegraphics[width=0.32\columnwidth]{tract-4805-net_shrunk.png}} & \raisebox{-0.5\height}{\includegraphics[width=0.32\columnwidth]{blockgroup-4805-net_shrunk.png}} & Missing \\
\end{tabular}
\imtitle{Texas 5}
\end{minipage}

\begin{minipage}{\columnwidth}
\begin{tabular}{lccc}
\raisebox{-0.5\height}{\includegraphics[width=0.32\columnwidth]{tract-4806-full_shrunk.png}} & \raisebox{-0.5\height}{\includegraphics[width=0.32\columnwidth]{blockgroup-4806-full_shrunk.png}} & Missing \\
\raisebox{-0.5\height}{\includegraphics[width=0.32\columnwidth]{tract-4806-net_shrunk.png}} & \raisebox{-0.5\height}{\includegraphics[width=0.32\columnwidth]{blockgroup-4806-net_shrunk.png}} & Missing \\
\end{tabular}
\imtitle{Texas 6}
\end{minipage}

\begin{minipage}{\columnwidth}
\begin{tabular}{lccc}
\raisebox{-0.5\height}{\includegraphics[width=0.32\columnwidth]{tract-4807-full_shrunk.png}} & \raisebox{-0.5\height}{\includegraphics[width=0.32\columnwidth]{blockgroup-4807-full_shrunk.png}} & Missing \\
\raisebox{-0.5\height}{\includegraphics[width=0.32\columnwidth]{tract-4807-net_shrunk.png}} & \raisebox{-0.5\height}{\includegraphics[width=0.32\columnwidth]{blockgroup-4807-net_shrunk.png}} & Missing \\
\end{tabular}
\imtitle{Texas 7}
\end{minipage}

\begin{minipage}{\columnwidth}
\begin{tabular}{lccc}
\raisebox{-0.5\height}{\includegraphics[width=0.32\columnwidth]{tract-4808-full_shrunk.png}} & \raisebox{-0.5\height}{\includegraphics[width=0.32\columnwidth]{blockgroup-4808-full_shrunk.png}} & Missing \\
\raisebox{-0.5\height}{\includegraphics[width=0.32\columnwidth]{tract-4808-net_shrunk.png}} & \raisebox{-0.5\height}{\includegraphics[width=0.32\columnwidth]{blockgroup-4808-net_shrunk.png}} & Missing \\
\end{tabular}
\imtitle{Texas 8}
\end{minipage}

\begin{minipage}{\columnwidth}
\begin{tabular}{lccc}
\raisebox{-0.5\height}{\includegraphics[width=0.32\columnwidth]{tract-4809-full_shrunk.png}} & \raisebox{-0.5\height}{\includegraphics[width=0.32\columnwidth]{blockgroup-4809-full_shrunk.png}} & Missing \\
\raisebox{-0.5\height}{\includegraphics[width=0.32\columnwidth]{tract-4809-net_shrunk.png}} & \raisebox{-0.5\height}{\includegraphics[width=0.32\columnwidth]{blockgroup-4809-net_shrunk.png}} & Missing \\
\end{tabular}
\imtitle{Texas 9}
\end{minipage}

\begin{minipage}{\columnwidth}
\begin{tabular}{lccc}
\raisebox{-0.5\height}{\includegraphics[width=0.32\columnwidth]{tract-4810-full_shrunk.png}} & \raisebox{-0.5\height}{\includegraphics[width=0.32\columnwidth]{blockgroup-4810-full_shrunk.png}} & Missing \\
\raisebox{-0.5\height}{\includegraphics[width=0.32\columnwidth]{tract-4810-net_shrunk.png}} & \raisebox{-0.5\height}{\includegraphics[width=0.32\columnwidth]{blockgroup-4810-net_shrunk.png}} & Missing \\
\end{tabular}
\imtitle{Texas 10}
\end{minipage}

\begin{minipage}{\columnwidth}
\begin{tabular}{lccc}
\raisebox{-0.5\height}{\includegraphics[width=0.32\columnwidth]{tract-4811-full_shrunk.png}} & \raisebox{-0.5\height}{\includegraphics[width=0.32\columnwidth]{blockgroup-4811-full_shrunk.png}} & Missing \\
\raisebox{-0.5\height}{\includegraphics[width=0.32\columnwidth]{tract-4811-net_shrunk.png}} & \raisebox{-0.5\height}{\includegraphics[width=0.32\columnwidth]{blockgroup-4811-net_shrunk.png}} & Missing \\
\end{tabular}
\imtitle{Texas 11}
\end{minipage}

\begin{minipage}{\columnwidth}
\begin{tabular}{lccc}
\raisebox{-0.5\height}{\includegraphics[width=0.32\columnwidth]{tract-4812-full_shrunk.png}} & \raisebox{-0.5\height}{\includegraphics[width=0.32\columnwidth]{blockgroup-4812-full_shrunk.png}} & Missing \\
\raisebox{-0.5\height}{\includegraphics[width=0.32\columnwidth]{tract-4812-net_shrunk.png}} & \raisebox{-0.5\height}{\includegraphics[width=0.32\columnwidth]{blockgroup-4812-net_shrunk.png}} & Missing \\
\end{tabular}
\imtitle{Texas 12}
\end{minipage}

\begin{minipage}{\columnwidth}
\begin{tabular}{lccc}
\raisebox{-0.5\height}{\includegraphics[width=0.32\columnwidth]{tract-4813-full_shrunk.png}} & \raisebox{-0.5\height}{\includegraphics[width=0.32\columnwidth]{blockgroup-4813-full_shrunk.png}} & Missing \\
\raisebox{-0.5\height}{\includegraphics[width=0.32\columnwidth]{tract-4813-net_shrunk.png}} & \raisebox{-0.5\height}{\includegraphics[width=0.32\columnwidth]{blockgroup-4813-net_shrunk.png}} & Missing \\
\end{tabular}
\imtitle{Texas 13}
\end{minipage}

\begin{minipage}{\columnwidth}
\begin{tabular}{lccc}
\raisebox{-0.5\height}{\includegraphics[width=0.32\columnwidth]{tract-4814-full_shrunk.png}} & \raisebox{-0.5\height}{\includegraphics[width=0.32\columnwidth]{blockgroup-4814-full_shrunk.png}} & Missing \\
\raisebox{-0.5\height}{\includegraphics[width=0.32\columnwidth]{tract-4814-net_shrunk.png}} & \raisebox{-0.5\height}{\includegraphics[width=0.32\columnwidth]{blockgroup-4814-net_shrunk.png}} & Missing \\
\end{tabular}
\imtitle{Texas 14}
\end{minipage}

\begin{minipage}{\columnwidth}
\begin{tabular}{lccc}
\raisebox{-0.5\height}{\includegraphics[width=0.32\columnwidth]{tract-4815-full_shrunk.png}} & \raisebox{-0.5\height}{\includegraphics[width=0.32\columnwidth]{blockgroup-4815-full_shrunk.png}} & Missing \\
\raisebox{-0.5\height}{\includegraphics[width=0.32\columnwidth]{tract-4815-net_shrunk.png}} & \raisebox{-0.5\height}{\includegraphics[width=0.32\columnwidth]{blockgroup-4815-net_shrunk.png}} & Missing \\
\end{tabular}
\imtitle{Texas 15}
\end{minipage}

\begin{minipage}{\columnwidth}
\begin{tabular}{lccc}
\raisebox{-0.5\height}{\includegraphics[width=0.32\columnwidth]{tract-4816-full_shrunk.png}} & \raisebox{-0.5\height}{\includegraphics[width=0.32\columnwidth]{blockgroup-4816-full_shrunk.png}} & Missing \\
\raisebox{-0.5\height}{\includegraphics[width=0.32\columnwidth]{tract-4816-net_shrunk.png}} & \raisebox{-0.5\height}{\includegraphics[width=0.32\columnwidth]{blockgroup-4816-net_shrunk.png}} & Missing \\
\end{tabular}
\imtitle{Texas 16}
\end{minipage}

\begin{minipage}{\columnwidth}
\begin{tabular}{lccc}
\raisebox{-0.5\height}{\includegraphics[width=0.32\columnwidth]{tract-4817-full_shrunk.png}} & \raisebox{-0.5\height}{\includegraphics[width=0.32\columnwidth]{blockgroup-4817-full_shrunk.png}} & Missing \\
\raisebox{-0.5\height}{\includegraphics[width=0.32\columnwidth]{tract-4817-net_shrunk.png}} & \raisebox{-0.5\height}{\includegraphics[width=0.32\columnwidth]{blockgroup-4817-net_shrunk.png}} & Missing \\
\end{tabular}
\imtitle{Texas 17}
\end{minipage}

\begin{minipage}{\columnwidth}
\begin{tabular}{lccc}
\raisebox{-0.5\height}{\includegraphics[width=0.32\columnwidth]{tract-4818-full_shrunk.png}} & \raisebox{-0.5\height}{\includegraphics[width=0.32\columnwidth]{blockgroup-4818-full_shrunk.png}} & Missing \\
\raisebox{-0.5\height}{\includegraphics[width=0.32\columnwidth]{tract-4818-net_shrunk.png}} & \raisebox{-0.5\height}{\includegraphics[width=0.32\columnwidth]{blockgroup-4818-net_shrunk.png}} & Missing \\
\end{tabular}
\imtitle{Texas 18}
\end{minipage}

\begin{minipage}{\columnwidth}
\begin{tabular}{lccc}
\raisebox{-0.5\height}{\includegraphics[width=0.32\columnwidth]{tract-4819-full_shrunk.png}} & \raisebox{-0.5\height}{\includegraphics[width=0.32\columnwidth]{blockgroup-4819-full_shrunk.png}} & Missing \\
\raisebox{-0.5\height}{\includegraphics[width=0.32\columnwidth]{tract-4819-net_shrunk.png}} & \raisebox{-0.5\height}{\includegraphics[width=0.32\columnwidth]{blockgroup-4819-net_shrunk.png}} & Missing \\
\end{tabular}
\imtitle{Texas 19}
\end{minipage}

\begin{minipage}{\columnwidth}
\begin{tabular}{lccc}
\raisebox{-0.5\height}{\includegraphics[width=0.32\columnwidth]{tract-4820-full_shrunk.png}} & \raisebox{-0.5\height}{\includegraphics[width=0.32\columnwidth]{blockgroup-4820-full_shrunk.png}} & Missing \\
\raisebox{-0.5\height}{\includegraphics[width=0.32\columnwidth]{tract-4820-net_shrunk.png}} & \raisebox{-0.5\height}{\includegraphics[width=0.32\columnwidth]{blockgroup-4820-net_shrunk.png}} & Missing \\
\end{tabular}
\imtitle{Texas 20}
\end{minipage}

\begin{minipage}{\columnwidth}
\begin{tabular}{lccc}
\raisebox{-0.5\height}{\includegraphics[width=0.32\columnwidth]{tract-4821-full_shrunk.png}} & \raisebox{-0.5\height}{\includegraphics[width=0.32\columnwidth]{blockgroup-4821-full_shrunk.png}} & Missing \\
\raisebox{-0.5\height}{\includegraphics[width=0.32\columnwidth]{tract-4821-net_shrunk.png}} & \raisebox{-0.5\height}{\includegraphics[width=0.32\columnwidth]{blockgroup-4821-net_shrunk.png}} & Missing \\
\end{tabular}
\imtitle{Texas 21}
\end{minipage}

\begin{minipage}{\columnwidth}
\begin{tabular}{lccc}
\raisebox{-0.5\height}{\includegraphics[width=0.32\columnwidth]{tract-4822-full_shrunk.png}} & \raisebox{-0.5\height}{\includegraphics[width=0.32\columnwidth]{blockgroup-4822-full_shrunk.png}} & Missing \\
\raisebox{-0.5\height}{\includegraphics[width=0.32\columnwidth]{tract-4822-net_shrunk.png}} & \raisebox{-0.5\height}{\includegraphics[width=0.32\columnwidth]{blockgroup-4822-net_shrunk.png}} & Missing \\
\end{tabular}
\imtitle{Texas 22}
\end{minipage}

\begin{minipage}{\columnwidth}
\begin{tabular}{lccc}
\raisebox{-0.5\height}{\includegraphics[width=0.32\columnwidth]{tract-4823-full_shrunk.png}} & \raisebox{-0.5\height}{\includegraphics[width=0.32\columnwidth]{blockgroup-4823-full_shrunk.png}} & Missing \\
\raisebox{-0.5\height}{\includegraphics[width=0.32\columnwidth]{tract-4823-net_shrunk.png}} & \raisebox{-0.5\height}{\includegraphics[width=0.32\columnwidth]{blockgroup-4823-net_shrunk.png}} & Missing \\
\end{tabular}
\imtitle{Texas 23}
\end{minipage}

\begin{minipage}{\columnwidth}
\begin{tabular}{lccc}
\raisebox{-0.5\height}{\includegraphics[width=0.32\columnwidth]{tract-4824-full_shrunk.png}} & \raisebox{-0.5\height}{\includegraphics[width=0.32\columnwidth]{blockgroup-4824-full_shrunk.png}} & Missing \\
\raisebox{-0.5\height}{\includegraphics[width=0.32\columnwidth]{tract-4824-net_shrunk.png}} & \raisebox{-0.5\height}{\includegraphics[width=0.32\columnwidth]{blockgroup-4824-net_shrunk.png}} & Missing \\
\end{tabular}
\imtitle{Texas 24}
\end{minipage}

\begin{minipage}{\columnwidth}
\begin{tabular}{lccc}
\raisebox{-0.5\height}{\includegraphics[width=0.32\columnwidth]{tract-4825-full_shrunk.png}} & \raisebox{-0.5\height}{\includegraphics[width=0.32\columnwidth]{blockgroup-4825-full_shrunk.png}} & Missing \\
\raisebox{-0.5\height}{\includegraphics[width=0.32\columnwidth]{tract-4825-net_shrunk.png}} & \raisebox{-0.5\height}{\includegraphics[width=0.32\columnwidth]{blockgroup-4825-net_shrunk.png}} & Missing \\
\end{tabular}
\imtitle{Texas 25}
\end{minipage}

\begin{minipage}{\columnwidth}
\begin{tabular}{lccc}
\raisebox{-0.5\height}{\includegraphics[width=0.32\columnwidth]{tract-4826-full_shrunk.png}} & \raisebox{-0.5\height}{\includegraphics[width=0.32\columnwidth]{blockgroup-4826-full_shrunk.png}} & Missing \\
\raisebox{-0.5\height}{\includegraphics[width=0.32\columnwidth]{tract-4826-net_shrunk.png}} & \raisebox{-0.5\height}{\includegraphics[width=0.32\columnwidth]{blockgroup-4826-net_shrunk.png}} & Missing \\
\end{tabular}
\imtitle{Texas 26}
\end{minipage}

\begin{minipage}{\columnwidth}
\begin{tabular}{lccc}
\raisebox{-0.5\height}{\includegraphics[width=0.32\columnwidth]{tract-4827-full_shrunk.png}} & \raisebox{-0.5\height}{\includegraphics[width=0.32\columnwidth]{blockgroup-4827-full_shrunk.png}} & Missing \\
\raisebox{-0.5\height}{\includegraphics[width=0.32\columnwidth]{tract-4827-net_shrunk.png}} & \raisebox{-0.5\height}{\includegraphics[width=0.32\columnwidth]{blockgroup-4827-net_shrunk.png}} & Missing \\
\end{tabular}
\imtitle{Texas 27}
\end{minipage}

\begin{minipage}{\columnwidth}
\begin{tabular}{lccc}
\raisebox{-0.5\height}{\includegraphics[width=0.32\columnwidth]{tract-4828-full_shrunk.png}} & \raisebox{-0.5\height}{\includegraphics[width=0.32\columnwidth]{blockgroup-4828-full_shrunk.png}} & Missing \\
\raisebox{-0.5\height}{\includegraphics[width=0.32\columnwidth]{tract-4828-net_shrunk.png}} & \raisebox{-0.5\height}{\includegraphics[width=0.32\columnwidth]{blockgroup-4828-net_shrunk.png}} & Missing \\
\end{tabular}
\imtitle{Texas 28}
\end{minipage}

\begin{minipage}{\columnwidth}
\begin{tabular}{lccc}
\raisebox{-0.5\height}{\includegraphics[width=0.32\columnwidth]{tract-4829-full_shrunk.png}} & \raisebox{-0.5\height}{\includegraphics[width=0.32\columnwidth]{blockgroup-4829-full_shrunk.png}} & Missing \\
\raisebox{-0.5\height}{\includegraphics[width=0.32\columnwidth]{tract-4829-net_shrunk.png}} & \raisebox{-0.5\height}{\includegraphics[width=0.32\columnwidth]{blockgroup-4829-net_shrunk.png}} & Missing \\
\end{tabular}
\imtitle{Texas 29}
\end{minipage}

\begin{minipage}{\columnwidth}
\begin{tabular}{lccc}
\raisebox{-0.5\height}{\includegraphics[width=0.32\columnwidth]{tract-4830-full_shrunk.png}} & \raisebox{-0.5\height}{\includegraphics[width=0.32\columnwidth]{blockgroup-4830-full_shrunk.png}} & Missing \\
\raisebox{-0.5\height}{\includegraphics[width=0.32\columnwidth]{tract-4830-net_shrunk.png}} & \raisebox{-0.5\height}{\includegraphics[width=0.32\columnwidth]{blockgroup-4830-net_shrunk.png}} & Missing \\
\end{tabular}
\imtitle{Texas 30}
\end{minipage}

\begin{minipage}{\columnwidth}
\begin{tabular}{lccc}
\raisebox{-0.5\height}{\includegraphics[width=0.32\columnwidth]{tract-4831-full_shrunk.png}} & \raisebox{-0.5\height}{\includegraphics[width=0.32\columnwidth]{blockgroup-4831-full_shrunk.png}} & Missing \\
\raisebox{-0.5\height}{\includegraphics[width=0.32\columnwidth]{tract-4831-net_shrunk.png}} & \raisebox{-0.5\height}{\includegraphics[width=0.32\columnwidth]{blockgroup-4831-net_shrunk.png}} & Missing \\
\end{tabular}
\imtitle{Texas 31}
\end{minipage}

\begin{minipage}{\columnwidth}
\begin{tabular}{lccc}
\raisebox{-0.5\height}{\includegraphics[width=0.32\columnwidth]{tract-4832-full_shrunk.png}} & \raisebox{-0.5\height}{\includegraphics[width=0.32\columnwidth]{blockgroup-4832-full_shrunk.png}} & Missing \\
\raisebox{-0.5\height}{\includegraphics[width=0.32\columnwidth]{tract-4832-net_shrunk.png}} & \raisebox{-0.5\height}{\includegraphics[width=0.32\columnwidth]{blockgroup-4832-net_shrunk.png}} & Missing \\
\end{tabular}
\imtitle{Texas 32}
\end{minipage}

\begin{minipage}{\columnwidth}
\begin{tabular}{lccc}
Missing & Missing & Missing \\
Missing & Missing & Missing \\
\end{tabular}
\imtitle{Texas 33}
\end{minipage}

\begin{minipage}{\columnwidth}
\begin{tabular}{lccc}
Missing & Missing & Missing \\
Missing & Missing & Missing \\
\end{tabular}
\imtitle{Texas 34}
\end{minipage}

\begin{minipage}{\columnwidth}
\begin{tabular}{lccc}
Missing & Missing & Missing \\
Missing & Missing & Missing \\
\end{tabular}
\imtitle{Texas 35}
\end{minipage}

\begin{minipage}{\columnwidth}
\begin{tabular}{lccc}
Missing & Missing & Missing \\
Missing & Missing & Missing \\
\end{tabular}
\imtitle{Texas 36}
\end{minipage}
\bchap{Utah}
\begin{minipage}{\columnwidth}
\begin{tabular}{lccc}
\raisebox{-0.5\height}{\includegraphics[width=0.32\columnwidth]{tract-4901-full_shrunk.png}} & \raisebox{-0.5\height}{\includegraphics[width=0.32\columnwidth]{blockgroup-4901-full_shrunk.png}} & Missing \\
\raisebox{-0.5\height}{\includegraphics[width=0.32\columnwidth]{tract-4901-net_shrunk.png}} & \raisebox{-0.5\height}{\includegraphics[width=0.32\columnwidth]{blockgroup-4901-net_shrunk.png}} & Missing \\
\end{tabular}
\imtitle{Utah 1}
\end{minipage}

\begin{minipage}{\columnwidth}
\begin{tabular}{lccc}
\raisebox{-0.5\height}{\includegraphics[width=0.32\columnwidth]{tract-4902-full_shrunk.png}} & \raisebox{-0.5\height}{\includegraphics[width=0.32\columnwidth]{blockgroup-4902-full_shrunk.png}} & Missing \\
\raisebox{-0.5\height}{\includegraphics[width=0.32\columnwidth]{tract-4902-net_shrunk.png}} & \raisebox{-0.5\height}{\includegraphics[width=0.32\columnwidth]{blockgroup-4902-net_shrunk.png}} & Missing \\
\end{tabular}
\imtitle{Utah 2}
\end{minipage}

\begin{minipage}{\columnwidth}
\begin{tabular}{lccc}
\raisebox{-0.5\height}{\includegraphics[width=0.32\columnwidth]{tract-4903-full_shrunk.png}} & \raisebox{-0.5\height}{\includegraphics[width=0.32\columnwidth]{blockgroup-4903-full_shrunk.png}} & Missing \\
\raisebox{-0.5\height}{\includegraphics[width=0.32\columnwidth]{tract-4903-net_shrunk.png}} & \raisebox{-0.5\height}{\includegraphics[width=0.32\columnwidth]{blockgroup-4903-net_shrunk.png}} & Missing \\
\end{tabular}
\imtitle{Utah 3}
\end{minipage}

\begin{minipage}{\columnwidth}
\begin{tabular}{lccc}
Missing & Missing & Missing \\
Missing & Missing & Missing \\
\end{tabular}
\imtitle{Utah 4}
\end{minipage}
\bchap{Vermont}
\begin{minipage}{\columnwidth}
\begin{tabular}{lccc}
\raisebox{-0.5\height}{\includegraphics[width=0.32\columnwidth]{tract-5000-full_shrunk.png}} & \raisebox{-0.5\height}{\includegraphics[width=0.32\columnwidth]{blockgroup-5000-full_shrunk.png}} & Missing \\
\raisebox{-0.5\height}{\includegraphics[width=0.32\columnwidth]{tract-5000-net_shrunk.png}} & \raisebox{-0.5\height}{\includegraphics[width=0.32\columnwidth]{blockgroup-5000-net_shrunk.png}} & Missing \\
\end{tabular}
\imtitle{Vermont 0}
\end{minipage}
\bchap{Virginia}
\begin{minipage}{\columnwidth}
\begin{tabular}{lccc}
\raisebox{-0.5\height}{\includegraphics[width=0.32\columnwidth]{tract-5101-full_shrunk.png}} & \raisebox{-0.5\height}{\includegraphics[width=0.32\columnwidth]{blockgroup-5101-full_shrunk.png}} & Missing \\
\raisebox{-0.5\height}{\includegraphics[width=0.32\columnwidth]{tract-5101-net_shrunk.png}} & \raisebox{-0.5\height}{\includegraphics[width=0.32\columnwidth]{blockgroup-5101-net_shrunk.png}} & Missing \\
\end{tabular}
\imtitle{Virginia 1}
\end{minipage}

\begin{minipage}{\columnwidth}
\begin{tabular}{lccc}
\raisebox{-0.5\height}{\includegraphics[width=0.32\columnwidth]{tract-5102-full_shrunk.png}} & \raisebox{-0.5\height}{\includegraphics[width=0.32\columnwidth]{blockgroup-5102-full_shrunk.png}} & Missing \\
\raisebox{-0.5\height}{\includegraphics[width=0.32\columnwidth]{tract-5102-net_shrunk.png}} & \raisebox{-0.5\height}{\includegraphics[width=0.32\columnwidth]{blockgroup-5102-net_shrunk.png}} & Missing \\
\end{tabular}
\imtitle{Virginia 2}
\end{minipage}

\begin{minipage}{\columnwidth}
\begin{tabular}{lccc}
\raisebox{-0.5\height}{\includegraphics[width=0.32\columnwidth]{tract-5103-full_shrunk.png}} & \raisebox{-0.5\height}{\includegraphics[width=0.32\columnwidth]{blockgroup-5103-full_shrunk.png}} & Missing \\
\raisebox{-0.5\height}{\includegraphics[width=0.32\columnwidth]{tract-5103-net_shrunk.png}} & \raisebox{-0.5\height}{\includegraphics[width=0.32\columnwidth]{blockgroup-5103-net_shrunk.png}} & Missing \\
\end{tabular}
\imtitle{Virginia 3}
\end{minipage}

\begin{minipage}{\columnwidth}
\begin{tabular}{lccc}
\raisebox{-0.5\height}{\includegraphics[width=0.32\columnwidth]{tract-5104-full_shrunk.png}} & \raisebox{-0.5\height}{\includegraphics[width=0.32\columnwidth]{blockgroup-5104-full_shrunk.png}} & Missing \\
\raisebox{-0.5\height}{\includegraphics[width=0.32\columnwidth]{tract-5104-net_shrunk.png}} & \raisebox{-0.5\height}{\includegraphics[width=0.32\columnwidth]{blockgroup-5104-net_shrunk.png}} & Missing \\
\end{tabular}
\imtitle{Virginia 4}
\end{minipage}

\begin{minipage}{\columnwidth}
\begin{tabular}{lccc}
\raisebox{-0.5\height}{\includegraphics[width=0.32\columnwidth]{tract-5105-full_shrunk.png}} & \raisebox{-0.5\height}{\includegraphics[width=0.32\columnwidth]{blockgroup-5105-full_shrunk.png}} & Missing \\
\raisebox{-0.5\height}{\includegraphics[width=0.32\columnwidth]{tract-5105-net_shrunk.png}} & \raisebox{-0.5\height}{\includegraphics[width=0.32\columnwidth]{blockgroup-5105-net_shrunk.png}} & Missing \\
\end{tabular}
\imtitle{Virginia 5}
\end{minipage}

\begin{minipage}{\columnwidth}
\begin{tabular}{lccc}
\raisebox{-0.5\height}{\includegraphics[width=0.32\columnwidth]{tract-5106-full_shrunk.png}} & \raisebox{-0.5\height}{\includegraphics[width=0.32\columnwidth]{blockgroup-5106-full_shrunk.png}} & Missing \\
\raisebox{-0.5\height}{\includegraphics[width=0.32\columnwidth]{tract-5106-net_shrunk.png}} & \raisebox{-0.5\height}{\includegraphics[width=0.32\columnwidth]{blockgroup-5106-net_shrunk.png}} & Missing \\
\end{tabular}
\imtitle{Virginia 6}
\end{minipage}

\begin{minipage}{\columnwidth}
\begin{tabular}{lccc}
\raisebox{-0.5\height}{\includegraphics[width=0.32\columnwidth]{tract-5107-full_shrunk.png}} & \raisebox{-0.5\height}{\includegraphics[width=0.32\columnwidth]{blockgroup-5107-full_shrunk.png}} & Missing \\
\raisebox{-0.5\height}{\includegraphics[width=0.32\columnwidth]{tract-5107-net_shrunk.png}} & \raisebox{-0.5\height}{\includegraphics[width=0.32\columnwidth]{blockgroup-5107-net_shrunk.png}} & Missing \\
\end{tabular}
\imtitle{Virginia 7}
\end{minipage}

\begin{minipage}{\columnwidth}
\begin{tabular}{lccc}
\raisebox{-0.5\height}{\includegraphics[width=0.32\columnwidth]{tract-5108-full_shrunk.png}} & \raisebox{-0.5\height}{\includegraphics[width=0.32\columnwidth]{blockgroup-5108-full_shrunk.png}} & Missing \\
\raisebox{-0.5\height}{\includegraphics[width=0.32\columnwidth]{tract-5108-net_shrunk.png}} & \raisebox{-0.5\height}{\includegraphics[width=0.32\columnwidth]{blockgroup-5108-net_shrunk.png}} & Missing \\
\end{tabular}
\imtitle{Virginia 8}
\end{minipage}

\begin{minipage}{\columnwidth}
\begin{tabular}{lccc}
\raisebox{-0.5\height}{\includegraphics[width=0.32\columnwidth]{tract-5109-full_shrunk.png}} & \raisebox{-0.5\height}{\includegraphics[width=0.32\columnwidth]{blockgroup-5109-full_shrunk.png}} & Missing \\
\raisebox{-0.5\height}{\includegraphics[width=0.32\columnwidth]{tract-5109-net_shrunk.png}} & \raisebox{-0.5\height}{\includegraphics[width=0.32\columnwidth]{blockgroup-5109-net_shrunk.png}} & Missing \\
\end{tabular}
\imtitle{Virginia 9}
\end{minipage}

\begin{minipage}{\columnwidth}
\begin{tabular}{lccc}
\raisebox{-0.5\height}{\includegraphics[width=0.32\columnwidth]{tract-5110-full_shrunk.png}} & \raisebox{-0.5\height}{\includegraphics[width=0.32\columnwidth]{blockgroup-5110-full_shrunk.png}} & Missing \\
\raisebox{-0.5\height}{\includegraphics[width=0.32\columnwidth]{tract-5110-net_shrunk.png}} & \raisebox{-0.5\height}{\includegraphics[width=0.32\columnwidth]{blockgroup-5110-net_shrunk.png}} & Missing \\
\end{tabular}
\imtitle{Virginia 10}
\end{minipage}

\begin{minipage}{\columnwidth}
\begin{tabular}{lccc}
\raisebox{-0.5\height}{\includegraphics[width=0.32\columnwidth]{tract-5111-full_shrunk.png}} & \raisebox{-0.5\height}{\includegraphics[width=0.32\columnwidth]{blockgroup-5111-full_shrunk.png}} & Missing \\
\raisebox{-0.5\height}{\includegraphics[width=0.32\columnwidth]{tract-5111-net_shrunk.png}} & \raisebox{-0.5\height}{\includegraphics[width=0.32\columnwidth]{blockgroup-5111-net_shrunk.png}} & Missing \\
\end{tabular}
\imtitle{Virginia 11}
\end{minipage}
\bchap{Washington}
\begin{minipage}{\columnwidth}
\begin{tabular}{lccc}
\raisebox{-0.5\height}{\includegraphics[width=0.32\columnwidth]{tract-5301-full_shrunk.png}} & \raisebox{-0.5\height}{\includegraphics[width=0.32\columnwidth]{blockgroup-5301-full_shrunk.png}} & Missing \\
\raisebox{-0.5\height}{\includegraphics[width=0.32\columnwidth]{tract-5301-net_shrunk.png}} & \raisebox{-0.5\height}{\includegraphics[width=0.32\columnwidth]{blockgroup-5301-net_shrunk.png}} & Missing \\
\end{tabular}
\imtitle{Washington 1}
\end{minipage}

\begin{minipage}{\columnwidth}
\begin{tabular}{lccc}
\raisebox{-0.5\height}{\includegraphics[width=0.32\columnwidth]{tract-5302-full_shrunk.png}} & \raisebox{-0.5\height}{\includegraphics[width=0.32\columnwidth]{blockgroup-5302-full_shrunk.png}} & Missing \\
\raisebox{-0.5\height}{\includegraphics[width=0.32\columnwidth]{tract-5302-net_shrunk.png}} & \raisebox{-0.5\height}{\includegraphics[width=0.32\columnwidth]{blockgroup-5302-net_shrunk.png}} & Missing \\
\end{tabular}
\imtitle{Washington 2}
\end{minipage}

\begin{minipage}{\columnwidth}
\begin{tabular}{lccc}
\raisebox{-0.5\height}{\includegraphics[width=0.32\columnwidth]{tract-5303-full_shrunk.png}} & \raisebox{-0.5\height}{\includegraphics[width=0.32\columnwidth]{blockgroup-5303-full_shrunk.png}} & Missing \\
\raisebox{-0.5\height}{\includegraphics[width=0.32\columnwidth]{tract-5303-net_shrunk.png}} & \raisebox{-0.5\height}{\includegraphics[width=0.32\columnwidth]{blockgroup-5303-net_shrunk.png}} & Missing \\
\end{tabular}
\imtitle{Washington 3}
\end{minipage}

\begin{minipage}{\columnwidth}
\begin{tabular}{lccc}
\raisebox{-0.5\height}{\includegraphics[width=0.32\columnwidth]{tract-5304-full_shrunk.png}} & \raisebox{-0.5\height}{\includegraphics[width=0.32\columnwidth]{blockgroup-5304-full_shrunk.png}} & Missing \\
\raisebox{-0.5\height}{\includegraphics[width=0.32\columnwidth]{tract-5304-net_shrunk.png}} & \raisebox{-0.5\height}{\includegraphics[width=0.32\columnwidth]{blockgroup-5304-net_shrunk.png}} & Missing \\
\end{tabular}
\imtitle{Washington 4}
\end{minipage}

\begin{minipage}{\columnwidth}
\begin{tabular}{lccc}
\raisebox{-0.5\height}{\includegraphics[width=0.32\columnwidth]{tract-5305-full_shrunk.png}} & \raisebox{-0.5\height}{\includegraphics[width=0.32\columnwidth]{blockgroup-5305-full_shrunk.png}} & Missing \\
\raisebox{-0.5\height}{\includegraphics[width=0.32\columnwidth]{tract-5305-net_shrunk.png}} & \raisebox{-0.5\height}{\includegraphics[width=0.32\columnwidth]{blockgroup-5305-net_shrunk.png}} & Missing \\
\end{tabular}
\imtitle{Washington 5}
\end{minipage}

\begin{minipage}{\columnwidth}
\begin{tabular}{lccc}
\raisebox{-0.5\height}{\includegraphics[width=0.32\columnwidth]{tract-5306-full_shrunk.png}} & \raisebox{-0.5\height}{\includegraphics[width=0.32\columnwidth]{blockgroup-5306-full_shrunk.png}} & Missing \\
\raisebox{-0.5\height}{\includegraphics[width=0.32\columnwidth]{tract-5306-net_shrunk.png}} & \raisebox{-0.5\height}{\includegraphics[width=0.32\columnwidth]{blockgroup-5306-net_shrunk.png}} & Missing \\
\end{tabular}
\imtitle{Washington 6}
\end{minipage}

\begin{minipage}{\columnwidth}
\begin{tabular}{lccc}
\raisebox{-0.5\height}{\includegraphics[width=0.32\columnwidth]{tract-5307-full_shrunk.png}} & \raisebox{-0.5\height}{\includegraphics[width=0.32\columnwidth]{blockgroup-5307-full_shrunk.png}} & Missing \\
\raisebox{-0.5\height}{\includegraphics[width=0.32\columnwidth]{tract-5307-net_shrunk.png}} & \raisebox{-0.5\height}{\includegraphics[width=0.32\columnwidth]{blockgroup-5307-net_shrunk.png}} & Missing \\
\end{tabular}
\imtitle{Washington 7}
\end{minipage}

\begin{minipage}{\columnwidth}
\begin{tabular}{lccc}
\raisebox{-0.5\height}{\includegraphics[width=0.32\columnwidth]{tract-5308-full_shrunk.png}} & \raisebox{-0.5\height}{\includegraphics[width=0.32\columnwidth]{blockgroup-5308-full_shrunk.png}} & Missing \\
\raisebox{-0.5\height}{\includegraphics[width=0.32\columnwidth]{tract-5308-net_shrunk.png}} & \raisebox{-0.5\height}{\includegraphics[width=0.32\columnwidth]{blockgroup-5308-net_shrunk.png}} & Missing \\
\end{tabular}
\imtitle{Washington 8}
\end{minipage}

\begin{minipage}{\columnwidth}
\begin{tabular}{lccc}
\raisebox{-0.5\height}{\includegraphics[width=0.32\columnwidth]{tract-5309-full_shrunk.png}} & \raisebox{-0.5\height}{\includegraphics[width=0.32\columnwidth]{blockgroup-5309-full_shrunk.png}} & Missing \\
\raisebox{-0.5\height}{\includegraphics[width=0.32\columnwidth]{tract-5309-net_shrunk.png}} & \raisebox{-0.5\height}{\includegraphics[width=0.32\columnwidth]{blockgroup-5309-net_shrunk.png}} & Missing \\
\end{tabular}
\imtitle{Washington 9}
\end{minipage}

\begin{minipage}{\columnwidth}
\begin{tabular}{lccc}
Missing & Missing & Missing \\
Missing & Missing & Missing \\
\end{tabular}
\imtitle{Washington 10}
\end{minipage}
\bchap{West Virginia}
\begin{minipage}{\columnwidth}
\begin{tabular}{lccc}
\raisebox{-0.5\height}{\includegraphics[width=0.32\columnwidth]{tract-5401-full_shrunk.png}} & \raisebox{-0.5\height}{\includegraphics[width=0.32\columnwidth]{blockgroup-5401-full_shrunk.png}} & Missing \\
\raisebox{-0.5\height}{\includegraphics[width=0.32\columnwidth]{tract-5401-net_shrunk.png}} & \raisebox{-0.5\height}{\includegraphics[width=0.32\columnwidth]{blockgroup-5401-net_shrunk.png}} & Missing \\
\end{tabular}
\imtitle{West Virginia 1}
\end{minipage}

\begin{minipage}{\columnwidth}
\begin{tabular}{lccc}
\raisebox{-0.5\height}{\includegraphics[width=0.32\columnwidth]{tract-5402-full_shrunk.png}} & \raisebox{-0.5\height}{\includegraphics[width=0.32\columnwidth]{blockgroup-5402-full_shrunk.png}} & Missing \\
\raisebox{-0.5\height}{\includegraphics[width=0.32\columnwidth]{tract-5402-net_shrunk.png}} & \raisebox{-0.5\height}{\includegraphics[width=0.32\columnwidth]{blockgroup-5402-net_shrunk.png}} & Missing \\
\end{tabular}
\imtitle{West Virginia 2}
\end{minipage}

\begin{minipage}{\columnwidth}
\begin{tabular}{lccc}
\raisebox{-0.5\height}{\includegraphics[width=0.32\columnwidth]{tract-5403-full_shrunk.png}} & \raisebox{-0.5\height}{\includegraphics[width=0.32\columnwidth]{blockgroup-5403-full_shrunk.png}} & Missing \\
\raisebox{-0.5\height}{\includegraphics[width=0.32\columnwidth]{tract-5403-net_shrunk.png}} & \raisebox{-0.5\height}{\includegraphics[width=0.32\columnwidth]{blockgroup-5403-net_shrunk.png}} & Missing \\
\end{tabular}
\imtitle{West Virginia 3}
\end{minipage}
\bchap{Wisconsin}
\begin{minipage}{\columnwidth}
\begin{tabular}{lccc}
Missing & Missing & Missing \\
Missing & Missing & Missing \\
\end{tabular}
\imtitle{Wisconsin 1}
\end{minipage}

\begin{minipage}{\columnwidth}
\begin{tabular}{lccc}
Missing & Missing & Missing \\
Missing & Missing & Missing \\
\end{tabular}
\imtitle{Wisconsin 2}
\end{minipage}

\begin{minipage}{\columnwidth}
\begin{tabular}{lccc}
Missing & Missing & Missing \\
Missing & Missing & Missing \\
\end{tabular}
\imtitle{Wisconsin 3}
\end{minipage}

\begin{minipage}{\columnwidth}
\begin{tabular}{lccc}
Missing & Missing & Missing \\
Missing & Missing & Missing \\
\end{tabular}
\imtitle{Wisconsin 4}
\end{minipage}

\begin{minipage}{\columnwidth}
\begin{tabular}{lccc}
Missing & Missing & Missing \\
Missing & Missing & Missing \\
\end{tabular}
\imtitle{Wisconsin 5}
\end{minipage}

\begin{minipage}{\columnwidth}
\begin{tabular}{lccc}
Missing & Missing & Missing \\
Missing & Missing & Missing \\
\end{tabular}
\imtitle{Wisconsin 6}
\end{minipage}

\begin{minipage}{\columnwidth}
\begin{tabular}{lccc}
Missing & Missing & Missing \\
Missing & Missing & Missing \\
\end{tabular}
\imtitle{Wisconsin 7}
\end{minipage}

\begin{minipage}{\columnwidth}
\begin{tabular}{lccc}
Missing & Missing & Missing \\
Missing & Missing & Missing \\
\end{tabular}
\imtitle{Wisconsin 8}
\end{minipage}
\bchap{Wyoming}
\begin{minipage}{\columnwidth}
\begin{tabular}{lccc}
\raisebox{-0.5\height}{\includegraphics[width=0.32\columnwidth]{tract-5600-full_shrunk.png}} & \raisebox{-0.5\height}{\includegraphics[width=0.32\columnwidth]{blockgroup-5600-full_shrunk.png}} & Missing \\
\raisebox{-0.5\height}{\includegraphics[width=0.32\columnwidth]{tract-5600-net_shrunk.png}} & \raisebox{-0.5\height}{\includegraphics[width=0.32\columnwidth]{blockgroup-5600-net_shrunk.png}} & Missing \\
\end{tabular}
\imtitle{Wyoming 0}
\end{minipage}
\bchap{American Samoa}
\begin{minipage}{\columnwidth}
\begin{tabular}{lccc}
Missing & Missing & Missing \\
Missing & Missing & Missing \\
\end{tabular}
\imtitle{American Samoa 98}
\end{minipage}
\bchap{Guam}
\begin{minipage}{\columnwidth}
\begin{tabular}{lccc}
Missing & Missing & Missing \\
Missing & Missing & Missing \\
\end{tabular}
\imtitle{Guam 98}
\end{minipage}
\bchap{Commonwealth of the Northern Mariana Islands}
\begin{minipage}{\columnwidth}
\begin{tabular}{lccc}
Missing & Missing & Missing \\
Missing & Missing & Missing \\
\end{tabular}
\imtitle{Commonwealth of the Northern Mariana Islands 98}
\end{minipage}
\bchap{Puerto Rico}
\begin{minipage}{\columnwidth}
\begin{tabular}{lccc}
\raisebox{-0.5\height}{\includegraphics[width=0.32\columnwidth]{tract-7298-full_shrunk.png}} & \raisebox{-0.5\height}{\includegraphics[width=0.32\columnwidth]{blockgroup-7298-full_shrunk.png}} & Missing \\
\raisebox{-0.5\height}{\includegraphics[width=0.32\columnwidth]{tract-7298-net_shrunk.png}} & \raisebox{-0.5\height}{\includegraphics[width=0.32\columnwidth]{blockgroup-7298-net_shrunk.png}} & Missing \\
\end{tabular}
\imtitle{Puerto Rico 98}
\end{minipage}
\bchap{U.S. Virgin Islands}
\begin{minipage}{\columnwidth}
\begin{tabular}{lccc}
Missing & Missing & Missing \\
Missing & Missing & Missing \\
\end{tabular}
\imtitle{U.S. Virgin Islands 98}
\end{minipage}


\newpage

\printindex


\end{document}